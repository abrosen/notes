\documentclass[10pt, a4paper]{article}
\usepackage[latin1]{inputenc}
\usepackage{amsmath}
\usepackage{amsfonts}
\usepackage{amssymb}
\usepackage{graphicx}
\usepackage[left=1.00in, right=1.00in, top=1.00in, bottom=1.00in]{geometry}
\author{Andrew Rosen}
\title{Teaching Philosophy}
\date{}
\begin{document}
\maketitle
	
I consider teaching Computer Science to be the most enjoyable and rewarding thing I do.
Computer Science itself has the privilege of simultaneously being both an old and new subject.
Computer Science is as old as mathematics and algorithms, much like astronomy is as old as the first human who decided to look at the stars.
Our subject is also new, as computers are a relatively recent invention in human history and the formal study of algorithms only truly began in the previous century.


Computer Science is a large field with many applications and that makes it particularly exciting both to learn and teach.
I often tell students that I chose Computer Science for my undergraduate degree because I had no idea what I wanted to do, but I knew that Computer Science and programming would give me the power to choose.
If I wanted to work for a large business, I could.
If I wanted to do scientific research in Physics or Biology or any of the hard sciences, Computer Science would give me the tools to be successful.
This is why I love teaching Computer Science:  I can engage with a wide variety of students, each with unique backgrounds and widely varying aspirations.




I believe Computer Science is a particularly difficult subject for students for a number of reasons.
The first factor is that Computer Science is a completely new and foreign field for the vast majority of students.
Computer Science courses most closely resemble mathematics courses, but this is a result of how much math CS draws upon.

In my high school, our music director stated that of all the subjects taught in school, it was only music that demanded 100\% perfection.
The musicians must hit the correct notes, and only the correct notes, and only at the correct times.
No other subject, he contended, demands absolute perfection from their students.

As Computer Science teachers, we know this is not necessarily true. 
We see it in the assignments we hand out, where a single missing character by a student is the difference between a success and a failure, a 100 or a 0.
Partial credit is minimal at best, because all that matters at the end of the day are the \textit{results}, not the effort.

This reason can make computer sscience a particularly dautingg subject to new students, but I find there is another stronger reason why Computer Science is veiwed as  a particulay  challenging subject.  
T
Computer Science challenges students to use new paradigms for solving problems, to think in a completely different manner.



I believe the best way to learn programming is not by watching someone lecture about code using a whiteboard, but by actually programming.
Thus, I live-code in class.
This achieves a number of objectives.

One of the reasons that live coding is not always done is because of the potential for errors.
However, I believe these errors are a valuable part of the learning process.
Xth, by making mistakes, intentionally or unintentionally, I help students become accustomed to the debugging process they will have to do themselves.
I believe that seeing an ostensible expert make mistakes and correct them makes coding much less intimidating.
As the course progresses, students become more comfortable with correcting any mistakes they see.




I judge my success in having students learn the material when they feel comfortable enough to ask and  answers questions in class and get over the fear of embarrassment.


Balaance student use of technol

\end{document}