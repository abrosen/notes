\documentclass[]{letter}

\usepackage{amsmath}
\usepackage{amsfonts}
\usepackage{amssymb}
\usepackage{graphicx}
\usepackage{tabularx}
\usepackage{algorithm}
\usepackage{cjhebrew}
%\usepackage{venturis2}
\begin{document}
% If you want headings on subsequent pages,
% remove the ``%'' on the next line:
% \pagestyle{headings}

\begin{letter}{Somewhere }
\address{Rosen Manor}

\opening{To my descendants,}


Greetings from the year 2017.
I hope this message finds you well.
My son, Eleazer Nathaniel Rosen, just turn 8 months old a few days ago and a few recent events have persuaded me to embark on a project I wanted shortly before he was born.
When Lazer was born, I saw my parents (Stephanie and Steven Rosen) looking on at their grandchild, I realized that this would most likely happen to me someday, and someday to my children and my children's children and I think you can do the rest this yourself.

And it made me slightly sad.
All these amazing people, and I'll most likely only get to really know a few of them.
I have many flaws, but one of my big ones is that I always want more time.
There's never enough of it to got around, and certainly not enough of it to meet and love and teach everyone in a family tree of potentially infinite depth.


Then I remembered I have a time machine.
It's not a particularly good one, I admit. 
It only goes in one direction, and it only takes \textbf{me} so far, but anything I write has a good chance of traveling farther, especially in this day and age where the Internet has really turned into something amazing and fault tolerant.\footnote{For those of you happen to have decided to foolishly join me in studying Computer Science, this fault tolerant thing is actually something I got tangentally involved in.  Check out my research and those of my coauthors from slightly before this year. }

It turns out to be the same one that everyone else has, but I can use it to send you all sorts of things (there's plenty of room):
letters, some videos about what I know, lectures, congratulations, advice, encouragement, trust, and, most importantly, knowledge.

%Knowledge is the most important thing I can give you, so before I close off this very first letter, let me give you a few irreverent maxims.
%
%\begin{itemize}
%	\item It's probably not a good idea to live in the vicinity of an active fault line (see San Andreas and the Cascadia subduction zone).
%	\item Seeing the Grand Canyon in person is worth it.  Pictures just won't do it justice (at least until VR gets really, really good.)
%	\item Pulling an all-nighter is self-defeating,
%	\item Murphy's Law states that whatever can go wrong, will go wrong.  You can, however, plan accordingly. For example, I've always told my students 
%\end{itemize}
%

\signature{Professor Andrew Benjamin Rosen}

\closing{Your Ancestor (or possibly crazy relation up the tree) }

%enclosure listing
%\encl{}


\begin{cjhebrew}
	.hnwK
\end{cjhebrew}
\end{letter}
\end{document}
