%TODO: godfrey2005heterogeneity
\chapter{Proposed Work}
\label{chapter:experiments}


\section{UrDHT: Our DHT Framework}
As mentioned in Chapter \ref{chapter:intro}, our plan is to create a highly configurable and easy to use DHT framework based off the DHT abstractions we have discovered.
Rather than making a fully-functional DHT application on our own, we will be making a minimally functional DHT framework that will be easy to fork for a variety of applications.

We call this proposed framework UrDHT, \textit{ur-} being the Germanic prefix denoting primal, or primitive, or original.
UrDHT is a project which presents a minimal and extensible implementation for all the essential components for a DHT: the different aspects for a protocol definition, the storage of values, and the networking components.


\begin{itemize}
	\item This will be done jointly with Brendan and anyone else who is interested as a completely open source.
	\item The language used will either be Python3 and/or Go.
	\begin{itemize}
		\item Python3 is easy to read and write and has support for coroutines.  We know Python, but it does not do type checking and can be more dangerous.
		\item Go is safer than Python and is designed for concurrency. It would be safer than Python.
		\item Regardless of the language, we will be using test driven design.
	\end{itemize}
	\item The goal of this step is \textbf{not} to create a DHT, but to create an easily extensible abstract framework for DHTs.
	\item The abstraction comes from implementing the relationship we found between DHT spheres of responsibility and Voronoi tessellations.  
	This is the exciting part and novel of the project.
	Our previous research \cite{vhash} has led us to assert that there is a mathematical formulation for different aspects which every DHT shares in common, such as a distance metric and closeness definition.
	
\end{itemize}

\subsection{Expected Publications}
We expect a pair of publications: one from the mathematical definitions for distributed hash tables and one from the UrDHT application itself.
The UrDHT project is probably the stronger publication, since it is a fully fledged and novel framework.
Furthermore, as an open-source application, we hope that it will be forked by many developers who wish to build applications that rely on DHTs.

The mathematical formulations, on the other hand, serve as novel formulations and definitions of DHTs.
They provide new insight, but do not serve as a new application or framework.
However, the formulations can and should be presented as an atomic unit on their own.


\section{Distributed DHT Computing}

The next step is to use the UrDHT framework to re-implement ChordReduce.
Our goal is a DHT based platform for solving embarrassingly parallel problems using DHTs.
The steps involved in this are listed below.
\begin{itemize}
	\item We will use UrDHT to implement a few of the more popular DHTs.
	\begin{itemize}
		\item We want to compare each of the DHTs to see if there is a difference between using one or another for distributed computing.
		\item Using UrDHT for all the implementations will minimize the non-protocol differences between each DHT, which will allow for as fair a comparison as possible.
		\item This will serve as both examples for home to implement our framework as well and the building blocks for the next step. 
	\end{itemize}
	\item Implement a distributed computing mechanism on each of the implemented DHTs for solving computing tasks.
	\begin{itemize}
		\item The emphasis of our distributed computing application is robustness and fault-tolerance.
	\end{itemize}
	\item Test each application using a variety of embarrassingly parallel problems, such as:
	\begin{itemize}
		\item Brute-force cryptanalysis.
		\item MapReduce problems.
		\item Monte-Carlo computations.
	\end{itemize}
\end{itemize}

\subsection{Expected Publications}
Here there is one very clearly well defined application and publication.
This would be a reimplementation of our original ChordReduce paper with a completely new problems and new results.


\section{Autonomous Load Balancing}

As we discussed multiple times, it would be highly beneficial to take advantage of each 

The most straightforward way of doing this in a DHT context is to have more powerful node acquire a wider range

The goals here are straightforward:

\begin{itemize}
	\item Further establish that the ``strategy'' of randomly inducing churn works at reducing.
	If we successfully establish it exists,  we must determine if the process can be modeled.%stochastically modeled or otherwise predicted via theoretical analysis.
	
	\item Create a  processor scoring mechanism for creating virtual nodes.
	\begin{itemize}
		\item This score is a relative score, since the amount of work a node can carry is determined by the amount other nodes can carry.
		\item Furthermore, the score should also be a function of the node's range, since the amount of work is dictated by the amount of range ``owned'' the node. 
		This means creating the first virtual node does not double the amount of work the node has.
		\item This is compared to the scores of neighboring nodes  of all the node's replicas and should give a gauge of the node's relative  power.
	\end{itemize}
	\item Use this score in conjunction with various implemented strategies for autonomous load-balancing to find the most effective.
	A few brainstormed strategies:
	\begin{itemize}
		\item Passive load balancing -  Here, each node looks at range of its Sybil locations.
		Based on its score, it choses the $k$ largest locations and injects Sybils.
		No actual communication with other nodes is used.
		\item IRM \cite{irm} based  strategy.
		\item Invitation based -  Here we flip to strategy around.
		If the node detects its region is too large or it has too much work; it invites other nodes to help.
		The nodes look at the range in question and offer to help if they have a Sybil that fits and are not overloaded themselves.
		The invited looks at the offers of help and selects the best candidate.
	\end{itemize}
	\item This may necessitate new node types, such as tracker nodes which exist only to eavesdrop on traffic loads.
	%\item We also plan on modifying virtual nodes so they do not take on additional routing c.  
	%We want to see how this change will impact the system.
	\item We must determine under which contexts this kind of highly responsive load-balancing can be used. 
	Is it only useful during distributed computation, or is it useful in file-sharing as well?
	
	
\end{itemize}


\subsection{Expected Publications}
We can expect that our research in this area would lead to a publication on autonomous load-balancing.
In it, we would compare various strategies we implemented against each other, using the randomly induced churn as a baseline.
In addition, there may be some traditional load-balancing strategies, which we would use to establish the strength of our strategies.



