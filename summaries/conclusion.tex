\chapter{Conclusion}
\label{chapter:conclusion}
Distributed Hash Tables are extremely powerful frameworks for distributed applications that are based off the simple and powerful hash tables.
Because DHTs were designed with P2P applications in mind, DHTs are scalable, fault-tolerant, and load-balancing.
These are exactly the qualities needed in a distributed computing framework.

We have shown in the previous chapters that we were able to create a prototype distributed computing application \cite{chordreduce} based on the Chord DHT \cite{chord}.
ChordReduce gave us several new questions to ask, such as does it matter which DHT is used for distribute computation and how can we have nodes autonomously accept new work from overloaded nodes?
To effectively answer these questions we need to create a new framework for creating DHT applications, which we dub UrDHT.

UrDHT will be an invaluable resource to any other developer who wishes to create a DHT application.
By implementing a distributed computing application using UrDHT, we will create a completely decentralized framework for doing distributed computing.
This will allow distributed computing to take place not just in data centers, but within  P2P context as well.
