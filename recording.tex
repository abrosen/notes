\documentclass[sigconf]{acmart}
%\usepackage{cite}
%opening
\title{Recording Your Lecture Fun and (Your Student's) Profit}
\author{Ponder Stibbons}
\affiliation{\institution{Unseen University}}
\email{ps@uu.discworld.edu}


\begin{abstract}
	Recording your lectures may seem overwhelming, but in reality is actually quite easy to do.  
	This paper serves as a guide to do so.
	We explore the benefits to students and instructors for recording lectures, such as review material, mitigating illnesses and outbreaks, and future use for flipped classrooms. 
	We discuss the software options available to record lectures and weigh the benefits and consequences of various hosting options.
\end{abstract}
\begin{document}

\maketitle



\section{Introduction}


This paper advocates that every Computer Science teacher record their lectures.

Recording your lectures has many benefits for the students, providing them with a way to supplement their studies, makeup lectures they missed due to illness \cite{traphagan2010impact}, and reduce they stress and work involved in note-taking
In addition, the instructor can benefit from the recordings after a implementing them for at least a semester, such as using previous semester's work to experiment with flipped classes, creating a portfolio, and reviewing your work. 

However, one strong reason that instructors may choose to avoid recording their lecture is the perceived additional overhead and work required \cite{maynor2013student}.
Besides the actual act of recording a lecture, an instructor needs to somehow make it available to the students, which requires finding a service to host the lecture, not to mention the need to setup and learn how to use the lecture recording software.
These only some of the difficulties an instructor my think of when considering how to create recordings.
The goal of this paper is to provide a simple guide to eliminate that uncertainty and overhead that exists around recording your lectures for the first time.




%Availability of recorded lectures is associated withlowerfinalgrades(Fernandes,Maley&Cruickshank,2008),higherfinalgrades(Traphagan et al,2010)orwithnosignificantdifferenceingrades(Babb&Ross,2009;Brotherton&Abowd, 2004; Wieling & Hofman, 2010), which is rather incovient 

\begin{itemize}
	\item In Section \ref{why}, we discuss the motivations for this paper and provide an overview of the literature studying the impacts on students as a result of recording lectures.  Much of the focus of the cited research focused on the effects on attendance and student performance.
	\item We present the benefits and costs of using an enterprise option, such as Panopto  or echo360,  to handle the recording and hosting needs for a course in Section \ref{enter}.
	\item Section \ref{opensource} covers Open Broadcaster Software Studio (OBS), an extremely popular recording and broadcasting software.
	\item Section \ref{hardware} covers both the baseline hardware requirements, as well as optional equipment to help with your recording.
	\item Section \ref{edit} discusses the option of editing your videos, as well as some basic techniques to make your edits look cleaner and more professional.
	\item Section \ref{hosting} compares and contrasts the three main options for hosting your videos: YouTube, Vimeo, or on the cloud or a server.
	\item Section \ref{implementation} covers our own personal implementation of the advice and technologies we present here.
	\item Section \ref{legal} talks about the common licences you may see with, as well as some legal issue to be aware of.
\end{itemize}
\section{Motivation}
\label{why}

%We began to record our lectures in the Fall 2016 semester in order serve students who were taking an online version of this course on a satellite campus.
%The class was canceled due to logistical reasons, but well after we had already laid the groundwork for recording our lectures.
%We had not previously recorded our lectures, but after getting our software setup, we implemented recording in all of courses.
%The stude


Recording lectures has been talking about for the past few decades, and research into the effects has been going on for almost just as long.
However, the recording all of a professor's lecture in high fidelity is something that has become possible only in recent years.
This is dues to a number of factors: multiple companies

Modern laptops come with at least 1080p screens, which translates to a crisp high-resolution screen capture, and have popowerful processors able to handle both recording 

The emergence of codecs such as AV1 \cite{AV1comp} (not to be confused with .avi files) means that even as storage becomes more cheaper and plentiful, high quality files are getting smaller and smaller, with less restrictive licenses. 
Consumer cameras get better every year and can record in UHD (2160p resolutions or greater).



Larkin but they wontcome to class restiance to academic \
fear class may be optional
Students would still come to class and the recordings were used to aid in studying or to make up a lecture that was missed.




%Thus,competence or otherwise in relation to technology specifically, cannot necessarily be explained through generational attributes alone.


%Anecdotally, a common response by staff to demands or requests to incorporate these
%online approaches into their teaching and learning technologies, is a concern that if
%recorded lectures are made available, students will opt out of attending lectures. A
%large scale study by Gosper, Green, McNeill, Phillips, Preston and Woo (2008)
%confirmed this is a common concern amongst teaching staff in Australian universities.
%It could be argued that this concern reflects a ‘Level 1’ theory of teaching as described
%by Biggs and Tang (2007), that views teaching as purely the transmission of knowledge
%and a lecture as merely the vehicle for delivering information. This ‘sage on the stage’
%perspective supports a flawed belief that “the fundamental problems in the quality of
%university education can be solved by transferring knowledge more efficiently, using
%some form of information technology” (Ramsden, 2003, p. 108).
%As discussed by Ramsden (2003) and others, these views don’t recognise the
%interactive nature of face to face teaching and a need to focus on what the student does
%and needs to know that is consistent with contemporary views of teaching, outlined
%for example by Biggs and Tang (2007). As discussed by McGarr (2009), the often
%passive role ascribed to students in lectures is contrary to these contemporary views.
%While both critics and supporters of lectures continue to argue their relative
%advantages and disadvantages, the traditional lecture continues to be a dominant
%method for the delivery of teaching and learning in higher education (McGarr, 2009;
%Williams & Fardon, 2007). Several authors (McGarr, 2009; Taylor, 2009; Lazzari &
%Betella as cited in McGarr, 2009) argue the need to investigate the impact of lecture
%recordings on lecture attendance specifically, and teaching and learning in general. As
%described by McGarr (2009) “questions linger in relation to its true educational value,
%the ways it can best be utilised to support teaching and learning, and its affect on
%attendance and student engagement” (p. 312). This question is addressed in an
%Australian study by von Konsky, Ivins and Gribble (2009) who found that the act of
%making lectures available online did not have a significant impact on student
%attendance at lectures. This study recorded week by week attendance at lectures, the
%frequency of downloading of lectures and student perceptions of their use.



%A comparison of student perceptions and usage patterns
%Brian R. von Konsky, Jim Ivins and Susan J. Gribble
%Curtin University of Technology
%This paper investigates the impact of web based lecture recordings on learning and
%attendance at lectures. Student opinions regarding the perceived value of the
%recordings were evaluated in the context of usage patterns and final marks, and
%compared with attendance data and student perceptions regarding the usefulness of
%lectures. The availability of recordings was not seen to impact lecture attendance,
%although students showed some tendency to listen to the recording for a missed
%lecture. Students who achieved a high mark tended to supplement lecture attendance
%with recording usage more than students who achieved a low mark, but they did so
%with greater variation. If students perceived that a learning experience was of value to
%their learning, they were more likely to use it. Individual case studies describing
%perceptions, usage patterns, and attendance records of selected students highlight the
%fact that there is great variation in successful learning patterns, and suggest that
%engagement is an important factor impacting learning. Although the use of recordings
%to supplement lectures was seen to enhance the learning of some students, its uptake
%and effectiveness was not uniform across the cohort. This observation highlights the
%need for a range of learning modes in engineering education, appealing to a diverse
%set of individual learning styles. Future work is described in the context of these
%findings.
\subsection{Student Reference and study aid}


Ease of review
eases Note taking pressure


One study \cite{shimoff2001effects}  on an introductory Psychology course found that quiz grades increased by about a letter grade with the lectures recorded compared to the same course without recordings.

catch up if you fall behind \cite{young2008lectures}

Attention span is not as bad as you think it is; 10-15 minute limit overblown \cite{bradbury2016attention} \cite{wilson2007attention}

One study reported ~80\% found recordings helpful \cite{maynor2013student}


Disabilty resources

\subsubsection{Effect on Grades}
Besides attendance, instructors are concerned with effect lecture recordings will have on grades \cite{maynor2013student}.
There is, again, an natural assumption that can be made and tested: since
However, just like with attendance, studies have been potentially contradictory.

Better knowledge base, but not better analysis or other critical thinking.\cite{bos2016use}



No significant impact on student learning outcomes compared to students who don't use the recordings \cite{leadbeater2013evaluating}.
Poorer outcomes for students \cite{johnston2013digital}
More viewing= better outcomes 
\cite{traphagan2010impact}
\subsection{Regular Illnesses and Outbreaks}
Our students are coming under more and more pressure to succeed.  
Enrollments in computer science are up across all universities, which means more students in a class. 

Missing class has a negative impact on student performance \cite{traphagan2010impact}.
These increases in students and a drop in vaccination rates accross the US have lead to greater concern about outbreaks at numerous universities such as measles outbreak at UCLA \cite{uclameas} and a mumps outbreak at Temple University \cite{emezienna2019resurgence} in 2019.

Sometimes, the only 

These outbreaks are in addition to regular infections, like influenza. 


Sick students have to make a choice. 
The first choice is miss the lecture and try to make up the missed content by getting notes from a classmate or reading the text.
However, lecture notes might miss some of the content in the class, and the instructor may cover material that was not presented in class.

%https://www.cdc.gov/meningococcal/outbreaks/index.html
%https://web.archive.org/web/20190604160258/https://www.cdc.gov/meningococcal/outbreaks/index.html
%https://www.tandfonline.com/doi/abs/10.1080/00039896.1962.10663216?journalCode=vzeh20
%https://academic.oup.com/jpubhealth/article/22/4/492/1527001
%https://jamanetwork.com/journals/jamapediatrics/article-abstract/380566
%https://www.ncbi.nlm.nih.gov/pmc/articles/PMC4577933/







\subsection{Effects on Attendance}
One often-voiced concern instructors have about recording lectures is that it may create an incentive for students to skip class \cite{larkin2010but} \cite{young2008lectures}.


Students don't replace lectures with online recordings.\cite{larkin2010but}

Studies on this have been mixed and somewhat contradictory \cite{bos2016use}


Recording lectures has been found to have no negative effects on attendance, if not increased effects on attendance \cite{shimoff2001effects}


Students say they are less likely to attend class, but a very small minority (~10\%) reported they would use it as a substitute \cite{maynor2013student}

Negative impact on lectures, but not nearly as much as having some other resources, such as slides, available for download 
\cite{traphagan2010impact}.

Anecdotally, we have not seen a drop in attendance that 

\subsubsection{Ensuring Attendance}
If, for some reason, students do stop attending or you want to ensure students will keep attending, there are a few suggested steps you can take \cite{larkin2010but} \cite{young2008lectures}:
\begin{itemize}
	\item Create more active learning exercises.
	
	\item Take attendance and make it part of the grade, however small.
	\item Use short in-class quizzes.
	\item Wait a number of days before making the recording accessible.
	\item Let students know you will stop recording if there's no attendance. 
\end{itemize}

Regardless, we suggest that for your first time recording, a simple approach that requires less work will yield best results.  
After that, make incremental changes throughout the semester.

\footnote{Also, ensuring that the course is not held at 8am or some other early hour will be more likely to ensure attendance.}

\subsection{Perceived Barriers to Adoption}
Attendance is greatest concern,\cite{maynor2013student}
Otehr concerns include decreaded performance, not viewed as appriopate, concerns about technology.\cite{maynor2013student}
I'm scared of being on camera because I mess up.

This looks like too much work!

Accessibility concerns (see above about too much work, but just ask your drs.  These are supplemental resources, they are not required.  Speak with your students).

Legal concerns (punt, ask your chair, or host on drive.)


\subsection{Our View}
Despite the threat of reduced attendance and students replacing the actual lectures with recordings, we strongly advocate that professors record their lectures.


Aid to students
Having these lectures mitigates a number of random threats students can face during a semester, such as falling ill during a week of critical content.
Having lectures available mitagted the impact of absences on student grades\cite{traphagan2010impact}.
Reduced student stress \cite{traphagan2010impact}.

Hepful to students with impairments 
Helpful to ESL students


Just because quantitative differences in learning can't be shown doesn't mean students don't find it helpful.


OER engagement \cite{llamas2014generating}


It is not only to the benefit to the the students, but an instructor

Illness of instructor, 
Experiment with flipped class room even for a lesson or two.

Portfolio creation

Tech shouldn't be a concern; largest concern is plugging in the laptop (which can be finicy)

\section{Enterprise Options}
\label{enter}
I've listed this first because this is probably the most straightforward option.

Your university, college, or department might already have a license or contract with an enterprise solution.
Companies such has Panopto or Echo360 offer a "one stop shop" hosting your data AND recording.


Editor on cloud means their computers do the rendering if you make edits

\subsection{Cons}

However, there are drawbacks.
First, files may be proprietary format, thus requiring an extra step to edit
Provided Editor may not be strong or easy to use or have the features you want
may not be able to control offline content precisely enough


\section{Open Source Recording}
\label{opensource}
The current best open source option for both recording and streaming (transmitting video live) is OBS Studio. (cite image)

OBS Studio is an extremely configurable, cross-platform recording program that is currently used by many streamers and has popular usage on Twitch, a video game streaming platform that brings in over one million viewers daily.  (image of hbomberguy or eu4 on twitch, example live stream )

It also has a number of filters built in, including a  ``Chroma Key'' filter, which can be used to remove a green screen if one is used (more on this later).



\section{Required Hardware and Equipment}
\label{hardware}
A modern laptop is all that is required.


%\subsubsection{Example laptops}
%\subsubsection{Specs}






\subsection{Additional Equipment}
Additional equipment costs up to around \$200, with tax, depending on the needs.

For example, I used have used a good quality microphone (\$100), a 1080p 60fps webcam (\$70), and a foldable greenscreen (\$40), to produce high qulality video recordings at home


\subsection{Concerns}
\begin{itemize}
	\item Your hardware must be good. Using an industrial strength IDE such as IntelliJ or Eclipse and any recording software can be taxing to older laptops.
	There are a few  ways to mitigate this  if getting a better laptop is problematic. 
	First is to have the IDE open before recording.
	Alternatively, migrate away from Intellij, Eclipse, and the like and use a more lightweight editor, such as Sublime Text, Atom, Visual Studio Code, Geany, or the old classics of Vim and Emacs.
	Another fix is to change the recording encoding setting to perform less encoding during recording; the result of this is that the file sizes will be higher, but it will be less taxing on the CPU.
	
	High end laptops (\$1000+ laptops with GTX 1050 graphics cards or better) can make use of hardware encoding, such as NVENC, which uses the graphics card to do the heavy lifting of the recording, which frees up a lot of processing power for the computer (INSERT PIC HERE).
	
	
	If you are just recording powerpoint slides or capturing the board with the webcamera, you should run into no difficulties.
	
	I also highly recommend having your laptop plugged in  so that you get maximum performance.
	\item Student questions can't be heard by your mic.  This is both a good thing, as it will alleviate privacy concerns of students, but students  reviewing the recording will be missing some information.  Thus, get into the habit of repeating and summarizing student queries when recording.  This will also help in large lectures, where the students present in the lecture won't necessarily hear the question either.
	
\end{itemize}

\section{Editing}
\label{edit}

\section{Hosting}
\label{hosting}

In this section, we explore and look at the benefits of different solutions to host your videos.  
This sections assumes that you are not using an enterprise option, such as Panopto or echo360, as that service provides a hosting solution with it.
The three primary choices an instructor has are YouTube, Vimeo, or storing your files on a server.

Other popular options exist, such as Twitch.tv, but are not used often by academia.

\subsection{YouTube}
One of the most well known websites in the world, YouTube is nearly synonymous with video hosting and streaming, with good reason.
YouTube reports users watching a billion hours of video each day.\cite{ytstats}


Choosing YouTube as your hosting platform has a number of inherent benefits.
First and foremost, making sure your videos are always accessible is now the job of one of the biggest players on the Internet.
Videos are stored in Google datacenters\footnote{As we're not privy to the internal organization of datacenters within Alphabet, we'll just be referring to anything owned by Alphabet as owned by Google} and are cached at servers with numerous ISPs \cite{peering}, which means that your videos are very unlikely to be inaccessible by students. 
%https://peering.google.com/#/
YouTube's website
Analytic  (show image)


One benefit unique to hosting videos on YouTube is that will autogenerate subtitles in the detected spoken language \cite{liao2013large} and sync it with the video.  While the speech-to-text algorithm isn't perfect, it provides a huge reduction in effort in making videos more accessible, especially as YouTube provides an online tool to easily modify these subtitle files.


Watermark

Speed adjuster

\subsubsection{Negative Considerations and Mitigation}

So, why \textit{wouldn't} YouTube be the only choice in this section?
After all, those benefits  
YouTube doesn't go down very often.  
But it does sometimes \cite{outage2018} (June 2: 3:30 Outage) \cite{outage2019}, and when it does, student will probably complain to you 

YouTube comments are their own special hell.
Personally, we use comments as part of a ``bug bounty'' for our students.
Delay all comments for review or turn them off.
If we make some error during in our videos, students can leave a comment describing the error for a point of extra credit.

Reuploading

Additional steps need to be taken to allow students to obtain an offline copy of videos.

Legal ramifications of double dipping

related videos are not acutally 


Videos getting flagged and taken down.

Trolls.  God help you if you bring up feminism in video games or similar topics in an ethics course you decide to post on YouTube.  Expect harassment, up to and including death threats. %thing that ant specie con

\subsection{Vimeo}
Vimeo is a competitor of YouTube, located in New York City.
It provides many of the same features as YouTube: hosting for the video, a video player with a number functions, the ability to create embedded players on your own site.
However, Vimeo differs 


%While they provide the same basic service to the viewer as YouTube, Vimeo has a number of significant differences that are worth considering.
One of the most noticeable disadvantages of using YouTube is the featuring of ``related'' videos, which are often quite unrelated and, more importantly, immersion breaking and distracting to students.
This is because YouTube strives to keep its viewers within the YouTube ecosystem for as long as possible.



\subsubsection{Positive Considerations}
Professional looking
Customization embedding
No Ads
Can make videos easy to download

More link sharing options including a password 
Collaberation tools (with commenting)

\subsubsection{Negative Considerations}
The biggest difference between hosting videos on Vimeo and hosting videos on YouTube is that hosting on Vimeo requires money for the quantity of videos that you will most likely be uploading.

Vimeo plans cost \$240 annually for 1TB storage every year with a 20GB/week upload limit and \$400 for 5TB total storage and no upload limit.



\subsection{Drive or Static Webpage}
A much more informal way of hosting your recordings is to simply host your videos on a static webpage on your school's or department's server or use whatever cloud storage (such as Google Drive) that your institution is contracted with.
Students can either utilize their browsers built in player to watch the recordings or download the video and watch it locally.

A large benefit of this is that you and your school retain complete control over the recordings; you don't have to give any licenses or host it outside of your university's control.
In addition, there are no additional hosting costs involved.

However, it may be more difficult to embed these video onto a webpage or site such as canvas or blackboard, 
It may also make it trickier for students to watch the recordings on mobile devices compared to other options.


%\subsection{Fine, I'll do it myself}
%
%One final option some readers might consider is to host the recordings on their own video server.
%
%\subsubsection{Pros}
%The pros of this solution.
%\subsubsection{Cons}
%
%Really?  Where are you going to find the time?
%Home server? You probably won't have the bandwidth
%Server at school?  IT deptartment is going to LOVE you.


\section{My Implementation and Student Feedback}
\label{implementation}

Our implementation for recording is fairly straightforward.
We install OBS \cite{bailey2017open} on the instructor's laptop, which is equipped with an Intel i7-8550U processor, which is more than enough to handle the recording.

In a standard lecture, the instructor uses OBS to capture and record what is on the laptop's screen as they code and present content on the laptop.
One of the corners of the video output (the top-left or the top-right) is reserved for recording the webcam, as shown in IMAGE, so students can see the instructor as he speaks.
Occasionally, if content is put on the whiteboard, we use a user-defined hotkey command to switch between the screen-capture view and just capturing the webcam.
This allows us to control what the student sees, but it is important to remember to switch \textit{back to screen capture} once the instructor moves back to the laptop.  

For non-traditional lectures, such as a flipped classroom, we only capture the content that is information disseminated by the instructor, such as a quick review, instructions and further guidance for assignments, or answering student questions.
We don't record anything that is just students working on exercises or otherwise primarily student driven.

Once the recording is done, we upload the video onto YouTube and add it to the course's playlist (IMAGE??).
Our campus has a reliable high speed internet connection and thus requires less than 10 minutes to upload a 5 gigabyte recording.
From personal experience, we generally don't recommend uploading the recordings at home, as most residential internet plans have a much lower upload bandwidth than download bandwidth.
In other words, just because you can download things at home quickly won't necessarily mean you can upload quickly.

BUG BOUNTY


The overhead for adoption is about 1 hour to learn how to use OBS, then about 10 minutes to upload the recording.



%\section{Recap}



\section{A Note on Licenses and Legal Issues}
\label{legal}

Laws on privacy and recording vary from country to country and state to state.
In the United States, state laws regarding this can broadly be divided into two categories.
First, there are states that have one-party consent laws, where any person in the conversation (as opposed to outside it) can record the conversation.
The other states have two-party consent laws, where everyone privy to the conversation has to consent to recording.
However, we highly recommend telling students that you will be recording the lecture on the first day, as well as putting it in your syllabus to avoid any legal trouble and, more importantly, because its the decent thing to do.
Let your students know that you take their privacy seriously and the \textit{you} are the focus of the camera, not them (this is also a situation where the microphone having trouble picking up someone other than the lecturer is feature, as opposed to a bug).

This is especially important if your classroom is configured in a non-traditional setup, such as in a room designed for group-work with the lectern at the center, where the camera might catch students who are seated \textit{behind} the instructor.



FERPA and recording
don't record your roster or student's emails.

%Your school may not like discovering that the content of their classes is now available online.

%\section{A Note on Accessibility}
\bibliographystyle{acm}
\bibliography{recording}
\end{document}
