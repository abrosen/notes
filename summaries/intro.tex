\chapter{Introduction}
\label{chapter:intro}
% % % layout
% % % Distributed Computing Challenges
% % % Qualities of DHTs
% % % Hypothesis = These Problems + These qualiteies -> solution
% % % Framework of what these solutions are and what they can do

Distributed Computing is well understood to be the approach to take to solve large problems.  
Many problems can be broken up into multiple parts that can be solved simultaneously, yielding a much quicker result than a single working attacking the problem.
However, there are two broad obstacles in distributed computing.

%Is this 1 problem or two
The first is figuring out how the mechanics of efficiently distributing a problem to multiple workers and asynchronously coordinating their effort.  
The second is creating and maintaining the computation platform itself.


Some specific challenges are:
\begin{description}
	\item[Scalability] - Distributed computing platforms should not be completely static; if the platform would be improved by the addition of a new resource, it should be possible to add that resource.  
	The addition of new workers in a distributed computing framework should be a minimally disruptive process.
    This ties into the network's fault tolerance
	\item[Load-Balancing]  This is one of the most important issues to consider when creating a framework for distributed computing. 
    How do you split up a problem then distribute it so that no single worker is under or over-utilized?
    Failing that, how do you minimize the imbalance in work?
    Is there a way to do so at run time?
    \item[Fault Tolerance]  Even in a network that is expected to remain static for long periods of time, the platform still has to deal with failure.  
    In a centralized environment, hardware failures are common given enough machines.
    We want our platform to gracefully handle failures during runtime and be able to quickly reassign work to other workers.
    In addition, the network should be equally graceful in handling the introduction of new nodes during runtime.
\end{description}

Fortunately, these challenges are not unique to distributed computing but are also obstacles in distributed file storage.  
In particular, distributed file storage applications that utilize Distributed Hash Tables are designed to handle these particular challenges.
\section{Distributed Hash Tables}
%\section{DHTs are better for distributed computing under many circumstances}
Distributed Hash Tables (DHTs) are traditionally used as the backbone of structured Peer-to-Peer (P2P) file-sharing applications.
The largest such application by far is Bittorrent \cite{bittorrent}, which is built using Mainline DHT \cite{mainline},  a  derivative of Kademlia \cite{kademlia}.
The number of users on Bittorrent ranges from 15 million to 27 million users daily, with a turnover of 10 million users a day \cite{mainlineMeasure}.

Most research on DHTs assumes that they will be used in the context of a large P2P file-sharing application  (or at least, an application \textit{potentially} incorporating millions of nodes).
This lends the DHT to having particular qualities.
The network must be able to handle members joining and leaving arbitrarily.
The resulting application must be agnostic towards hardware.
The network must be decentralized and split whatever burden there is equally among its members.

In other words, distributed hash tables provide scalability, load-balancing, robustness, and heterogeneity to an application.
More recent applications have examined leveraging these qualities, since these qualities are desirable in many different frameworks.
For example, one paper \cite{Mateescu2011440} used a DHT as the name resolution layer of a large distributed database.
Research has also been done in using DHTs as an organizing mechanism in distributed machine learning \cite{liparameter}. 


I describe each of the aforementioned qualities and their ramifications below in sections \ref{subsec:ft}, \ref{subsec:lb}, \ref{subsec:scalability}, and \ref{subsec:hetero} .
While these properties are  be individually enumerated, they are greatly intertwined and the division between their impacts can be somewhat arbitrary.

\subsection{Robustness and Fault-Tolerance}
\label{subsec:ft}
One of the most important assumptions of DHTs is that they are deployed on a non-static network.
DHTs need to be built to account for a high level what is called \textit{churn}.  
Churn refers to the disruption of routing caused by the constant joining and leaving of nodes.
This is mitigated by a few factors.

First, the network is decentralized, with no single node acting as a single point of failure.
This is accomplished by each node in the routing table having a small portion of the both the routing table and the information stored on the DHT (see the Load Balancing property below).

Second is that each DHT has an inexpensive maintenance processes that mitigates the damage caused by churn.
DHTs often integrate a backup process into their protocols so that when a node goes down, one of the neighbors can immediately assume responsibility.
The join process also causes disruption to the network, as affected nodes have adjust their peerlists to accommodating the joiner. 

The last property is that the hash algorithm used to distribute content evenly across the network(again see load balancing) also distributes nodes evenly across the DHT.  
This means that nodes in the same geographic region occupy vastly different positions in the keyspace.  
If an entire geographic region is affected by a network outage, this damage is spread evenly across the DHT, which can be handled.

This property is the most important, as it deals with failure of entire sections of the network, rather than a single node.
Recent research in using DHTs for High End Computing \cite{li2013zht} shows what can happen if we remove this assumption by placing the network that is almost completely static.

The fault tolerance mechanisms in DHTs also provide near constant availability for P2P applications.
The node that is responsible for a particular key can always be found, even when numerous failures or joins occur \cite{chord}.

\subsection{Load Balancing}
\label{subsec:lb}
All Distributed Hash Tables use some kind of consistent hashing algorithm to associate nodes and file identifiers with keys.  
These keys are generated by passing the identifiers into a hash function, typically SHA-160.
The chosen hash function is typically large enough to avoid hash collisions and generates keys in a uniform manner. 
The result of this is that as more nodes join the network, the distribution of nodes in the keyspace becomes more uniform, as does the distribution of files.

However, because this is a random process, it is highly unlikely that each node will be spread evenly throughout the network.
This appears to be a weakness, but can be turned into an advantage in heterogeneous systems by using \textit{virtual nodes} \cite{godfrey2005heterogeneity} \cite{dynamo}.
When a node joins the network, it joins not at one position, but multiple virtual positions in the network \cite{dynamo}.
Using virtual nodes allows load-balance optimization in a heterogeneous network; more powerful machines can create more virtual nodes and handle more of the overall responsibility in the network.

DeCandia et al\. discussed various load balancing techniques that were tested on Dynamo \cite{dynamo}.  
Each node was assigned a certain number of tokens and the node would create a virtual node for each token.
The realization DeCandia et al\. had was that there was no reason to use the same scheme for data partitioning and data placement.
DeCandia et al\. introduced two new strategies which work off assigning nodes equally sized partitions.


% HELP
Under these schemes, each virtual node maps to an ID as before, but the patitions each node is responsible for are equally sized\footnote{Need help here}.

\subsubsection{Heterogeneity}
\label{subsec:hetero}
Heterogenity presents a challenge for load balancing DHTs due to conflicting assumptions and goals. 
DHTs assume that members are usually going to be varied in hardware, while the load-balancing process defined in DHTs treats each node equally.
It is much simpler to treat each node as equal unit.
In other words, DHTs support heterogeneity, but do not attempt to exploit it.

This doesn't mean that heterogeneity cannot be exploited
Nodes can be given addition responsibilities manually, by running multiple instances of the P2P application on the same machine or creating more virtual nodes.
However, this is not a feasible option for any kind of truly decentralized system and would need to be done automatically.
There is no well-known mechanism to that exists to automatically  allocate virtual nodes on the fly \footnote{citation needed, although this can be similar to IRM}. 
A few options present themselves.  % and are discuessed in \ref{}
%add the above if the below is moved

%This might be the wrong place for this
One is to use adapt a request tracking mechanism, such as what is used in IRM, except instead of tracking file requests, it tracks requests that are directed to a particular (real) node. 
If a particular (real) node receives an inordinate amount of requests, the node doing the detecting suggests that the node obtain another token/create another virtual node.
Another strategy is to use the preference lists/successor predecessor lists, and observe the distribution of the workload, adjusting the virtual nodes based on that. 

Dynamic load balancing may not be essential to P2P file-sharing applications, but is absolutely essential to any kind of P2P distributed computation.
In our ChordReduce experiments, we observed that just approximating dynamic load-balancing by simulating high levels of churn noticeably improved results\footnote{We found this by accident, just by testing the network's fault tolerance in regards to a high level of churn}.



%While most DHTs follow this scheme, there is some minor variation on how keys are generated.
%Some DHTs can or do use geographic information to generate keys.
%In VHash, keys are not static, but move according to a simplified spring model.

\subsection{Scalability}
\label{subsec:scalability}
In order to maintain scalability, a DHT has to ensure that as the network grows larger:

\begin{itemize}
    \item Churn does not have a disproportionate overhead.  
    For example, in a 1000 node network, a joining or leaving node will affect only an extremely small subset of these nodes.
    \footnote{We will see that this requirement can  be  relaxed in  very specific cases \cite{li2013zht}.}
    \item Lookup request speeds (usually measured in hops) grow by a much smaller amount, possibly not at all.
\end{itemize}

Using consistent hashing allows the network to scale up incrementally, adding one node at a time \cite{dynamo}.
In addition, each join operation has minimal impact on the network, since a node affects only its immediate neighbors on a join operation.
Similarly, the only nodes that need to react to a node leaving are its neighbors.
This is almost instantaneous if the network is using backups.
Other nodes can be notified of the missing node passively through maintenance or in response to a lookup.

There have been multiple proposed strategies for tackling scalability, and it is these strategies which play the greatest role in driving the variety of DHT architectures. 
Each DHT must strikes a balance between memory cost of the peerlist and lookup time. 
The vast majority of DHTs choose to use $\lg(N)$ sized routing tables and  $\lg(n)$ hops\footnote{log n or log N, matters}. 
Chapter \ref{chapter:background} discusses these tradeoffs in greater detail and how they affect the each DHT.





%\section{Different or subproblem: Certain DHTs are better at one application than another due to differences}
%\subsection{Design Differences Impacts}
%\subsection{Geometries}
%\subsection{Routing Table Construction}
%\subsection{Implementation Differences Impacts}
%\paragraph{Recursive or iterative seek}


\section{Hypothesis:  problems in distributed computing + solutions =  dissertation topic}
Distributed computing platforms need to be scalable, fault-tolerant, and load balancing.
In addition, the ability to incorporate heterogeneous hardware is a definite benefit.
Distributed Hash Tables can provide an application with all of these qualities.
P2P applications have been using DHTs for large-scale distributed file sharing applications for years now and are particularly effective.


I propose that DHTs can be used to create P2P distributed computing platforms that are completely decentralized.
Rather than keys being assigned to some data, we can assign keys to tasks and automatically distribute those tasks to the responsible nodes
There would be no need for some central coordinator or scheduler.

A successful DHT based computing platform would need to address the problem of dynamic load-balancing.
This is currently an unsolved problem and if an application can dynamically reassign work to nodes added at runtime, this opens up new options for resource management.
If a computation  is running too slow, new nodes can be added to the network  during runtime or idle nodes can boot up more virtual nodes (now that I think of it this is two different but highly related problems: internal and external).


The next chapter will delve into how DHTs work and examine specific DHTs.
The remainder of the paper will then discuss the work I plan on doing to demonstrate the viability of using DHTs for distributed computing.


%Add these bullets to the above paragraph
%\begin{itemize} 
%	\item DHTs can use consistent hashing supplemented by virtual nodes to efficiently load-balance.
%	The larger the network grows, the more evenly distributed the load becomes. 
%	\item DHTs are highly resilient to damage and can handle abnormally high rates of disruption.  
%	This is extremely desirable in any kind of distributed application %
%	\item Large-scale P2P file sharing applications have been using DHTs for a long time and
%    \item DHTs are extremely good if your problem is embarrassingly par
%    \item Heterogeneity
%\end{itemize}
