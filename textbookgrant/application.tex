\documentclass[]{article}

%opening
\title{Proposal for Adopting and Adapting an Open-Source Curriculum and Textbook for Python }
\author{Andrew Rosen\\ Assistant Professor of Instruction}

\begin{document}


%In narrative form, please describe your plan for moving toward a more affordable course material model by replacing commercial textbooks and other fee-based learning materials. 
%
%Proposals should display consideration for project planning, student accessibility and success, identifying resources, leveraging university, library, and technical resources, and copyright issues. Be sure to address any foreseeable challenges and be certain to describe your plan for evaluating and assessing student outcomes as well as your own experience. Specific questions you may consider addressing include:
%
%Provide your plan for seeking appropriate alternative course material. 
%
%How will you evaluate the usefulness of these course materials?
%
%How does the project plan leverage content through the library or the Special Collections Research Center?
%
%What problems do you anticipate encountering?
%
%How will you evaluate the implementation of your project plan, including student outcomes?
%
%Refer to sample project proposal narrative for an example proposal: http://bit.ly/2uveKN3 
%
%[500 word minimum]



\maketitle

\section{Premise and Motication}
CIS 1051 is one of two introductory programming courses offered by the Computer and Information Sciences Deptartment.  
Students must take this course or CIS 1057 (An introductory programming course in C) before moving onto CIS 1068 (a more advanced programming course in Java). CIS 1051 is geared towards students who have no background in computer science or programming, with only the most rudimentary math prerequisites.  This is the only programming course many students take.

I currently teach most of the sections for 1051, which amounts to about 150 students.  
The typical textbook costs between \$60 to \$150, depending on the source (the overall cost is left as an exercise to the reader).
This is typical for a computer science textbook from a major publisher.
This cost leads many students to just pirate the textbook (a behavior which many faculty turn a blind eye to), rely on the slides, or do without.



Previous semesters have utilized \textit{Think Python}, by Allen Downey, who is quite passionate about free books and has written a number of them for computer science and programming topics.
\textit{Think Python} is a well known free textbook, licenced under the CC BY-NC 3.0 creative commons license.  The hard copy MSRP is  \$45, but typically sells for \$30.
\textit{Think Python} is well written but the textbook is short, sparse on examples and exercises, and perhaps just a bit \textit{too} concise.
I have found our instructors typically just used as a reference rather than a companion for learning and diverge from the material at some point.

This most recent semester used \textit{Automate the Boring Stuff with Python}, by Al Sweigart, also under a creative commons license (CC BY-NC-SA 3.0).
\textit{Automate the Boring Stuff with Python} is an excellent book, with numerous examples, jokes, and perhaps a slightly unhealthy obsession with RoboCop.
It is intended for an audience of learners who do not intend to go into a programming related career, but might find their daily life imporived by 

missing turtles

\section{Plan}
open source and free software is highly valuable virtues in computing and should be exposed to students early.

integrate some chapters and examples from automate 
Utilize Runestone academy and python-snek

\section{Benefits}

\subsection{Progress Tracking}

\subsection{Parson Puzzles}



\section{Possible Negative Outcomes or Complications}

\subsection{Instructor Time Sink}
\subsubsection{Coordination With Canvas}
\subsection{Additional Login}


%\subsection{Accessibility Concerns}
%
%\subsubsection{Deaf and Hard of Hearing } 
%No complications are predicted.
%\subsubsection{Blind and Visually Impaired}
%Teaching a student who is blind or visually impaired always present a great challenge in programming, perhaps moreso in Python, where indentation is key factor in making sure programs built correctly.
%Such students would need to work closely with the instructor to address any a
%
%However, the adoption of the electronic textbooks.
%First, the textbook is in html, which 


\end{document}
