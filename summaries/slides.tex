\documentclass[11pt,aspectratio=169]{beamer}
\usetheme{Darmstadt}
%\usecolortheme{dolphin}


\usepackage[utf8]{inputenc}
\usepackage[english]{babel}
\usepackage{amsmath}
\usepackage{amsfonts}
\usepackage{amssymb}
\usepackage{graphicx}

\author{Andrew Rosen}
\title[DHT Distributed Computing]{Proposal Defense\\ Towards a Framework for DHT Distributed Computing}

\logo{\includegraphics[height=1cm]{figs/logo}}
\institute{Georgia State University}
\date{July 15th, 2015}
%\subject{}
%\setbeamercovered{transparent}
%\setbeamertemplate{navigation symbols}{}



\AtBeginSection[]
{
  \begin{frame}
    \frametitle{Table of Contents}
    \tableofcontents[currentsection]
  \end{frame}
}

\begin{document}
\maketitle


\section{Introduction}


\subsection{What I am doing}
\begin{frame}{Objective}
Our objective is to create a generalized framework for distributed computing using Distributed Hash Tables.

\end{frame}

\subsection{Distributed Computing and Challenges}

\begin{frame}{What is Distributed Computing}
	
\end{frame}


\begin{frame}{Challenges}
	\begin{itemize}
		\item<1-> Scalability
		\item<2-> Fault-Tolerance
		\item<3-> Load-Balancing
	\end{itemize}
	
\end{frame}


\subsection{What Are Distributed Hash Tables}

\begin{frame}{Distributed Key/Value Stores}
\end{frame}


\begin{frame}{Current Applications}
	Applications that use or incoperate DHTs:
	\begin{itemize}
		\item P2P File Sharing applications, such as Bittorrent \cite{bittorrent} \cite{mainline}.
		\item Distributed File Storage \cite{CFS}.
		\item Distributed Machine Learning \cite{liparameter}.
		\item Name resolution in a large  distributed database \cite{Mateescu2011440}.
	\end{itemize}
\end{frame}

\begin{frame}{Strengths of DHTs }
	
\end{frame}




\section{Background}

\subsection{The Components and Terminology}
\begin{frame}{Functions}
	\begin{description}
		\item[put($ key $, $ value $)] Stores $value$ at the node responsible for $key$, where $key =  hash(value)$.
		\item[get($ key $)] Returns the $ value $ associated with $key$.
		\item[lookup($ key $)] Finds the node responsible for a given key.
	\end{description}
\end{frame}


\subsection{Example DHTs}

\begin{frame}{Chord}
	content...
\end{frame}

\begin{frame}{Kademlia}
\end{frame}

\begin{frame}{VHash}
	content...
\end{frame}
\section{Previous Work}

\subsection{ChordReduce}

\begin{frame}{ChordReduce}
	content
\end{frame}

\subsection{VHash}


\begin{frame}{DGVH}
	content
\end{frame}


\subsection{Sybil}

\begin{frame}{Sybil Analysis}
	content
\end{frame}

\section{Proposed Work}



\subsection{UrDHT}
\begin{frame}{UrDHT}
	This kind of framework does not exist.
\end{frame}


\subsection{DHT Distributed Computing}
\begin{frame}{DHT Distributed Computing}
	content
\end{frame}



\subsection{Autonomous Load-Balancing}
\begin{frame}{Autonomous Load-Balancing}
	content
\end{frame}


\bibliography{notes,dht,mapreduce,voronoi,dns,botnets,mine}
\bibliographystyle{plain}
\end{document}