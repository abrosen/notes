\documentclass[sigconf]{acmart}
%\usepackage{cite}
%opening
\title{Recording Your Lecture Fun and (Your Student's) Profit}
\author{Ponder Stibbons}
\affiliation{\institution{Unseen University}}
\email{ps@uu.discworld.edu}


\begin{abstract}
	Recording your lectures may seem overwhelming, but in reality is actually quite easy to do.  
	This paper serves as a guide to do so.
	We explore the benefits to students and instructors for recording lectures, such as review material, mitigating illnesses and outbreaks, and future use for flipped classrooms. 
	We discuss the software options available to record lectures and weigh the benefits and consequences of various hosting options.
\end{abstract}
\begin{document}

\maketitle



\section{Introduction}


This paper advocates that every CS teacher record their lectures.
Recording your lectures has many benefits for the students, some benefits for you

This paper is intended for an audience of teachers, primarily people who teach using a laptop.




\begin{itemize}
	\item In Section \ref{why}, we discuss the motivations for this paper and impacts  on students for recording your lecture.
\end{itemize}
\section{Motivation}
\label{why}
Rec

Recording lectures has been

recording all of a professor's lecture in high fidelity is something that has become possible only in recent years.
Modern laptops come with at least 1080p screens and powerful processors able to handle both recording 

The emergence of codecs such as AV1 \cite{AV1comp} (not to be confused with .avi files) means that even as storage becomes more cheaper and plentiful, high quality files are getting smaller and smaller, with less restrictive licenses. 
Consumer cameras get better every year and can record in UHD (2160p resolutions or greater).



Larkin but they wontcome to class restiance to academic \
fear class may be optional
Students would still come to class and the recordings were used to aid in studying or to make up a lecture that was missed.

%The move to increasingly flexible platforms for student learning and experience
%through provision of online lecture recordings is often interpreted by educators as
%students viewing attendance at lectures as optional. The trend toward the use of this
%technology is often met with resistance from some academic staff who argue that
%student attendance will decline. This study aimed to explore students’ use of online
%lectures and to measure the impact of them on student attendance at lectures. A pre
%and post evaluation methodology was undertaken using a self administered
%questionnaire that gathered both quantitative and qualitative data. Overall attendance
%was recorded at each lecture throughout the semester. Results indicated that
%attendance remained high throughout the semester and while only a minority of
%students used the recordings, those who did found them to be helpful. Most students
%used them to either supplement their learning or to make up for a lecture that they
%had not been able to attend. This study provides evidence that contrary to popular
%belief, Generation Y students in general, do not aspire to replace lectures with
%downloadable, online versions. Many of the students in this study valued the
%opportunity for interactive learning provided by face to face teaching. Finally, a model
%that outlines the attributes that contribute to quality teaching is used to describe how
%this technology can contribute to positive student experiences and can enhance
%reflective teaching practice.


%Thus,competence or otherwise in relation to technology specifically, cannot necessarily be explained through generational attributes alone.


%Anecdotally, a common response by staff to demands or requests to incorporate these
%online approaches into their teaching and learning technologies, is a concern that if
%recorded lectures are made available, students will opt out of attending lectures. A
%large scale study by Gosper, Green, McNeill, Phillips, Preston and Woo (2008)
%confirmed this is a common concern amongst teaching staff in Australian universities.
%It could be argued that this concern reflects a ‘Level 1’ theory of teaching as described
%by Biggs and Tang (2007), that views teaching as purely the transmission of knowledge
%and a lecture as merely the vehicle for delivering information. This ‘sage on the stage’
%perspective supports a flawed belief that “the fundamental problems in the quality of
%university education can be solved by transferring knowledge more efficiently, using
%some form of information technology” (Ramsden, 2003, p. 108).
%As discussed by Ramsden (2003) and others, these views don’t recognise the
%interactive nature of face to face teaching and a need to focus on what the student does
%and needs to know that is consistent with contemporary views of teaching, outlined
%for example by Biggs and Tang (2007). As discussed by McGarr (2009), the often
%passive role ascribed to students in lectures is contrary to these contemporary views.
%While both critics and supporters of lectures continue to argue their relative
%advantages and disadvantages, the traditional lecture continues to be a dominant
%method for the delivery of teaching and learning in higher education (McGarr, 2009;
%Williams & Fardon, 2007). Several authors (McGarr, 2009; Taylor, 2009; Lazzari &
%Betella as cited in McGarr, 2009) argue the need to investigate the impact of lecture
%recordings on lecture attendance specifically, and teaching and learning in general. As
%described by McGarr (2009) “questions linger in relation to its true educational value,
%the ways it can best be utilised to support teaching and learning, and its affect on
%attendance and student engagement” (p. 312). This question is addressed in an
%Australian study by von Konsky, Ivins and Gribble (2009) who found that the act of
%making lectures available online did not have a significant impact on student
%attendance at lectures. This study recorded week by week attendance at lectures, the
%frequency of downloading of lectures and student perceptions of their use.



%A comparison of student perceptions and usage patterns
%Brian R. von Konsky, Jim Ivins and Susan J. Gribble
%Curtin University of Technology
%This paper investigates the impact of web based lecture recordings on learning and
%attendance at lectures. Student opinions regarding the perceived value of the
%recordings were evaluated in the context of usage patterns and final marks, and
%compared with attendance data and student perceptions regarding the usefulness of
%lectures. The availability of recordings was not seen to impact lecture attendance,
%although students showed some tendency to listen to the recording for a missed
%lecture. Students who achieved a high mark tended to supplement lecture attendance
%with recording usage more than students who achieved a low mark, but they did so
%with greater variation. If students perceived that a learning experience was of value to
%their learning, they were more likely to use it. Individual case studies describing
%perceptions, usage patterns, and attendance records of selected students highlight the
%fact that there is great variation in successful learning patterns, and suggest that
%engagement is an important factor impacting learning. Although the use of recordings
%to supplement lectures was seen to enhance the learning of some students, its uptake
%and effectiveness was not uniform across the cohort. This observation highlights the
%need for a range of learning modes in engineering education, appealing to a diverse
%set of individual learning styles. Future work is described in the context of these
%findings.
\subsection{Student Reference and study aid}


Ease of review
eases Note taking pressure


One study \cite{shimoff2001effects}  on an introductory Psychology course found that quiz grades increased by about a letter grade with the lectures recorded compared to the same course without recordings.

\subsection{Regular Illnesses and Outbreaks}
Our students are coming under more and more pressure to succeed.  
Enrollments in computer science are up across all universities, which means more students in a class. 
These increases in students and a drop in vaccination rates accross the US have lead to greater concern about outbreaks at numerous universities such as X at ABC, Menegitis , and a Mumps outbreak at Temple University \cite{emezienna2019resurgence}.

Sometimes, the only 

These outbreaks are in addition to regular infections, like influenza. 


Sick students have to make a choice
Students often come to class sick CITE, thus infecting other students.

%https://www.cdc.gov/meningococcal/outbreaks/index.html
%https://web.archive.org/web/20190604160258/https://www.cdc.gov/meningococcal/outbreaks/index.html
%https://www.tandfonline.com/doi/abs/10.1080/00039896.1962.10663216?journalCode=vzeh20
%https://academic.oup.com/jpubhealth/article/22/4/492/1527001
%https://jamanetwork.com/journals/jamapediatrics/article-abstract/380566
%https://www.ncbi.nlm.nih.gov/pmc/articles/PMC4577933/



\subsection{Use by instructor}
Illness of instructor, 
Experiment with flipped class room.


\subsection{Effects on Attendance}
One concern instructors may have about recording lectures is that it may create an incentive for students to skip class \cite{larkin2010but}.




Recording lectures has been found to have no negative effects on attendance, if not increased effects on attendance \cite{shimoff2001effects}

\subsubsection{Ensuring Attendance}
If, for some reason, students do stop attending or you want to ensure students will keep attending, there are a few suggested steps you can take \cite{larkin2010but}:
\begin{itemize}
	\item Create more active learning exercises.
\end{itemize}

Regardless, we suggest that for your first time recording, a simple approach that requires less work will yield best results.  
After that, make incremental changes throughout the semester.

\footnote{Also, ensuring that the course is not held at 8am or some other early hour will be more likely to ensure attendance.}

\subsection{Perceived Barriers to Adoption}
I'm scared of being on camera because I mess up.

This looks like too much work!

Accessibility concerns (see above about too much work, but just ask your drs.  These are supplemental resourses, they are not required.  Speak with your students).

Legal concerns (punt, ask your chair, or host on drive.)

\section{Enterprise Options}
I've listed this first because this is probably the most straightforward option.

Your university, college, or department might already have a license or contract with an enterprise solution.
Companies such has Panopto  offer a "one stop shop" hosting your data AND recording.


Editor on cloud means their computers do the rendering if you make edits

However, there are drawbacks.
First, files may be proprietary format, thus requiring an extra step to edit
Provided Editor may not be strong or easy to use or have the features you want
may not be able to control offline content precisely enough

\subsection{Cons}
Your students will need to access

\section{Open Source Recording}
The current best open source option for both recording and streaming (transmitting video live) is OBS Studio. (cite image)

OBS Studio is an extremely configurable, cross-platform recording program that is currently used by many streamers and has popular usage on Twitch, a video game streaming platform that brings in over one million viewers daily.  (image of hbomberguy or eu4 on twitch, example live stream )

It also has a number of filters built in, including a  ``Chroma Key'' filter, which can be used to remove a green screen if one is used (more on this later).



\section{Required Hardware and Equipment}
A modern laptop is all that is required.


%\subsubsection{Example laptops}
%\subsubsection{Specs}






\subsection{Additional Equipment}
Additional equipment costs up to around \$200, with tax, depending on the needs.

For example, I used have used a good quality microphone (\$100), a 1080p 60fps webcam (\$70), and a foldable greenscreen (\$40), to produce high qulality video recordings at home


\subsection{Concerns}
\begin{itemize}
	\item Your hardware must be good. Using an industrial strength IDE such as IntelliJ or Eclipse and any recording software can be taxing to older laptops.
	There are a few  ways to mitigate this  if getting a better laptop is problematic. 
	First is to have the IDE open before recording.
	Alternatively, migrate away from Intellij, Eclipse, and the like and use a more lightweight editor, such as Sublime Text, Atom, Visual Studio Code, Geany, or the old classics of Vim and Emacs.
	Another fix is to change the recording encoding setting to perform less encoding during recording; the result of this is that the file sizes will be higher, but it will be less taxing on the CPU.
	
	High end laptops (\$1000+ laptops with GTX 1050 graphics cards or better) can make use of hardware encoding, such as NVENC, which uses the graphics card to do the heavy lifting of the recording, which frees up a lot of processing power for the computer (INSERT PIC HERE).
	
	
	If you are just recording powerpoint slides or capturing the board with the webcamera, you should run into no difficulties.
	
	I also highly recommend having your laptop plugged in  so that you get maximum performance.
	\item Student questions can't be heard by your mic.  This is both a good thing, as it will alleviate privacy concerns of students, but students  reviewing the recording will be missing some information.  Thus, get into the habit of repeating and summarizing student queries when recording.  This will also help in large lectures, where the students present in the lecture won't necessarily hear the question either.
	
\end{itemize}

\section{Editing}

\section{Hosting}

In this section, we explore and look at the benefits of different solutions to host your videos.  
This sections assumes that you are not using an enterprise option, such as Panopto, as that typically provides a hosting solution with it.

\subsection{YouTube}
One of the most well known websites in the world, YouTube is nearly synonymous with video hosting and streaming.


\subsubsection{Positive Considerations}
Google is now in charge of making sure your stuff is accessible.
Videos are stored in Google datacenters\footnote{As we're not privy to the internal organization of datacenters within Alphabet, we'll just be referring to anything owned by Alphabet as owned by Google} and are cached at servers in numerous ISP, which means that your videos are very unlikely to be unaccessible by students. 
%https://peering.google.com/#/

In addition, easy to access from every platform.

Analytic  (show image)


One benefit unique to hosting videos on YouTube is that will autogenerate subtitles in the detected spoken language \cite{liao2013large} and sync it with the video.  While the speech-to-text algorithm isn't perfect, it provides a huge reduction in effort in making videos more accessible, especially as YouTube provides an online tool to easily modify these subtitle files.


Speed adjuster

\subsubsection{Negative Considerations and Mitigation}
YouTube doesn't go down very often.  
But it does sometimes \cite{outage2018} (June 2: 3:30 Outage) \cite{outage2019}, and when it does, student will probably complain to you 

YouTube comments are their own special hell.
Personally, we use comments as part of a ``bug bounty'' for our students.
If we make some error during in our videos, students can leave a comment describing the error for a point of extra credit.

Reuploading

Additional steps need to be taken to allow students to obtain an offline copy of videos.

Legal ramifications of double dipping

related videos are not acutally 


Videos getting flagged and taken down.

Trolls.  God help you if you bring up feminism in video games or similar topics in an ethics course you decide to post on YouTube.  Expect harassment, up to and including death threats. %thing that ant specie con

\subsection{Vimeo}
Vimeo is a competitor of YouTube, located in New York City.
It provides many of the same features as YouTube: hosting for the video, a video player with a number functions, the ability to create embedded players on your own site.
However, Vimeo differs 


%While they provide the same basic service to the viewer as YouTube, Vimeo has a number of significant differences that are worth considering.
One of the most noticeable disadvantages of using YouTube is the featuring of ``related'' videos, which are often quite unrelated and, more importantly, immersion breaking and distracting to students.
This is because YouTube strives to keep its viewers within the YouTube ecosystem for as long as possible.



\subsubsection{Positive Considerations}
Professional looking
Customization embedding
No Ads
Can make videos easy to download

More link sharing options including a password 
Collaberation tools (with commenting)

\subsubsection{Negative Considerations}
The biggest difference between hosting videos on Vimeo and hosting videos on YouTube is that hosting on Vimeo requires money for the quantity of videos that you will most likely be uploading.

Vimeo plans cost \$240 annually for 1TB storage every year with a 20GB/week upload limit and \$400 for 5TB total storage and no upload limit.



\subsection{Drive or Static Webpage}
A much more informal way of hosting your recordings is to simply host your videos on a static webpage on your school's or department's server or use whatever cloud storage (such as Google Drive) that your institution is contracted with.
Students can either utilize their browsers built in player to watch the recordings or download the video and watch it locally.

A large benefit of this is that you and your school retain complete control over the recordings; you don't have to give any licenses or host it outside of your university's control.
In addition, there are no additional hosting costs involved.

However, it may be more difficult to embed these video onto a webpage or site such as canvas or blackboard, 
It may also make it trickier for students to watch the recordings on mobile devices compared to other options.


%\subsection{Fine, I'll do it myself}
%
%One final option some readers might consider is to host the recordings on their own video server.
%
%\subsubsection{Pros}
%The pros of this solution.
%\subsubsection{Cons}
%
%Really?  Where are you going to find the time?
%Home server? You probably won't have the bandwidth
%Server at school?  IT deptartment is going to LOVE you.


\section{My Implementation and Student Feedback}

Our implementation for recording is fairly straightforward.
We install OBS \cite{bailey2017open} on the instructor's laptop, which is equipped with an Intel i7-8550U processor, which is more than enough to handle the recording.

In a standard lecture, the instructor uses OBS to capture and record what is on the laptop's screen as they code and present content on the laptop.
One of the corners of the video output (the top-left or the top-right) is reserved for recording the webcam, as shown in IMAGE, so students can see the instructor as he speaks.
Occasionally, if content is put on the whiteboard, we use a user-defined hotkey command to switch between the screen-capture view and just capturing the webcam.
This allows us to control what the student sees, but it is important to remember to switch \textit{back to screen capture} once the instructor moves back to the laptop.  

For non-traditional lectures, such as a flipped classroom, we only capture the content that is information disseminated by the instructor, such as a quick review, instructions and further guidance for assignments, or answering student questions.
We don't record anything that is just students working on exercises or otherwise primarily student driven.

Once the recording is done, we upload the video onto YouTube and add it to the course's playlist (IMAGE??).
Our campus has a reliable high speed internet connection and thus requires less than 10 minutes to upload a 5 gigabyte recording.
From personal experience, we generally don't recommend uploading the recordings at home, as most residential internet plans have a much lower upload bandwidth than download bandwidth.
In other words, just because you can download things at home quickly won't necessarily mean you can upload quickly.

The overhead for adoption is about 1 hour to learn how to use OBS, then about 10 minutes to upload the recording.



%\section{Recap}



\section{A Note on Licenses and Legal Issues}

Laws on privacy and recording vary from country to country and state to state.
In the United States, state laws regarding this can broadly be divided into two categories.
First, there are states that have one-party consent laws, where any person in the conversation (as opposed to outside it) can record the conversation.
The other states have two-party consent laws, where everyone privy to the conversation has to consent to recording.
However, we highly recommend telling students that you will be recording the lecture on the first day, as well as putting it in your syllabus to avoid any legal trouble and, more importantly, because its the decent thing to do.
Let your students know that you take their privacy seriously and the \textit{you} are the focus of the camera, not them (this is also a situation where the microphone having trouble picking up someone other than the lecturer is feature, as opposed to a bug).

This is especially important if your classroom is configured in a non-traditional setup, such as in a room designed for group-work with the lectern at the center, where the camera might catch students who are seated \textit{behind} the instructor.


%Your school may not like discovering that the content of their classes is now available online.

%\section{A Note on Accessibility}
\bibliographystyle{acm}
\bibliography{recording}
\end{document}
