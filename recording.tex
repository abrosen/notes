\documentclass[sigconf]{acmart}
%\usepackage{cite}
%opening
\title{Recording Your Lectures for Fun and (Possible) Profit}
\author{Andrew Rosen}
\affiliation{\institution{Temple University}}
\email{andrew.rosen@temple.edu}


\begin{abstract}
	Recording your lectures may seem overwhelming, but in reality is actually quite easy to do.
	
\end{abstract}
\begin{document}

\maketitle



\section{Introduction}


This paper advocates that every CS teacher record their lectures.
Recording your lectures has many benefits for the students, some benefits for you \cite{Nobody06}

This paper is intended for an audience of teachers, primarily people who teach using a laptop.
While this text is geared towards a tech-savvy audience of Computer Science educators, I believe 




\begin{itemize}
	\item In Section \ref{why}, we discuss the motivations for this paper and for recording your lecture
\end{itemize}
\section{Motivation}
\label{why}
Rec


The idea of recording all of a professor's lecture in high fidelity is something that has become possible only 
The emergence of codecs such as AV1 (not to be confused with .avi files) means that even as storage becomes more cheaper and plentiful, high quality files are getting smaller and smaller \cite{AV1comp}, with less restrictive licenses. 
Consumer cameras get better every year and can record in UHD.


\subsection{Student Reference and study aid}


Ease of review
eases Note taking pressure


One study \cite{shimoff2001effects}  on an introductary Psychology course found that quiz grades increased by about a letter grade with the lectures recorded compared to the same course without recordings.

\subsection{Regular Illnesses and Outbreaks}
Our students are coming under more and more pressure to succeed.  
Enrollments in computer science are up across all universities, which means more students in a class. 
These increases in students and a drop in vaccincation rates accross the US have lead to outbreaks at numerous universities such as X at ABC, and a Mumps outbreak at Temple University.

Sometimes, the only 

These outbreaks are in addition to regular infections, like influenza. 

Students often come to class sick CITE, thus infecting other students.

%https://www.cdc.gov/meningococcal/outbreaks/index.html
%https://web.archive.org/web/20190604160258/https://www.cdc.gov/meningococcal/outbreaks/index.html




\subsection{Use by instructor}
Illness of instructor, 
Experiment with flipped class room.


\subsection{Effects on Attendance}
One concern instructors may have about recording lectures is that it may create an incentive for students to skip class \cite{larkin2010but}.

In fact, N


Recording lectures has been found to have no negative effects on attendance (cite for this) \cite{shimoff2001effects}, if not increased effects on attendance \cite{shimoff2001effects}

\subsubsection{Ensuring Attendance}
If for some reason, students do stop attending or you want to ensure students will keep attending, there are a few suggested steps you can take \cite{larkin2010but}:
\begin{itemize}
	\item Create more active learning exercises.
\end{itemize}

Regardless, the author does suggest that for your first time recording, a simple approach that requires less work will yield best results.  After that, make incremental changes throughout the semester.

\footnote{Also, ensuring that the course is not held at 8am or some other early hour will be more likely to ensure attendance.}

\subsection{Perceived Barriers to Adoption}
I'm scared of being on camera because I mess up.

This looks like too much work!

Accessibility concerns (see above about too much work, but just ask your drs.  These are supplemental resourses, they are not required.  Speak with your students).

Legal concerns (punt, ask your chair, or host on drive.)

\section{Enterprise Options}
I've listed this first because this is probably the most straightforward option.
Your institution may be paying for this already.




options like panopto  offer a "one stop shop" hosting your data AND your


\subsection{Cons}
Your students will need to access

\section{Open Source Recording}
The current best open source option for both recording and streaming (transmitting video live) is OBS Studio.

OBS Studio is an extremely configurable, cross-platform recording program that is currently used by many streamers and has popular usage on Twitch, a video game streaming platform that brings in over one million viewers daily.

It also has a number of filters built in, including a  ``Chroma Key'' filter, which can be used to remove a green screen if one is used.



\section{Required Hardware and Equipment}
A modern laptop is all that is required.

\subsubsection{Example laptops}
\subsubsection{Specs}

Regardless of the hardware, using an industrial strength IDE such as IntelliJ or Eclipse and any recording software can be taxing to any laptop.
Two ways to mitigate this is to have the IDE open before recording.
Alternatively, migrate away from Intellij, Eclipse, and the like and use a more lightweight editor, such as Sublime Text, Atom, Visual Studio Code, Geany, or the old classics of Vim and Emacs.



If you are just recording powerpoint slides or capturing the board with the webcamera, you should run into no difficulties.

I also highly recommend having your laptop plugged in  so that you get maximum performance.

\subsection{Additional Equipment}



\subsection{Concerns}
\begin{itemize}
	\item Your hardware must be good
	\item Student questions can't be heard by your mic.
\end{itemize}

\section{Editing}

\section{Hosting}

\subsection{YouTube}


\subsubsection{Positive Considerations}
Google is now in charge of making sure your stuff is accessible.



Auto generated subtitle files

Youtube doesn't go down very often.  But it does sometimes \cite{outage2018} and (June 2: 3:30 Outage) \cite{outage2019}

\subsubsection{Negative Considerations}

It's Youtube

Youtube comments are their own special hell.


Reuploading

Additional steps need to be taken to allow students to obtain an offline copy of videos.

Legal ramifications of double dipping


\subsection{Vimeo}

\subsubsection{Positive Considerations}
Professional looking
Customization embedding

\subsubsection{Negative Considerations}


Pay for hosting

Also goes down occasionally.

\subsection{Drive}

Easy offline copies

No legal ramifications if using the school service

\subsubsection{Negative Considerations}
Not the prettiest

Harder to embed videos on canvas or the like.

You may have to pay for hosting




\subsection{Fine, I'll do it myself}

One final option some readers might consider is to host the recordings on their own video server.
\subsubsection{Pros}
TOTAL CONTROL
\subsubsection{Cons}

Really?  Where are you going to find the time?
Home server? You probably won't have the bandwidth
Server at school?  IT deptartment is going to LOVE you.


\section{My Implementation and Student Feedback}

\section{Recap}



\section{A Note on Licenses}



Your school may not like dicovering that the content of their classes is now available online.

\section{A Note on Accessibility}
\bibliographystyle{acm}
\bibliography{recording}
\end{document}
