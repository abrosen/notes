\documentclass[12pt,a4paper]{article}
\usepackage{mathptmx}
\usepackage[latin1]{inputenc}
\usepackage{amsmath}
\usepackage{amsfonts}
\usepackage{amssymb}
\usepackage{graphicx}
\usepackage[left=1.00in, right=1.00in, top=1.00in, bottom=1.00in]{geometry}
\author{Andrew Rosen}
\title{Grant Proposal}
\date{}
\begin{document}
\maketitle


%This summary should include a
%very brief summary description of the project including specific aims or goals, a
%statement of the significance or potential impact of the proposed project within the
%field of study and to the world in general, a statement or two about the general
%methods to be employed, and a statement about the expected outcome(s) of the project
%(research question(s) to be answered, new information or data to be uncovered, artistic
%product to be completed, etc.)
\begin{abstract}
	
\end{abstract}

\newpage



%Project Description (no more than three pages, single-spaced): This section should
%outline briefly the past work in the field (i.e. practical, theoretical and empirical work) 
%as a framework for discussing why the work outlined in the proposal is important to
%the field of study and to the world in general. The project description should also
%include the specific aims or goals for the proposed project and provide a clear, detailed
%account of the methodology to be employed and how those methods will lead to the
%expected outcomes. A reference list should be included (if appropriate), but references
%are not subject to the page limitations



\section{What are DHTs}
My dissertation focusing on creating a generalized framework for distributed computing based on Distributed Hash Tables (DHTs).
Distributed Hash Tables are a tool which allows a computers (nodes) to self-organize into a decentralized network for storing and retrieving data.

There are many different types of DHTs, but all DHTs share very important qualities.
They are \textit{scalable}, which means that each additional node in the network minimally impacts the cost of keeping the network organized.
DHTs are also highly \textit{fault-tolerant}.
Unlike many other systems, DHTs assume that nodes will be continuously entering and leaving the network.
Because of the way DHTs are organized, DHTs can handle large scale failures, such as power outage affecting an entire city.
The last quality of DHTs is that they are  \textit{load-balancing}. 
This mean the data the is stored in a DHTs is evenly distributed among nodes in the network.


\section{What Have We done so far and why should the reader care}

All of these qualities are highly desirable in a distributed computing, so it was a natural step for us to build distributed computing framework based on DHTs.
We have shown in previous work that DHTs can be used to distribute work among nodes the same way data is distributed \cite{chordreduce}.


\section{What do we plan on doing}


\subsection{Why is this exciting }
Some example problems are Monte-Carlo Methods (of interest to mathematicians and economists), 
Our framework would not only be of interest to computer scientists.
The ``plug-and-play'' nature we envision would make it easy for experts in other fields to use our framework to solve computationally intensive problems.


\section{What do I need funding for?}

Heterogeneous networks

We have successfully developed our software to achieve this, but need funding to do real world network tests. 
These tests would allow us to experimentally show that our software conclusively works on a large scale and on heterogeneous networks.



I need hardware of different types and locations.



This project would help by enabling allowing  both organizations with large amounts of computing power and average developers with a fewer resources spend less time setting up and configuring their hardware to work together;  the goal is for our software to make distributed computing more of matter of ``plug-and-play.''


\newpage


\bibliography{mine,dht}
\bibliographystyle{plain}
\end{document}