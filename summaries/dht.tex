\documentclass[10pt,letterpaper]{report}
\usepackage[utf8]{inputenc}
\usepackage{amsmath}
\usepackage{amsfonts}
\usepackage{amssymb}
\usepackage{graphicx}
\author{Andrew Rosen}
\title{Attributes of Distributed Hash Tables and Their Ramifications}
\begin{document}
\maketitle


\chapter{Introduction}


\chapter{So What Are Those Attributes Anyway?}


\section{Routing}
\section{Churn and Fault Tolerance}  % not nessuicarily the same, ie what happens when we route wrong
\section{Security}

\subsection{Sybil Attacks}

\subsection{Eclipse Attacks}

\chapter{The Four Kings}% Structured Overlays for Distributed Hash Tables

\section{Shared Attributes}
% log routing log table

\section{Chord}

\section{Pastry}

\section{Tapestry}

\section{CAN}




\chapter{The Challengers}

\section{Kademlia}


\chapter{The Small New World}

\section{The Small World}

\paragraph{Kevin Bacon}
\paragraph{The experiment}
\paragraph{Kleinberg}


\section{Voronoi Based Schemes}

\subsection{RayNet}

Beaumont et al argues that a loose structure enough for searching.  Assume a $d$-dimension space, each dimension tied to some attribute of an object and each object identified by a unique set of values.  Objects should be linked to other objects that are close in the space.

\chapter{Momentum}
Or who's idea was it that the darn things don't actually move!


\end{document}