\documentclass{res}
%\usepackage{helvetica} % uses helvetica postscript font (download helvetica.sty)
\pagestyle{headings}
\setlength{\topmargin}{-0.6in}  % Start text higher on the page 
\setlength{\textheight}{9.8in}  % increase textheight to fit more on a page
\setlength{\headsep}{0.2in}     % space between header and text
\setlength{\headheight}{12pt}   % make room for header
\usepackage{fancyhdr}  % use fancyhdr package to get 2-line header
\usepackage{hyperref}
\renewcommand{\headrulewidth}{0pt} % suppress line drawn by default by fancyhdr
\lhead{\hspace*{-\sectionwidth}Andrew Rosen} % force lhead all the way left
\rhead{Page \thepage}  % put page number at right
\cfoot{}  % the footer is empty
\pagestyle{fancy} % set pagestyle for the document

\begin{document}
\thispagestyle{empty} % this page does not have a header

% Center the name over the entire width of resume:
 \moveleft.5\hoffset \centerline{\large\bf Andrew Benjamin Rosen}
% Draw a horizontal line the whole width of resume:
 \moveleft\hoffset\vbox{\hrule width\resumewidth height 2.5pt}\smallskip
  \moveleft.5\hoffset\centerline{Curriculum Vitae}

\moveleft\hoffset\vbox{
\begin{minipage}{0.38\textwidth}
\begin{flushleft}
{\bf Postal Address} \\ 
Andrew B. Rosen\\ 
Temple University \\ 
Department  of Computer \& Information Science \\ 
1925 N. 12th St, Rm. 349\\ 
Philadelphia, PA 19122
\end{flushleft}
\end{minipage}
\begin{minipage}{0.7\textwidth}
\begin{flushright}
\hfill {\bf Other Contact Information} \\ 
\hfill E-mail: \href{mailto:andrew.rosen@temple.edu}{andrew.rosen@temple.edu} \\ 
\end{flushright} 
\end{minipage}
}
 
\begin{resume} 

\section{Research Interests}
	\begin{itemize}
    \item Delay and Fault Tolerant Networks
    \item Peer-to-Peer Networks
	\item Distributed Hash Tables
	\item Interplanetary Internet 
	\item Astroinformatics
	
	\end{itemize}
\section{Education}          
    \begin{itemize}    
    \item {\bf Ph.D.} in Computer Science, Georgia State University. May 2016
    \begin{itemize}
    	\item Dissertation: Towards a Framework for DHT Distributed Computing
    \end{itemize}
    \item {\bf M.S.} in Computer Science, Georgia State University. May 2014 
    \item {\bf B.S.} in Computer Science, Georgia Institute of Technology. May 2010
	\item {\bf Minor} in Music, Georgia Institute of Technology. May 2010
    
    \end{itemize}   


\section{Academic Experience}
	\begin{itemize}
		\item  Assistant Professor of Instruction in Computer and Information Sciences, Temple University, Fall 2016-Present
		\item  Visiting Assistant Professor of Instruction in Computer and Information Sciences, Temple University - Japan Campus, Summer 2018
		
	\end{itemize}



\section{Awards}
\begin{itemize}    
	\item Dean's Distinguished Teaching Award, 2018
	\item Outstanding Graduate Teaching Award, 2015
\end{itemize}   


\section{Publications}
	\begin{enumerate}
	\item Andrew Rosen, Brendan Benshoof, Robert W. Harrison, Anu G. Bourgeois
			``MapReduce on a Chord Distributed Hash Table''
			Presentation ICA CON 2014, Poster at IPDPS 2014 PhD Forum
	\item Brendan Benshoof, Andrew Rosen, Anu G. Bourgeois, Robert W. Harrison
			``VHASH: Spatial DHT based on Voronoi Tessellation''
			ICA CON 2014
	\item  Erin-Elizabeth A. Durham, Andrew Rosen, Robert W. Harrison
    ``A Model Architecture for Big Data applications using Relational Databases''
    2014 IEEE BigData - C4BD2014 - Workshop on Complexity for Big Data
    \item Chinua Umoja, J.T. Torrance, Erin-Elizabeth A. Durham, Andrew Rosen, Dr. Robert Harrison
    	``A Novel Approach to Determine Docking Locations Using Fuzzy Logic and Shape Determination''
    	2014 IEEE BigData - Poster and Short Paper
    \item  Erin-Elizabeth A. Durham, Andrew Rosen, Robert W. Harrison
    ``Optimization of Relational Database Usage Involving Big Data''
     IEEE SSCI 2014 - CIDM 2014 - The IEEE Symposium Series on Computational Intelligence and Data Mining
	\item Brendan Benshoof, Andrew Rosen, Anu G. Bourgeois, Robert W. Harrison
	 ``A Distributed Greedy Heuristic for Computing Voronoi Tessellations With Applications Towards Peer-to-Peer Network''
	IEEE IPDPS 2015 - Workshop on Dependable Parallel, Distributed and Network-Centric Systems
	\item Brendan Benshoof, Andrew Rosen, Anu G. Bourgeois, Robert W. Harrison
	``Distributed Decentralized Domain Name Service''
	IEEE IPDPS 2016 - Workshop on Dependable Parallel, Distributed and Network-Centric Systems
	\item Chaoyang Li, Andrew Rosen, Caroline Johnson, Anu G. Bourgeois 
	``Full-view coverage holes detection and healing solutions''
	IEEE ICCC 2016
	\item Chaoyang Li, Andrew Rosen, Anu G. Bourgeois 
	``On k-full-view-coverage-algorithms in camera sensor networks''
	IEEE ICCC 2016
	\item Chaoyang Li, Andrew Rosen, Anu G. Bourgeois
	``On camera sensor density minimization problem for triangular lattice-based deployment in full-view covered camera sensor networks'' 
	IEEE ICCC 2016
	%\item Andrew Rosen, Brendan Benshoof, Robert W. Harrison, Anu G. Bourgeois
	%``UrDHT: A Unified Model for Distributed Hash Tables''
	%In preparation
%	\item Andrew Rosen, Brendan Benshoof, Robert W. Harrison, Anu G. Bourgeois
%    ``The Sybil Attack on Peer-to-Peer Networks From the Attacker’s Perspective''
%    In preparation
%%	\item Chaoyang Li, Andrew Rosen,  and Anu G. Bourgeois
%%        ``A Novel Approach to Efficiently Detect 3D Full-View Coverage for Camera Sensor Networks''
%        In Preparation
        
    \end{enumerate}
    
    
  
\section{Research and Projects}

{\bf UrDHT, 2015 - Present }
\begin{itemize}
	\item We designed and built a framework which maps distributed hash tables to the primitives of Voronoi Tesselation and Delaunay Triangulation.
	\item UrDHT allows developers to quickly create new DHT topologies by completing a few simple functions. 
	\item Prototype implementation in Python.
	\item Project repo here: \url{https://github.com/UrDHT}
\end{itemize}

{\bf Sybil Attack Cost Analysis, 2015}
\begin{itemize}
	\item Analyzed the computational and monetary cost of performing a large scale Sybil attack.
	\item Code and Paper here: \url{https://github.com/abrosen/datasec/tree/master/project}
\end{itemize}



{\bf Performing MapReduce on a Chord Distributed Hash Table, 2013 - 2014}
    \begin{itemize}
    \item We examined using the self-organizing features of a DHT for distributed computing.
    \item We tested the system by deploying it on Amazon EC2 and computing Monte-Carlo methods and word frequency counts. 
    \item Code and paper can be found here \url{https://github.com/BrendanBenshoof/Chronus}
    \end{itemize}
    
{\bf VHash, 2014}
    \begin{itemize}
    \item We designed a new DHT that uses Voronoi regions to determine responsibility for resources.
    \item We detail algorithms that extend into an arbitrary number of dimensions, a feature lacking in similar works.
    \item Code and paper can be found here \url{https://github.com/BrendanBenshoof/pyVHash}
    \end{itemize}   


{\bf D$^3$DNS, 2013 }
    \begin{itemize}
    \item We created a secure and fault-tolerant prototype replacement for DNS.
    \item Our solution is reverse compatible with the current system.
    \item Code: \url{https://github.com/BrendanBenshoof/P2PDNS}
    \item Paper: \url{https://github.com/BrendanBenshoof/P2PDNS/blob/master/P3DNS.pdf}
    \end{itemize}

 
%    
%{\bf Analysis of Unevenly Sampled Periodic Data, 2013 - Present}
%    \begin{itemize}
%    \item Techniques for analysis are relatively unknown outside of Astronomy, where it has been rigorously covered.
%    \item Looking for new areas to apply analysis, such as estimation of traffic time.
%    \end{itemize}


%{\bf A Survey of Routing Protocols for Vehicular Ad-Hoc Networks, 2012}
%    \begin{itemize}
%    \item Explores common obstacles experienced in challenged and delay-tolerant networks.
%    \item Examines in-depth various routing protocols for VANETs.
%    \item Current work involves covering protocols for other challenged networks.
%    \item \url{http://www.cs.gsu.edu/~arosen6/survey/survey.pdf}
%    \end{itemize}

%{\bf Reducing Traffic and Delays in P2P Systems with Replicated Mutable Files,  2011 }
%   \begin{itemize}
%    \item Reduced overhead of maintenance of mutable files while diverting traffic away from file sources.
%    \item Strategies can be implemented on other DHT based P2P systems.
%    \item \url{http://www.cs.gsu.edu/~arosen6/papers/IRMLP.pdf}
%    \end{itemize}

%{\bf First Year Retention Rate Analysis, 2011}
%    \begin{itemize}
%    \item Data Mining term project.
%    \item Examined the performance of first-year computer science attending Georgia State University to discover trends in year one retention.
%    \item \url{http://www.cs.gsu.edu/~arosen6/papers/dm.pdf}
%    \end{itemize}

%{\bf Asthma Educational Game, 2009}
%    \begin{itemize}
%    \item Flash based educational game developed with two other students to teach about
%asthma developed for senior project in Computer Science.
%   \end{itemize}
\newpage
\section{Teaching}


\textbf{CIS 2168 Data Structures (Fall 2016 -  Present)}
\begin{itemize}
	\item Flipped Class since Fall 2018
	\item Class size of about 40 
	\item Covered Data Structures and intermediate Java.
	
\end{itemize}

\textbf{CIS 1068 Program Design \& Abstraction (Fall 2016 -  Summer 2018)}
\begin{itemize}
	\item Class size of about 40 to 60
	\item Covered introductory Java and object-oriented programming.
\end{itemize}


\textbf{CIS 1068 Program Design \& Abstraction (Fall 2016 -  Summer 2018)}
\begin{itemize}
	\item Class size of about 40 to 60
	\item Covered introductory Java and object-oriented programming.
\end{itemize}




{\bf CSc 3320  System Level Programming (Spring 2015) }
\begin{itemize}
	\item Instructor - class size of 51
	\item Covered programming in and writing scripts for the Unix operating system.
	\item Introduced Python as a scripting language to the students.
	\item Taught more advanced topics in C: pointers and pointer arithmetic, memory management, segmentation faults, and buffer overflows.
\end{itemize}


{\bf CSc 2010 Principles of Computer Science (Spring 2014 and Spring 2016) }
\begin{itemize}
	\item Instructor - class size of ~100
	\item Covered introductory Java topics including syntax, methods, and objects.
	\item Introduced foundational topics in Computer Science, such as the design an analysis of algorithms, binary, circuits, and architecture.
\end{itemize}


{\bf CSc 3320  System Level Programming (Fall 2013)}
    \begin{itemize}
    \item Teaching Assistant - class size of 50
      \item Helped answer during office hours and during class.  Graded homework and exams.  
    \end{itemize}


{\bf CSc 3210 Computer Organization and Programming (Summer 2013)}
    \begin{itemize}
    \item Teaching Assistant - class size of ~30
      \item Helped answer during office hours.  Graded homework.  Helped maintain course server.
    \end{itemize}



{\bf CSc 2010 Intro to Computer Programming - Robots Section (Spring 2011 and Spring 2013) }
    \begin{itemize}
      \item Teaching Assistant - class size of 25
      \item Helped maintain robots.  Helped developed critical thinking skills.  Created tests and quizzes.
    \end{itemize}


{\bf CSc 3410 Data Structures  - CTW (Fall 2011 and  Fall 2012) }
    \begin{itemize}
    \item Instructor - class size of 25
      \item Covered advanced topics in Java.  Covered various data structures such as linked lists, queues, stacks, trees, and graphs.  Emphasized critical thinking skills and object-oriented design.
    \end{itemize}

\newpage
\section{Service}


\textbf{CIS Merit Committee,} Temple University, Fall 2018

\textbf{CIS Graduate Committee,} Temple University, 2016 - Spring 2018

{\bf Vice Chair,} Georgia State University Chapter of the Association for Computing Machinery (ACM),  May 2014 - 2016
%     \begin{itemize}
%     \item Coordinate meetings and event preparations.
%     \item Assume responsibility when the Chair is unavailable.
%    \end{itemize}   

{\bf Treasurer,} Georgia State University Chapter of the Association for Computing Machinery (ACM), May 2012 - May 2014
%     \begin{itemize}
%     \item Plan and maintain budget.
%     \item Collect fees.
%    \end{itemize}   

{\bf New Graduate Student Orientation Panelist}, Georgia State University, 2014-2015


{\bf Department Representative,} Georgia State University Arts and Sciences Technology Fee Committee, 2013 - 2015
\begin{itemize}
	\item Voted on submitted proposals to allocate tech fee funds each year.
	\item Allocated $\approx \$1,000,000$ each year.
\end{itemize}

{\bf Subreviewer,} ISBRA 2015

{\bf Volunteer and Judge,} HackGSU, Spring 2016

{\bf GSU Reddit AMA,} 2016



{\bf New Graduate Student Orientation Panelist}, Georgia State University, 2014-2015



\section{Appointments}

{\bf 2CI Astroinformatics Fellow,} Georgia State University, Aug 2012 - Jun 2016
     \begin{itemize}
     \item Refactored database for near-earth stellar objects and developed an automated tool to load data into the database.
     \item Currently working with Astronomy Department on developing tools for analysis of suspected periodic signals.
     \item Examining using techniques for analyzing unevenly sampled periodic data in network traffic analysis.
     \end{itemize}   


{\bf Graduate Research Assistant,} Georgia State University, Aug 2011 - Jun 2016
     \begin{itemize}
     \item Researched the use of self-organizing features of DHTs to in performing distributed computations.
     \item Researched various protocols used for delay tolerant networking and interplanetary networking.
     \item Examined application of algorithms from one body of challenged networks to another.
  
     \end{itemize}   
     
{\bf Graduate Lab Assistant,} Georgia State University, May 2011 - 2013
     \begin{itemize}
     \item Deployed and maintained computer labs, faculty, and graduate student machines.
     \item Constructed and deployed new computers for faculty and graduate students.
     \item Migrated e-mail server.
     \end{itemize}
     

\section{External Funding}
\textbf{TCPP Travel Grant} for 28th IEEE International Parallel \& Distributed Processing Symposium




\section{Employment}
%{\bf Software Engineer,} OrcaTec, Atlanta, GA Summer 2011
%	\begin{itemize}
%		\item Worked on automating procedure for data-mining system.
%		\item Documented the various components of the system.
%    	\end{itemize}      
% 
{\bf Developer} Georgia Tech Sonification Lab, Atlanta, GA May-Dec 2010
	\begin{itemize}
		\item Set up and maintained new lab server.
		\item Extended the NASA Math Description Engine to incorporate the results of research on developing graph descriptions for the visual impaired.
		\item Developed a parser for formulas to interface with different software libraries.
		\item Software was presented to Washington lawmakers.
		\item \url{http://sonify.psych.gatech.edu/research/sonification_sandbox/index.html}.
    \end{itemize}      
    
{\bf Undergraduate Researcher} Georgia Tech Sonification Lab, Atlanta, GA Fall 2007 -Dec 2009
	\begin{itemize}
		\item Helped develop the Sonification Sandbox, a cross-platform tool which creates auditory graphs, by finding and extending libraries to add Excel-like operations, graph formulas, and generate and play midi information.
		\item Developed a tool to generate Spearcons - auditory icons developed by the Sonification Lab.
		\item Helped develop a web based tool to measure the use of sound in software by analyzing the source code of program.
    \end{itemize}      
    

%\section{Programming Skills}
%{\bf Languages:} Python, C, Java.

%{\bf Fields:} Distributed Hash Tables, Fault Tolerant Networks, Compilers, Simulations, Networking, TCP/IP, Data Mining,  HCI

\section{Other}
I compose and remix music in my spare time. \\
My Erd\"{o}s number is 5.\\
I have built multiple computers from parts for both personal and professional purposes.



\end{resume} 
\section{References available upon request.}
\end{document} 
