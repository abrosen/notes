\documentclass[11pt,aspectratio=169]{beamer}
\usetheme{Darmstadt}
%\usecolortheme{dolphin}

%\setbeameroption{show notes}
%\setbeameroption{show only notes}


\usepackage[utf8]{inputenc}
\usepackage[english]{babel}
\usepackage{amsmath}
\usepackage{amsfonts}
\usepackage{amssymb}
\usepackage{graphicx}

\author{Andrew Rosen}
\title[DHT Distributed Computing]{Proposal Defense\\ Towards a Framework for DHT Distributed Computing}

\logo{\includegraphics[height=1cm]{figs/logo}}
\institute{Georgia State University}
\date{July 15th, 2015}
%\subject{}
%\setbeamercovered{transparent}
%\setbeamertemplate{navigation symbols}{}



\AtBeginSection[]
{
  \begin{frame}
    \frametitle{Table of Contents}
    \tableofcontents[currentsection]
  \end{frame}
}

\begin{document}
	
	
	
\maketitle


\section{Introduction}


\subsection{What I am doing}
\begin{frame}{Objective}
Our objective is to create a generalized framework for distributed computing using Distributed Hash Tables.

\pause
\begin{center}
	Or
\end{center}

\pause
We want to build a completely decentralized distributed computing system.
\end{frame}

\subsection{Distributed Computing and Challenges}

\begin{frame}{What do I Mean by Distributed Computing?}
	A system where we can take a task and break it down into multiple parts, where each part is worked upon individually.
\end{frame}

\begin{frame}{Challenges of Distributed Computing}
	Distributed Computing platforms should be:
	\begin{description}
		\item<1->[Scalable] The larger the network, the more resources need to be spent on maintaining and organizing the network. \note<1>{Remember, computers aren't telepathic. There's always an overhead cost.  It will grow.   The challenge of scalability is designing a protocol that grows this organizational cost at an extremely slow rate. 
			For example, a single node keeping track of all members of the system might be a tenable situation up to a certain point, but eventually, the cost becomes too high for a single node.}
		\item<2->[Fault-Tolerant] As we add more machines, we need to be able to handle the increased risk of hardware failure. \note<2>{Hardware failure is a thing that can happen. Individually the chances are low, but this becomes high when we're talking about millions of machines.  Also, what happens in a P2P environment.}  
		\item<3->[Load-Balancing] Tasks need to be evenly distributed among all the workers. \note<3>{If we are splitting the task into multiple parts, we need some mechanism to ensure that each worker gets an even (or close enough) amount of work.}
	\end{description}
	
\end{frame}


\subsection{What Are Distributed Hash Tables?}

\begin{frame}{Distributed Key/Value Stores}
	Distributed Hash Tables are mechanisms for storing values associated with certain keys.
	\begin{itemize}
		\item Values, such as filenames, data, or IP/port combinations are associated with keys.
		\item These keys are generated by taking the hash of the value.
		\item We can get the value for a certain key by asking any node in the network.
	\end{itemize}
\end{frame}


\begin{frame}{Current Applications}
	Applications that use or incorporate DHTs:
	\begin{itemize}
		\item P2P File Sharing applications, such as Bittorrent \cite{bittorrent} \cite{mainline}.
		\item Distributed File Storage \cite{CFS}.
		\item Distributed Machine Learning \cite{liparameter}.
		\item Name resolution in a large distributed database \cite{Mateescu2011440}.
	\end{itemize}
\end{frame}


\begin{frame}{How Does It Work?}
	We'll explain in greater detail later, but briefly:
	\begin{itemize}
		\item DHTs organize a set of nodes, each identified by an \alert{ID} (their key). \note{We use ID for nodes and keys for data so we always know our context.}
		\item Nodes are responsible for the keys that are closest it their IDs.
		\item Nodes maintain a list of other peers in the network.
		\begin{itemize}
			\item Typically a size $ \log(n)$ subset of all nodes in the network.
		\end{itemize}
		\item Each node uses a very simple routing algorithm to find a node responsible for any given key.  %Explain what that is
	\end{itemize}
\end{frame}



\begin{frame}{Strengths of DHTs }
	DHTs are designed for large P2P applications, which means they need to be (and are):
	\begin{description}
		\item[Scalable]
		\item[Fault-Tolerant]
		\item[Load-Balancing]
	\end{description}
	
\end{frame}

\note[itemize]{
	\item Remember to mention Napster. 
	\item Distributed Hash Tables were designed to be used for completely decentralized P2P applications involving millions of nodes. 
	\item As a result of the P2P focus, DHTs have the following qualities
	}

\subsection{Why DHTs and Distributed Computing}

\begin{frame}{DHTs Address the Specified Challenges}
\end{frame}

\begin{frame}{Uses For DHT Distributed Computing}
	\begin{itemize}
		\item Can be used in either a P2P context or a more traditional deployment.
	\end{itemize}
\end{frame}


\section{Background}

\subsection{The Components and Terminology}
\begin{frame}{Functions}
	\begin{description}
		\item[put($ key $, $ value $)] Stores $value$ at the node responsible for $key$, where $key =  hash(value)$.
		\item[get($ key $)] Returns the $ value $ associated with $key$.
		\item[lookup($ key $)] Finds the node responsible for a given key.
	\end{description}
\end{frame}




\subsection{Example DHTs}

\begin{frame}{Chord}
	content...
\end{frame}

\begin{frame}{Kademlia}
\end{frame}

\begin{frame}{VHash}
	Maybe, if room
\end{frame}
\section{Previous Work}

\subsection{ChordReduce}

\begin{frame}{ChordReduce}
	content
\end{frame}

\subsection{VHash}


\begin{frame}{DGVH}
	content
\end{frame}


\subsection{Sybil}

\begin{frame}{Sybil Analysis}
	content
\end{frame}

\section{Proposed Work}



\subsection{UrDHT}
\begin{frame}{UrDHT}
	This kind of framework does not exist.
\end{frame}


\subsection{DHT Distributed Computing}
\begin{frame}{DHT Distributed Computing}
	content
\end{frame}



\subsection{Autonomous Load-Balancing}
\begin{frame}{Autonomous Load-Balancing}
	content
\end{frame}


\bibliography{notes,dht,mapreduce,voronoi,dns,botnets,mine}
\bibliographystyle{plain}
\end{document}