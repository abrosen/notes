\documentclass[10pt,letterpaper]{report}

\usepackage{amsmath}
\usepackage{amsfonts}
\usepackage{amssymb}
\usepackage{graphicx}
\usepackage{tabularx}
\usepackage{algorithm}
\usepackage{algpseudocode} 
\usepackage{subcaption}
\usepackage[font=small]{caption}
\usepackage{url}
\usepackage[english]{babel}

%\usepackage{times}
%\usepackage{venturis2}

\linespread{1.0}
\title{Proposal \\ Towards a Framework for DHT Distributed Computing}
\author{Andrew Rosen}
\begin{document}



\maketitle




\setcounter{tocdepth}{4}
\tableofcontents
\newpage

\begin{abstract}
Distributed Hash Tables (DHTs) are protocols and frameworks used by peer-to-peer (P2P) systems.
They are used as the organizational backbone for many P2P file-sharing systems due to their scalability, fault-tolerance, and load-balancing properties.
These same properties are highly desirable in a distributed computing environment, especially one that wants to use heterogeneous components.

We show that DHTs can be used not only as the framework to build a P2P file-sharing service, but as a P2P distributed computing platform.
We propose creating a P2P distributed computing framework using distributed hash tables, based on our prototype system ChordReduce.
This framework would make it simple and efficient for developers to create their own distributed computing applications.
Unlike Hadoop and similar MapReduce frameworks, our framework can be used both in both the context of a datacenter or as part of a P2P computing platform.  
This opens up new possibilities for building platforms to distributed computing problems.

One advantage our system will have is an autonomous load-balancing mechanism.
Nodes will be able to independently acquire work from other nodes in the network, rather than sitting idle.
More powerful nodes in the network will be able use the mechanism to acquire more work, exploiting the heterogeneity of the network.

By utilizing the load-balancing algorithm, a datacenter could easily leverage additional P2P resources at runtime on an as needed basis.
Our framework will allow MapReduce-like or distributed machine learning platforms to be easily deployed in a greater variety of contexts.


\end{abstract}

\chapter{Introduction}
\label{chapter:intro}
% % % layout
% % % Distributed Computing Challenges
% % % Qualities of DHTs
% % % Hypothesis = These Problems + These qualiteies -> solution
% % % Framework of what these solutions are and what they can do

%TODO: This chapter is a chapter long abstract/introduction.
%TODO: What I want to do, why you should care.
%TODO: Why you should care is applications.  What applications can you use be general but also give wide varity of specific examples
%TODO: Motivational selling point
%TODO: Get some structure!  Move background to background, this section should be purely motivation;  tell what dht's can do right now and what proposed uses there are and what YOU propose.  Also incorperate the challenges you need to overcome.  You can say these things are hard for DHTs and we'll tell you more in background.
%TODO:  Make sure you have roadmap:  Highlevel What you have done, what you plan to do.
%TODO:  A section with list of my papers (publications based on this work and ref the label for where comes up) (seperate list of stuff I've been involved in)




% What do I want to do?
\section{Objective}
Our goal is to create a framework to further generalize Distributed Hash Tables (DHTs) to be used for distributed computing.
Distributed computing platforms need to be scalable, fault-tolerant, and load-balancing.
%The ability to incorporate heterogeneous hardware is a definite benefit.
We will discuss what each of these mean and why they are important in section \ref{sec:challenges}, but briefly:

\begin{itemize}
	\item The system should be able to work effectively no matter how large it gets.
	As the system grows in size, we can expect the overhead to grow in size as well, but at an extremely slower rate.
	\item The more machine integrated into the system, the more we can expect to see hardware failures.
	The system needs to be able to automatically handle these hardware failures.
	\item Having a large number of machines to use is worthless if the amount of work is divided unevenly among the system.
	The same is true if the system hands out larger jobs to less powerful machines or smaller jobs to the more powerful machines.
	%\item We cannot assume that we be able to replace our broken machines with exact replicas, nor do we assume we would want to. 
\end{itemize}


These are many of the same challenges that Peer-to-peer (P2P) file sharing applications have.
Many P2P applications use DHTs to address these challenges, since DHTs are designed with these problems in mind.
We propose that DHTs can be used to create P2P distributed computing platforms that are completely decentralized.
%Rather than keys being assigned to some data, we can assign keys to tasks and automatically distribute those tasks to the responsible nodes
There would be no need for some central organized or scheduler to coordinate the nodes in the network.
Our framework would not be limited to only a P2P context, but could be applied in data centers, a normally centrally organized context.


A successful DHT-based computing platform would need to address the problem of dynamic load-balancing.
This is currently an unsolved\footnote{As far as we know.} problem. %I have to check a couple of papers of to confirm.
If an application can dynamically reassign work to nodes added at runtime, this opens up new options for resource management.
Similarly. if a distributed computation  is running too slow, new nodes can be added to the network during runtime or idle nodes can boot up more virtual nodes. %(now that I think of it this is two different but highly related problems: internal and external).

%Move this bit to the end?
Chapter \ref{chapter:background} will delve into how DHTs work and examine specific DHTs.
The remainder of the proposal will then discuss the work we plan on doing to demonstrate the viability of using DHTs for distributed computing.



% How I will do it is in the experiments chapter

% Why should you care?
\section{Applications of Distributed Hash Tables}

Distributed Hash Tables have been used in numerous applications:

\begin{itemize}
	\item \textbf{P2P file sharing} is by far the most prominent use of DHTs.  
	The most well-known application is BitTorrent \cite{bittorrent}, which is built on Mainline DHT \cite{mainline}.
	\item DHTs have been used for \textbf{distributed storage} systems \cite{CFS}.
	\item \textbf{Distributed Domain Name Systems} (DNS) have been built upon DHTs \cite{cox2002serving} \cite{pappas2006comparative}.
	Distributed DNSs are much more robust that DNS to orchestrated attacks, but otherwise require more overhead.
	%\item Distributed search (faroo)
	\item Distributed \textbf{machine learning} \cite{liparameter}.
	\item Many \textbf{botnets} are now P2P based and built using well established DHTs \cite{saad2011detecting}. 
	This is because the decentralized nature of P2P systems means there's no single vulnerable location in the botnet.
	\item \textbf{Live video streaming} (BitTorrent live) \cite{mol2009design}.
\end{itemize}

We can see from this list that DHTs are primarily used in P2P applications, but other applications, such as botnets, use DHTs for their decentralization.
We want to use DHTs primarily for their intuitive\footnote{Relatively intuitive, if you know how hash tables work.} way of organizing a distributed system.

We showed  \cite{chordreduce} that a DHT can be to create a distributed computing framework.
We used the same mechanism used in P2P applications that assigns nodes their location in the network to evenly distribute work among members of a DHT.
The most direct application of a DHT distributed computing framework is  a quick and intuitive way to solve embarrassingly parallel problems, such as:
\begin{itemize}
	\item Brute force cryptography.
	\item Genetic algorithms.
	\item Markov chain Monte Carlo methods.
	\item Random forests.
	\item Any problem that could be phrased as a MapReduce problem.
	
\end{itemize}
Unlike the current distributed applications which utilize DHTs, we want to create a complete framework which can be used to build decentralized applications.
We have found no existing projects that provide a means of building your own DHT or DHT based applications. %without a given DHT in mind at least


% So you're diong these things with this tool
\section{Why Use Distributed Hash Tables in Distributed Computing}
% Okay so this is all great, but what's special about a DHT
% First, let's talk about the problems with distributed computing

Using distributed hash tables for distributed computing isn't necessarily the most intuitive step.
To understand why we want to use DHTs for distributed computing, we will first examine some of the more prominent challenges in distributed computing.

\subsection{General Challenges of Distributed Computing}
\label{sec:challenges}

As we mentioned earlier, distributed computing platforms need to be scalable, fault-tolerant, and load-balancing.
We will look at these individually:


\begin{description}
	\item[Scalability] - Distributed computing platforms should not be completely static and should grow to accommodate new needs.
	However, as systems grow in size, the cost of keeping that system organized grows too.
	The challenge of scalability is designing a protocol that grows this organizational cost at an extremely slow rate.
	For example, a single node keeping track of all members of the system might be a tenable situation up to a certain point, but eventually, the cost becomes too high for a single node.
	%not sure about this line
	We want this organizational cost spread among many nodes to the point where this cost is insignificant. % 
	\item[Fault Tolerance]  
	The quality of fault-tolerance or \textit{robustness} \footnote{There is apparently a subtle difference between fault-tolerance and robustness, but we will use the two interchangeably here until we get told to stop.} means that the system still works even after a component breaks (or many components break).
	We want our platform to gracefully handle failures during runtime and be able to quickly reassign work to other workers.
	In addition, the network should be equally graceful in handling the introduction of new nodes during runtime.
	
	\item[Load-Balancing]
	The challenge of load balancing is to evenly distribute the work among nodes in the network.
	This is always an approximation; rarely  are there exactly enough pieces for  every node to get the same amount of work.
	The system needs an efficient set of rules for dividing arbitrary jobs into small pieces and sending those pieces to the nodes.
	
	A subproblem here is handling \textit{heterogeneity},\footnote{It could even be considered a problem in its own right.} or how should the system should handle different pieces of hardware with different amounts of computational power.
	
	
\end{description}
It should be noted that there is some crossover between these categories. 
For example, adding new nodes to the system needs to have a low organizational overhead (scalability) and will change the network configuration, which will need to be updated (fault-tolerance).


%move below
%One question we are particularly interested in answering touches on all three categories:  can we do load balancing during run-time?
%A goal we have is t


\subsection{How DHTs Address these Challenges}
%Without getting into the details of what a DHT is, what do they do?
Distributed Hash Tables are essentially distributed lookup tables.
DHTs use a consistent hashing algorithm, such as SHA-1 \cite{sha1}, to associate nodes and file identifiers with keys.  
These keys dictate where the nodes and files will be located on the network.
The connections between nodes are organized such that any node can efficiently lookup the value associated with any given key, even though the node only knows a small portion of the network.
We discuss the specifics this in Chapter \ref{chapter:background}.

Nearly every DHT was designed with large, P2P applications in mind, with millions of nodes in the network and new nodes entering and leaving all the time.
This has lead to DHTs being designed with specific qualities in mind.
\paragraph{Scalability}
The organization responsibility in DHTs is spread among all members of the network.
Each node only knows a small subset of the network,\footnote{Except for ZHT \cite{li2013zht}, which breaks this rule deliberately and with gusto by giving each node a full copy of the routing table.} but can use the nodes it knows to efficiently find any other node in the network.
Because each individual node only knows a small part of the network, the maintenance costs associated with organization are correspondingly small.

Using consistent hashing allows the network to scale up incrementally, adding one node at a time \cite{dynamo}.
In addition, each join operation has minimal impact on the network, since a node affects only its immediate neighbors on a join operation.
Similarly, the only nodes that need to react to a node leaving are its neighbors.
Other nodes can be notified of the missing node passively through maintenance or in response to a lookup.

There have been multiple proposed strategies for tackling scalability, and it is these strategies which play the greatest role in driving the variety of DHT architectures. 
Each DHT must strikes a balance between size of the lookup table and lookup time. 
The vast majority of DHTs choose to use $\lg(n)$ sized tables and  $\lg(n)$ hops. 
Chapter \ref{chapter:background} discusses these tradeoffs in greater detail and how they affect the each DHT.


\paragraph{Fault-Tolerance}
One of the most important assumptions of DHTs is that they are deployed on a non-static network.
DHTs are built to account for a high level of \textit{churn}.\footnote{Again, except for ZHT.}  
Churn refers to the disruption of routing caused by the constant joining and leaving of nodes.
In other words, the network topology is assumed to always be in flux
This is mitigated by a few factors.

First, the network is decentralized, with no single node acting as a single point of failure.
This is accomplished by each node in the routing table having a small portion of the both the routing table and the data stored on the DHT.

Second is that each DHT has an inexpensive maintenance processes that mitigates the damage caused by churn.
DHTs often integrate a backup process into their protocols so that when a node goes down, one of the neighboring nodes can immediately assume responsibility.
The join process also slightly disrupt the topology, as affected nodes must adjust their peerlists to accommodate the joiner. 

The last property is that the hashing algorithm used to distribute content evenly across the DHT also distributes nodes evenly across the DHT.  
This means that nodes in the same geographic region occupy vastly different locations in the network.  
If an entire geographic region is affected by a network outage, this damage is spread evenly across the DHT, which can be handled, rather than a contiguous portion.

%This property is the most important, as it deals with failure of entire sections of the network, rather than a single node.
%Recent research in using DHTs for High End Computing \cite{li2013zht} shows what can happen if we remove this assumption by working with an almost completely static network.

The fault tolerance mechanisms in DHTs also provide near constant availability for P2P applications.
The node that is responsible for a particular key can always be found, even when numerous failures or joins occur \cite{chord}.


\paragraph{Load-Balancing}


% Rather than avoiding failure by using a centralized manager, we embrace failure 

%We are not geared toward a datacenter-centric setup




%While most DHTs follow this scheme, there is some minor variation on how keys are generated.
%Some DHTs can or do use geographic information to generate keys.
%In VHash, keys are not static, but move according to a simplified spring model.




\section{Roadmap}

\subsection{Work I have already done}

\subsubsection{Published}

\subsubsection{Unpublished}

\subsection{Summary of Overall Plan}

The specifics are given in Chapter \ref{chapter:experiments}.

\section{Old stuff starts here}



Distributed Hash Tables (DHTs) are traditionally used as the backbone of structured Peer-to-Peer (P2P) file-sharing applications.
The largest such application is Bittorrent \cite{bittorrent}, which is built using Mainline DHT \cite{mainline},  a  derivative of Kademlia \cite{kademlia}.
The number of users on Bittorrent ranges from 15 million to 27 million users daily, with a turnover of 10 million users a day \cite{mainlineMeasure}.

Most research on DHTs assumes that DHTs will be used in the context of a large P2P file-sharing application (or at least, an application \textit{potentially} incorporating millions of nodes).
This leads the DHT to having particular qualities.
The application must be scalable and no single node knows every other node in the network.
The network must be able to handle members joining and leaving arbitrarily.
The resulting application must be agnostic towards hardware.
The network must be decentralized and split whatever burden there is equally among its members.

In other words, distributed hash tables provide scalability, fault-tolerance, and load-balancing to an application.
Recent applications have leveraged these qualities, since these qualities are desirable in many different frameworks.
For example, one paper \cite{Mateescu2011440} used a DHT as the name resolution layer of a large distributed database.
Research has also been done in using DHTs as an organizing mechanism in distributed machine learning \cite{liparameter}. 

We describe each of the aforementioned qualities and their ramifications below in sections \ref{subsec:ft}, \ref{subsec:lb}, \ref{subsec:scalability}, and \ref{subsec:hetero} .
While these properties are individually enumerated, they are greatly intertwined and the division between their impacts can be somewhat arbitrary.

\subsubsection{Scalability}
\label{subsec:scalability}
In order to maintain scalability, a DHT has to ensure that as the network grows larger:

\begin{itemize}
	\item Churn does not have a disproportionate overhead.  
	For example, in a 1000 node network, a joining or leaving node will affect only an extremely small subset of these nodes.
	\footnote{We will see that this requirement can  be  relaxed in  very specific cases \cite{li2013zht}.}
	\item Lookup request speeds (usually measured in hops) grow by a much smaller amount, possibly not at all.
\end{itemize}



\subsubsection{Fault-Tolerance}
\label{subsec:ft}


\subsubsection{Load Balancing}
\label{subsec:lb}
All Distributed Hash Tables use some kind of consistent hashing algorithm to associate nodes and file identifiers with keys.  
These keys are generated by passing the identifiers into a hash function, typically SHA-160.
The chosen hash function is typically large enough to avoid hash collisions and generates keys in a uniform manner. 
The result of this is that as more nodes join the network, the distribution of nodes in the keyspace becomes more uniform, as does the distribution of files.

However, because this is a random process, it is highly unlikely that each node will be spread evenly throughout the network.
This appears to be a weakness, but can be turned into an advantage in heterogeneous systems by using \textit{virtual nodes} \cite{dynamo} \cite{godfrey2005heterogeneity} .
When a node joins the network, it joins not at one position, but multiple virtual positions in the network \cite{dynamo}.
Using virtual nodes allows load-balance optimization in a heterogeneous network; more powerful machines can create more virtual nodes and handle more of the overall responsibility in the network.

DeCandia et al\. discussed various load balancing techniques that were tested on Dynamo \cite{dynamo}.  
Each node was assigned a certain number of tokens and the node would create a virtual node for each token.
The realization DeCandia et al\. had was that there was no reason to use the same scheme for data partitioning and data placement.
DeCandia et al\. introduced two new strategies which work off assigning nodes equally sized partitions.


% HELP
Under these schemes, each virtual node maps to an ID as before, but the patitions each node is responsible for are equally sized\footnote{Need help here}.

\paragraph{Heterogeneity}
\label{subsec:hetero}
Heterogenity presents a challenge for load balancing DHTs due to conflicting assumptions and goals. 
DHTs assume that members are usually going to be varied in hardware, while the load-balancing process defined in DHTs treats each node equally.
It is much simpler to treat each node as equal unit.
In other words, DHTs support heterogeneity, but do not attempt to exploit it.

This doesn't mean that heterogeneity cannot be exploited
Nodes can be given addition responsibilities manually, by running multiple instances of the P2P application on the same machine or creating more virtual nodes.
However, this is not a feasible option for any kind of truly decentralized system and would need to be done automatically.
There is no well-known mechanism to that exists to automatically  allocate virtual nodes on the fly \footnote{citation needed, although this can be similar to IRM}. 
A few options present themselves.  % and are discuessed in \ref{}
%add the above if the below is moved

%This might be the wrong place for this
One is to use adapt a request tracking mechanism, such as what is used in IRM, except instead of tracking file requests, it tracks requests that are directed to a particular (real) node. 
If a particular (real) node receives an inordinate amount of requests, the node doing the detecting suggests that the node obtain another token/create another virtual node.
Another strategy is to use the preference lists/successor predecessor lists, and observe the distribution of the workload, adjusting the virtual nodes based on that. 

Dynamic load balancing may not be essential to P2P file-sharing applications, but is absolutely essential to any kind of P2P distributed computation.
In our ChordReduce experiments, we observed that just approximating dynamic load-balancing by simulating high levels of churn noticeably improved results\footnote{We found this by accident, just by testing the network's fault tolerance in regards to a high level of churn}.




%\section{Different or subproblem: Certain DHTs are better at one application than another due to differences}
%\subsection{Design Differences Impacts}
%\subsection{Geometries}
%\subsection{Routing Table Construction}
%\subsection{Implementation Differences Impacts}
%\paragraph{Recursive or iterative seek}




%Add these bullets to the above paragraph
%\begin{itemize} 
%	\item DHTs can use consistent hashing supplemented by virtual nodes to efficiently load-balance.
%	The larger the network grows, the more evenly distributed the load becomes. 
%	\item DHTs are highly resilient to damage and can handle abnormally high rates of disruption.  
%	This is extremely desirable in any kind of distributed application %
%	\item Large-scale P2P file sharing applications have been using DHTs for a long time and
%    \item DHTs are extremely good if your problem is embarrassingly par
%    \item Heterogeneity
%\end{itemize}

\include{background}
\chapter{Completed Work}
\label{chapter:prev}

In this chapter, we will cover our completed research.
Our research has focused on novel implementation ]and Distributed Hash Tables (DHTs).
We have implemented and created an entirely new DHT called VHash \cite{vhash}, implemented MapReduce on Chord \cite{chordreduce}, and performed an analysis on an attack on DHTs.


\section{VHash}
DHTs all seek to minimize lookup time for their respective topologies.
This is done by minimizing the number of overlay hops needed for a lookup operation.
This is a good approximation for minimizing the latency of lookups, but does not actually do so, as each hop has a different amount of latency.
Furthermore, a network might need to minimize some arbitrary metric, such as energy consumption.

VHash is a multi-dimensional DHT that minimizes routing over some given metric.
It uses a fast approximation of a Delaunay Triangulation to compute the Voronoi tessilation of a multi-dimensional space.
%Approximated routing tables



Arguably all Distributed Hash Tables (DHTs) are built on the concept of Voronoi tessellations.
In all DHTs, a node is responsible for all points in the overlay to which it is the ``closest'' node.
Nodes are assigned a key as their location in some keyspace, based on the hash of certain attributes.
Normally, this is just the hash of the IP address (and possibly the port) of the node \cite{chord} \cite{kademlia} \cite{can} \cite{pastry}, but other metrics such as geographic location can be used as well \cite{ratnasamy2002ght}.

These DHTs have carefully chosen metric spaces such that these regions are very simple to calculate.
For example, Chord \cite{chord} and similar ring-based DHTs \cite{symphony} utilize a unidirectional, one-dimensional ring as their metric space, such that the region for which a node is responsible is the region between itself and its predecessor.

Using a Voronoi tessellation in a DHT generalizes this design.
Nodes are Voronoi generators at a position based on their hashed keys.
These nodes are responsible for any key that falls within its generated Voronoi region.

Messages get routed along links to neighboring nodes.
This would take $O(n)$ hops in one dimension.
In multiple dimensions, our routing algorithm (Algorithm \ref{alg:lookup}) is extremely similar to the one used in Ratnasamy et al.'s Content Addressable Network (CAN) \cite{can}, which would be $O(n^{\frac{1}{d}})$ hops.


\begin{algorithm}
	\caption{Lookup in a Voronoi-based DHT}
	\label{alg:lookup}
	\begin{algorithmic}[1]
		\State Given node $n$
		\State Given $m$ is a message addressed for $loc$
		\State $potential\_dests \leftarrow n \cup n.short\_peers \cup n.long\_peers$
		\State $c \leftarrow $ node in $ potential\_dests$ with shortest distance to $loc$
		\If{$c$ == $n$}
			\State \Return $n$
		\Else
			\State \Return $c.lookup(loc)$
		\EndIf
	\end{algorithmic}
\end{algorithm}


Efficient solutions, such as Fortune's sweepline algorithm \cite{fortune1987sweepline}, are not usable in spaces with 2 more dimensions.
As far as we can tell, there is no way efficient to generate higher dimension Voronoi tessellations, especially in the distributed Churn-heavy context of a DHT.
Our solution is the Distributed Greedy Voronoi Heuristic.

\subsection*{Distributed Greedy Voronoi Heuristic}
A Voronoi tessellation is the partition of a space into cells or regions along a set of objects $O$, such that all the points in a particular region are closer to one object than any other object.
We refer to the region owned by an object as that object's Voronoi region.
Objects which are used to create the regions are called Voronoi generators.
In network applications that use Voronoi tessellations, nodes in the network act as the Voronoi generators.

The Voronoi tessellation and Delaunay triangulation are dual problems, as an edge between two objects in a Delaunay triangulation exists if and only if those object's Voronoi regions border each other.
This means that solving either problem will yield the solution to both.
An example Voronoi diagram is shown in Figure \ref{voro-ex}.
For additional information, Aurenhammer \cite{voronoi} provides a formal and extremely thorough description of Voronoi tessellations, as well as their applications.


\begin{figure}
	\centering
	\includegraphics[width=0.5\linewidth]{figs/voronoi}
	\caption{An example Voronoi diagram for objects on a 2-dimensional space.  The black lines correspond to the borders of the Voronoi region, while the dashed lines correspond to the edges of the Delaunay Triangulation.}
	\label{voro-ex}
\end{figure}




The Distributed Greedy Voronoi Heuristic (DGVH) is a fast method for nodes to define their individual Voronoi region (Algorithm \ref{alg:dgvh}).
This is done by selecting the nearby nodes that would correspond to the points connected to it by a Delaunay triangulation.
The rationale for this heuristic is that, in the majority of cases, the midpoint between two nodes falls on the common boundary of their Voronoi regions.

%In addition, nodes should only have to compute their own Voronoi region, and possibly estimate those of its neighbors.
%Anything else is a waste of processing power.



\begin{algorithm} % make smaller
	\caption{Distributed Greedy Voronoi Heuristic}
	\label{alg:dgvh}
	\begin{algorithmic}[1]  % the numberis how many lines
		\State Given node $n$ and its list of $candidates$.
		\State Given the minimum $table\_size$
		\State $short\_peers \leftarrow$ empty set that will contain $n$'s one-hop peers
		\State $long\_peers \leftarrow$ empty set that will contain $n$'s two-hop peers
		\State Sort $candidates$ in ascending order by each node's distance to $n$
		\State Remove the first member of $candidates$ and add it to $short\_peers$
		\ForAll{$c$ in $candidates$}
		\State $m$ is the midpoint between $n$ and $c$
		\If{Any node in $short\_peers$ is closer to $m$ than $n$}
		\State Reject $c$ as a peer
		\Else
		\State Remove $c$ from $candidates$
		\State Add $c$ to $short\_peers$
		\EndIf
		\EndFor
		\While{$|short\_peers| < table\_size$ \textbf{and} $|candidates| >0$}
		\State Remove the first entry $c$ from $candidates$
		\State Add $c$ to $short\_peers$
		\EndWhile
		\State Add $candidates$ to the set of $long\_peers$
		\If{$|long\_peers| > table\_size^2$}
		\State $long\_peers \leftarrow$ random subset of $long\_peers$ of size $table\_size^2$
		\EndIf
	\end{algorithmic}
\end{algorithm}


During each cycle, nodes exchange their peer lists with a current neighbor and then recalculate their neighbors.
A node combines their neighbor's peer list with its own to create a list of candidate neighbors.
This combined list is sorted from closest to furthest.
A new peer list is then created starting with the closest candidate.
The node then examines each of the remaining candidates in the sorted list and calculates the midpoint between the node and the candidate.
If any of the nodes in the new peer list are closer to the midpoint than the candidate, the candidate is set aside.
Otherwise the candidate is added to the new peer list.


DGVH never actually solves for the actual polytopes that describe a node's Voronoi region.
This is unnecessary and prohibitively expensive \cite{raynet}.
Rather, once the heuristic has been run, nodes can determine whether a given point would fall in its region.

Nodes do this by calculating the distance of the given point to itself and other nodes it knows about.
The point falls into a particular node's Voronoi region if it is the node to which it has the shortest distance.
This process continues recursively until a node determines that itself to be the closest node to the point.
Thus, a node defines its Voronoi region by keeping a list of the peers that bound it.



\subsubsection{Algorithm Analysis}

DVGH is very efficient in terms of both space and time.
Suppose a node $n$ is creating its short peer list from $k$ candidates in an overlay network of $N$ nodes.
The candidates must be sorted, which takes $O(k\cdot\lg(k))$ operations.
Node $n$ must then compute the midpoint between itself and each of the $k$ candidates.
Node $n$ then compares distances to the midpoints between itself and all the candidates.
This results in a cost of

\[ k\cdot\lg(k) + k \text{ midpoints}  + k^{2} \text{ distances} \]


Since $k$ is  bounded by $\Theta(\frac{\log N}{\log \log N} )$ \cite{bern1991expected} (the expected maximum degree of a node), we can translate the above to

\[O(\frac{\log^{2} N}{\log^{2} \log N} )\]

In the vast majority of cases, the number of peers is equal to the minimum size of \textit{Short Peers}.
This yields $k=(3d+1)^{2}+3d+1$ in the expected case, where the lower bound and expected complexities are $\Omega(1)$.



\subsection{Experimental Results}
We evaluated the effectiveness of VHash and DGVH in creating a set of experiments.\footnote{Our results are pulled directly from \cite{dgvh} and \cite{vhash}.}
The first experiment showed how VHash could use DGVH to create a routing mesh.
Our second showed how optimizing for latency yielded better results than optimizing for least hops.

\subsubsection{Convergence}
Our first experiment examined how DGVH could be used to create a routing overlay and how well it performed in this task.
The simulation demonstrated how DGVH  formed a stable overlay from a chaotic starting topology after a number of cycles.
We compared our results to those in RayNet \cite{raynet}.
The authors of Raynet proposed a random $k$-connected graph would be a challenging initial configuration for showing a DHT relying on a gossip mechanism could converge to a stable topology.

In the initial two cycles of the simulation, each node bootstrapped its short peer list by appending 10 nodes, selected uniformly at random from the entire network.
In each cycle, the nodes gossiped , swapping peer list information.
They then ran DGVH using the new information.
We calculated the hit rate of successful lookups by simulating 2000 lookups from random nodes to random locations, as described in Algorithm \ref{alg:routesim}.
A lookup was considered successful if the network was able to determine which Voronoi region contained a randomly selected point.

Our experimental variables for this simulation were the number of nodes in the DGVH generated overlay and the number of dimensions.
We tested network sizes of 500, 1000, 2000, 5000, and 10000 nodes each in 2, 3, 4, and 5 dimensions.
The hit rate at each cycle is $\frac{hits}{2000}$, where $hits$ are the number of successful lookups.




\begin{algorithm}
	\caption{Routing Simulation Sample}
	\label{alg:routesim}
	\begin{algorithmic}[1]  % the number is how many
		\State $start \leftarrow$ random node
		\State$dest \leftarrow$ random set of coordinates
		\State $ans \leftarrow$ node closest to $dest$
		\If {$ans == start.lookup(dest)$}
		\State increment $hits$
		\EndIf
	\end{algorithmic}
\end{algorithm}

The results of our simulation are shown in Figure \ref{fig:conv}.
Our graphs show that a correct overlay was quickly constructed from a random configuration and that our hit rate reached 90\% by cycle 20, regardless of the number of dimensions.
Lookups consistently approached a hit rate of 100\% by cycle 30.
In comparison, RayNet's routing converged to a perfect hit rate at around cycle 30 to 35 \cite{raynet}.
As the network size and number of dimensions each increase, convergence slows, but not to a significant degree.

\begin{figure*}
	\centering
	\begin{tabular}{cc}

		\begin{subfigure}{0.5\columnwidth}
			\includegraphics[width=\linewidth]{figs/conv_d2}
			\caption{This plot shows the accuracy rate of lookups on a 2-dimensional network as it self-organizes.}
			\label{fig:conv2}
		\end{subfigure} &

		\begin{subfigure}{0.5\columnwidth}
			\includegraphics[width=\linewidth]{figs/conv_d3}
			\caption{This plot shows the accuracy rate of lookups on a 3-dimensional network as it self-organizes.}
			\label{fig:conv3}
		\end{subfigure} \\

		\begin{subfigure}{0.5\columnwidth}
			\includegraphics[width=\linewidth]{figs/conv_d4}
			\caption{This plot shows the accuracy rate of lookups on a 4-dimensional network as it self-organizes.}
			\label{fig:conv4}
		\end{subfigure} &


		\begin{subfigure}{0.5\columnwidth}
			\includegraphics[width=\linewidth]{figs/conv_d5}
			\caption{This plot shows the accuracy rate of lookups on a 5-dimensional network as it self-organizes.}
			\label{fig:conv5}
		\end{subfigure}

	\end{tabular}
	\caption{These figures show that, starting from a randomized network, DGVH forms a stable and consistent network topology.
		The Y axis shows the success rate of lookups and the X axis show the number of gossips that have occurred.
		Each point shows the fraction of 2000 lookups that successfully found the correct destination.}
	
	\label{fig:conv}

\end{figure*}

\subsubsection{Latency Distribution Test}
The goal of our second set of experiments was to demonstrate VHash's ability to optimize a selected network metric: latency in this case.
In our simulation, we used the number of hops on the underlying network as an approximation of latency.
We compared VHash's performance to Chord \cite{chord}.
As we discussed in Chapter \ref{chapter:background} Chord is a well established DHT with an $O(\log(n))$ sized routing table and $O(\log(n))$ lookup time measured in overlay hops.

Instead of using the number of hops on the overlay network as our metric, we are concerned with the actual latency lookups experience traveling through the \emph{underlay} network, the network upon which the overlay is built.
Overlay hops are used in most DHT evaluations as the primary measure of latency.
It is the best approach available when there are no means of evaluating the characteristics of the underlying network.
VHash is designed with a capability to exploit the characteristics of the underlying network.
With most realistic network sizes and structures, there is substantial room for latency reduction in DHTs.

For this experiment, we constructed a scale free network with 10000 nodes placed at random (which has an approximate diameter of 3 hops) as an underlay network \cite{cohen2000resilience} \cite{pastor2001epidemic} \cite{hagberg2004}.
We chose to use a scale-free network as the underlay, since  scale free networks model the Internet's topology \cite{cohen2000resilience} \cite{pastor2001epidemic}.
We then chose a random subset of nodes to be members of the overlay network.
Our next step was to measure the distance in underlay hops between 10000 random source-destination pairs in the overlay.
VHash generated an embedding of the latency graph utilizing a distributed force directed model, with the latency function defined as the number of underlay hops between it and its peers.

Our simulation created 100, 500, and 1000 node overlays for both VHash and Chord.
We used 4 dimensions in VHash and a standard 160 bit identifier for Chord.




\begin{figure}

\begin{subfigure}{\columnwidth}
\centering
	\includegraphics[width=0.5\linewidth]{figs/hist_100}
	\caption{Frequency of path lengths on Chord and VHash in a 100 node overlay.}
	\label{fig:hist100}
\end{subfigure}

\begin{subfigure}{\columnwidth}
	\centering
	\includegraphics[width=0.5\linewidth]{figs/hist_500}
	\caption{Frequency of path lengths on Chord and VHash in a 500 node overlay.}
	\label{fig:hist500}
\end{subfigure}

\begin{subfigure}{\columnwidth}
	\centering
	\includegraphics[width=0.5\linewidth]{figs/hist_1000}
	\caption{Frequency of path lengths on Chord and VHash in a 1000 node overlay.}
	\label{fig:hist1000}
\end{subfigure}

\caption{Figures \ref{fig:hist100}, \ref{fig:hist500}, and \ref{fig:hist1000} show the difference in the performance of Chord and VHash for 10,000 routing samples on a 10,000 node underlay network for differently sized overlays.
The Y axis shows the observed frequencies and the X axis shows the number of hops traversed on the underlay network.
VHash consistently requires fewer hops for routing than Chord.}
\label{fig:hist}

\end{figure}




Figure \ref{fig:hist} shows the distribution of path lengths measured in underlay hops in both Chord and VHash.
VHash significantly outperformed Chord and considerably reduced the underlay path lengths in three network sizes.

We also sampled the lookup length measured in overlay hops for a 1000 sized Chord and VHash network.
As seen in Figure \ref{fig:histover}, the paths measured in overlay for VHash were significantly shorter than those in Chord.
In comparing the overlay and underlay hops, we find that for each overlay hop in Chord, the lookup must travel 2.719 underlay hops on average; in VHash, lookups must travel 2.291 underlay hops on average for every overlay hop traversed.

Recall that this work is based on scale free networks, where latency improvements are difficult.
An improvement of 0.4 hops over a diameter of 3 hops is significant.
VHash has on average less overlay hops per lookup than Chord, and for each of these overlay hops we consistently traverse more efficiently across the underlay network.
\begin{figure}
	\centering
	\includegraphics[width=0.5\linewidth]{figs/hist_overlay_4d}
	\caption{Comparison of Chord and VHash in terms of overlay hops.  Each overlay has 1000 nodes.  The Y axis denotes the observed frequencies of overlay hops and the X axis corresponds to the path lengths in overlay hops.}
	\label{fig:histover}
\end{figure}




\subsection{Remarks}

Voronoi tessellations have a wide potential for applications in ad-hoc networks, massively multiplayer games, P2P, and distributed networks.
However, centralized algorithms for Voronoi tessellation and Delaunay triangulation are not applicable to decentralized systems.
In addition, solving Voronoi tessellations in more than 2 dimensions is computationally expensive.

We created a distributed heuristic for Voronoi tessellations in an arbitrary number of dimensions.
Our heuristic is fast and scalable, with a expected memory cost of $(3d+1)^{2}+3d+1$ and expected maximum runtime of O$(\frac{\log^{2} N}{\log^{2} \log N} )$.

We ran two sets of experiments to demonstrate VHash's effectiveness.
Our first set of experiments demonstrated that our heuristic is reasonably accurate  and our second set demonstrates that reasonably accurate is sufficient to build a P2P network which can route accurately.
Our second experiment showed that VHash  could significantly reduced the latency in Distributed Hash Tables.

%Our next step is to create a formal protocol and implementation for a Voronoi tessellation-based distributed hash table using DGVH.
%We can use this DHT to choose certain metrics we want to measure, such as latency, or trust, and embed that information as part of a node's identity.
%By creating an appropriate distance measurement, we can route along some path that minimizes or maximizes the desired metric.
%Rather than create an overlay that minimizes hops, we can have our overlay minimize latency, which is the actual goal of most routing algorithms.

%\subsection*{Peerlist and Topology}
%Like CAN \cite{can}, VHash tracks only neighbors for it's peers.
%We enforce a lower limit on the size of the peerlist to avoid nodes being


%\subsection*{Joining}


%\subsection*{Fault Tolerance}




\section{ChordReduce}
DHTs have received a great deal of research due to their popularity as the backbone for structured P2P system primarily used for file-sharing.
There are two recent and fairly open questions that we want to examine.

\begin{enumerate}
	\item How and in what contexts  can DHTs effectively be used for distributed computations?
	\item How can nodes in a DHT autonomously detect and redistribute imbalances in the network load?
\end{enumerate}

This section will talk about ChordReduce, in which we have made preliminary attempts to answer the first question by implementing MapReduce on a DHT.
The next section, Section \ref{sec:auto-load-bal}, discusses the our preliminary work in answering the second.

\subsection{Background and Motivation}

Distributed computing is a current trend and will continue to be the approach for intensive  applications.
We see this in the development of cloud computing \cite{p2p-cloud}, volunteer computing frameworks like BOINC \cite{anderson2004boinc} and Folding@Home \cite{larson2002folding}, and MapReduce  \cite{mapreduce}.
Google's MapReduce  in particular has rapidly become an integral part in the world of data processing.
A user can use MapReduce to take a large problem, split it into small, equivalent tasks and send those tasks to other processors for computation.
The results are sent back to the user and combined into one answer.

Popular platforms for MapReduce, such as Hadoop \cite{hadoop}  \cite{shvachko2010hadoop}, are explicitly designed to be used in large datacenters \cite{hadoopAssumptions} and the majority of research has been focused there.
However, as we have previously mentioned, there are notable issues with a centralized design.

First and foremost is the issue of fault-tolerance.
Centralized designs have a single point of failure \cite{shvachko2010hadoop}.
So long as all computing resources are located in one geographical area or rely on a particular node, a power outage or catastrophic event could interrupt computations or otherwise disrupt the platform \cite{babaoglu2014people}.

A centralized design assumes that the network is relatively unchanging and may not have mechanisms to handle node failure during execution or, conversely, cannot speed up the execution of a job by adding additional workers on the fly.
Many environments also anticipate a certain degree in homogeneity in the system.
Finally deploying these systems and developing programs for them has an extremely steep learning curve.

There is no reason that these assumptions need to be the case for MapReduce, or for many distributed computing frameworks in general.
Moving away from the data center context opens up more possibilities for distributed computing, such as P2P clouds \cite{p2p-cloud}.
However, without a centralized framework, the network needs some kind of protocol to organize the various components in the network.
As part of our research, we developed a highly robust and distributed MapReduce framework based on Chord, called ChordReduce \cite{chordreduce}.

There a number of reasons to used a DHT as the protocol for a distributed computing platform.
First, nodes ID and their location in the network are strongly bound to what data they are responsible for, such that any node can lookup which node is responsible a particular piece of data.
This obviates the need for a centralized organizer to maintain this bit of metadata or assign backups for data, as nodes can do this autonomously.
DHTs assume that the network is heterogeneous, rather than homogeneous.
They have been used for over a decade for P2P file-sharing applications for these reasons.


%\subsubsection{P2P cloud}
%http://www.cs.unibo.it:443/pub/TR/UBLCS/2011/2011-10.pdf

%Clouds and Volunteer Computing platforms are different.
%Clouds@home
%Nanodatacenter

\subsection{What is ChordReduce?}

ChordReduce \cite{chordreduce} is designed as a more abstract framework for MapReduce, able to run on any arbitrary distributed configuration.
ChordReduce leverages the features of distributed hash tables to handle distributed file storage, fault tolerance, and lookup.
We designed ChordReduce to ensure that no single node is a point of failure and that there is no need for any node to coordinate the efforts of other nodes during processing.

\begin{figure}
	\centering
	\includegraphics[width=0.8\linewidth]{figs/CR_architecture}
	\caption{A simple architectural diagram of the layers of ChordReduce.}	
	\label{fig:cr_arch}
\end{figure}



\subsubsection{File System}
Our central design philosophy was to leverage as many features of the underlying DHT as possible.
For example, we do not need to create a new distributed file system, as we can just use the DHT to hash file identifiers and use the DHT to store the file at the node responsible for that key.

If the file is large, we can instead use Dabek et al.'s Cooperative File System or CFS \cite{CFS}.
In CFS, files are split into approximately equally sized blocks.
Each block is treated as an individual file and is assigned a key equal to the hash of its contents.
The block is then stored at the node responsible for that key.
The node that would normally be responsible for the whole file instead stores a \textit{keyfile}.
The keyfile is an ordered list of the keys corresponding to the files' block and is created as the blocks are assigned their respective keys.
When the user wants to retrieve a file, they first obtain the keyfile and then request each block specified in the keyfile.


\subsubsection{Computation}
ChordReduce treats each task or target computation as an object of data.
This means we can distribute them in the same manner as files and rely on the protocol to route them and provide robustness.


In ChordReduce, each node takes on responsibilities of both a worker node and master node, in the same way that a node in a P2P file-sharing service acts as both a client and a server.
A user starts a job, contacts a node at a specified hash address and provides it with the tasks.
This address can be chosen arbitrarily or be a known node in the ring.
We call this node the \textit{stager} for this particular job.

The job of the stager is to divide the work into \emph{data atoms}, the smallest units of work.
This might represent a block of text, the result of a summation for a particular intermediate value, or a subset of items to be sorted.
The specifics of how to divide the work are defined by the user in a \emph{stage} function.
The data atoms also contain user created Map and Reduce functions.

If the user wants to perform a MapReduce job on a particular file on the network, the stager locates the keyfile for the data and creates a data atom for each block in the file.
Each data atom is then sent to the node responsible for their corresponding block.
When the data atom reaches its destination node, that node retrieves the necessary data and applies the Map function.
The results are stored in a new data atom,  which are then sent back to the stager's hash address (or some other user defined address).
This will take $O(\lg n)$ hops traveling over Chord's fingers.
At each hop, the node waits a predetermined minimal amount of time to accumulate additional results (In our experiments, this was 100 milliseconds).
Nodes that receive at least two results merge them using the Reduce function.
The results are continually merged until only one remains at the hash address of the stager.


Some MapReduce jobs do not rely on a file stored on the network, such as a Monte-Carlo approximation.
In this situation, the data atoms define a task to be run multiple times.
In this case, the created data atoms are then each given a random hash and sent to the node responsible for that hash address, guaranteeing they are evenly distributed throughout the network.
From there, the execution is identical to the above scenario.


%Once the data atoms are sent out, the stager's job is done and it behaves like any other node in the network. The staging period is the only time ChordReduce is vulnerable to churn, and only if the stager leaves the ring in the middle of sending out data atoms.  The user would get some results back, but only for the data the stager managed to send out.

Once all the Reduce tasks are finished, the user retrieves the results from the node at the stager's address.
This may not be the stager himself, as the stager may no longer be in the network.
The stager does not need to collect the results himself, since the work is sent to the stager's hash address, rather than the stager itself.
Thus, the stager could quit the network after staging, and both the user and the network would be unaffected by the change.
This eliminates the need for a single specific root node and provides fault tolerance.
% Here, we are leverging two features. First, we use the automatic assignment of responsibility to automatically route the data to the sucessor.  %Second, the same process Chord uses to backup files is used to backup the intermediate data.

Similar precautions are taken for nodes working on Map and Reduce tasks.
Those tasks are backed up by a node's successor, who will run the task if the node leaves before finishing its work (e.g. the successor loses his predecessor).
The task is given a timeout by the node.
If the backup node detects that the responsible node has failed, he starts the work and backs up again to \emph{his} successor.
Otherwise, the data is tossed once the timeout expires.
This is done to prevent a job being submitted twice.

An advantage of our system is the ease of development and deployment.
The developer does not need to worry about distributing work evenly, nor does he have to worry about any node in the network going down.
The stager does not need to keep track of the status of the network.
The underlying Chord ring handles that automatically.
If the user finds they need additional processing power during runtime, they can boot up additional nodes, which would automatically be assigned work based on their hash value.
If a node goes down while performing an operation, his successor takes over for him.
This makes the system extremely robust during runtime.


\subsubsection{Robustness}
Since the system is distributed, we need to assume that any member of the network can go down at any time.
When a node fails or leaves Chord, the failed node's successor will become responsible for all of the failed nodes keys.
Likewise, each node in the ChordReduce network relies on their successor to act as a backup.

To prevent data from becoming irretrievable, each node periodically sends backups to its successor.
In order to prevent a cascade of backups of backups, the node only sends data that it is currently responsible for.
This changes as nodes enter and leave the network.
If a node's successor leaves, the node sends a backup to his new successor.
If the node fails, the successor is able to take his place almost immediately.
The backup  scheme is used to not only backup files, but the computational tasks as well.

This procedure prevents any single node failure or sequences of failures from harming the network.
Only the failure of multiple neighboring nodes poses a threat to the network's integrity.
Recall that a node's ID in the network does not map to a geographical locations.
Any failure that affects multiple nodes simultaneously would be spread uniformly throughout the network.
This means if successive nodes to fail simultaneously, they do so independently.

Assume node has failure rate $r < 1$ and that the each node backs up their data with $s$ successive nodes downstream.
If one of these nodes fail, the next successive node takes its place and the next upstream node becomes another backup.
This ensures there will always be $s$ backups.
The integrity of the ring would only be jeopardized if $s+1$ successive nodes failed simultaneously.
The chances of this would be $r^s+1$, as each failure would be independent.


A final consequence of this is load-balancing during runtime.
When a joining node $n$ find his successor, $n$ asks if the successor is holding any data $n$ should be responsible for.
The successor looks at all the data $n$ is responsible for and sends it to $n$.
The successor maintains this data as a backup for $n$.
Because Map tasks are backed up in the same manner as data, a node can take the data and corresponding tasks he is responsible for and begin performing Map tasks immediately.

\subsection{Experiments}

We created a prototype of ChordReduce in order to demonstrate it was a viable framework \cite{chordreduce}.
To acheive this, we had to show ChordReduce had these three properties:
\begin{enumerate}
	\item ChordReduce provided significant speedup during a distributed job.
	\item ChordReduce scaled.
	\item ChordReduce handled churn during execution.
\end{enumerate}


We needed to demonstrate speedup by showing that a job handled by multiple workers generally finished sooner than the same job handled by a single worker.  
More formally we must establish that $\exists n$ such that $T_{n} < T_{1}$, where $T_{n}$ is the amount of time it takes for $n$ nodes to finish the job.

To establish scalability, we had to show that the cost of distributing the work grows logarithmically with the number of workers.  
We needed to demonstrate that the larger the job is, the more nodes can work on the problem before we begin experiencing diminishing returns. 
This can be stated as $$T_{n} = \frac{T_{1}}{n} + k \cdot \log_{2}(n)$$, where $\frac{T_{1}}{n}$ is the amount of time the job would take when distributed in an ideal universe and $k \cdot \log_{2}(n)$ is network induced overhead, $k$ being an unknown constant dependent on network latency and available processing power.

Finally, to demonstrate robustness, we had to show that ChordReduce can handle arbitrary node failure in the ring and that such failures minimally impacted the overall speed of computation

\subsubsection{Setup}


To stress test our framework, we ran a Monte-Carlo approximation of $\pi$. 
This process is analogous to having a square containing  the top-right quadrant of a circle (Figure \ref{fig:dartboard}), and then throwing darts at random locations.  
Counting the ratio of darts that land inside the circle to the total number of throws yields an approximation of $\frac{\pi}{4}$.  
The more darts we throw, i.e. the more samples that are taken, the more accurate the approximation.\footnote{This is not and was not intended to be a particularly good approximation of $\pi$.
Each additional digit of accuracy requires increasing the number of samples taken by an order of magnitude.}


We chose this experiment for a number of reasons. 
The job is extremely easy to distribute.  
This also made it very easy to test scalability. 
By doubling the amount of samples, we can double the amount of work each node gets.  
We could also test the effectiveness of distributing the job among different numbers of workers.


\begin{figure}
	\centering
	\includegraphics[width=0.5\linewidth]{figs/dartboard}
	\caption{The "dartboard." The computer throws a dart by choosing a random $x$ and $y$ between 0 and 1.  If $x^{2} + y^{2} < 1^{2} $, the dart landed inside the circle.  $A$ and $B$ are darts that landed inside the circle, while $C$ did not.}
	\label{fig:dartboard}
\end{figure}


Each Map job was defined as the number of throws the node must make and yielded total number of throws and the number of throws that landed inside the circular section.  
Reducing these results was then a matter of adding the respective fields together. 

We ran our experiments using Amazons's Elastic Compute Cloud (EC2) service.  
Amazon EC2 allows users to purchase an arbitrary amount of virtual machines by the hour. 
Each node was an individual EC2 small instance with a preconfigured Ubuntu 12.04 image.  
%These instances were capable enough to provide constant computation, but still weak enough that they would be overwhelmed by traffic on occasions, creating a constant churn effect in the ring.  

Once started, nodes retrieved the latest version of the code and run it as a service, automatically joining the network.  
We could choose any arbitrary node as the stager and tell it to run the MapReduce process. 
We found that the network was robust enough that we could take a node we wanted to be the stager out of the network, modify its MapReduce test code, have it rejoin the network, and then run the new code without any problems. 
Since only the stager had to know how to create the Map tasks, the other nodes did not have to be updated and execute the new tasks they are given.

We ran our experiments on groups of 1, 10, 20, 30, and 40 workers, which generated a $10^{8}$ sample set and a $10^{9}$ sample set.
Additionally, we gathered data on a $10^{7}$ sample set using 1, 5, 10, 20, 30 workers.  
To test churn, we ran an experiment where each node had an equal chance of leaving and joining the network and varied the level of churn over multiple runs.  

We also utilized a subroutine we wrote called $plot$, which sends a message sequentially around the ring to establish how many members there are.  
If $plot$ failed to return in under a second, the ring was experiencing structural instability.

\begin{figure}
	\centering
	\includegraphics[width=0.5\linewidth]{figs/expTime}
	\caption{Our results show that for a sufficiently large job, it was almost always preferable to distribute it.  
		When the job is too small, such as with the $10^{7}$ data set, our runtime is dominated by the overhead.  
		Our results are what we would expect when overhead grows logarithmically to the number of workers.}
	\label{fig:expTime}
\end{figure}


\begin{figure}
	\centering
	\includegraphics[width=0.5\linewidth]{figs/expSpeed}
	\caption{The larger the size of the job, the greater the gains of distributing with ChordReduce.  In addition, the larger the job, the more workers can be added before we start seeing diminishing returns.  This demonstrates that ChordReduce is scalable.}
	\label{fig:expSpeed}
\end{figure}

Figure \ref{fig:expTime} and Figure \ref{fig:expSpeed} summarize the experimental results of job duration and speedup.  
Our default series was the $10^{8}$ samples series.  
On average, it took a single node 431 seconds, or approximately 7 minutes, to generate $10^{8}$ samples.  
Generating the same number of samples using ChordReduce over 10, 20, 30, or 40 nodes was always quicker.  
The samples were generated fastest when there were 20 workers, with a speedup factor of 4.96, while increasing the number of workers to 30 yielded a speedup of only 4.03.  
At 30 nodes, the gains of distributing the work were present, but the cost of overhead ($k \cdot \log_{2}(n)$) had more of an impact.  
This effect is more pronounced at 40 workers, which experienced  a speedup of only 2.25.

Since our data showed that approximating $\pi$ on one node with $10^{8}$ samples took approximately 7 minutes, collecting $10^{9}$ samples on a single node would take 70 minutes at minimum.  
Figure \ref{fig:expSpeed} shows that the $10^{9}$ set gained greater benefit from being distributed than the $10^{8}$ set, with the speedup factor at 20 workers being 9.07 compared to 4.03.  
In addition, dimishing returns only showed up at 40 workers, compared with the $10^{8}$ data set, which began its drop off at 30 workers.
This behavior showed that the larger the job being distributed, the greater the gains of distributing the work using ChordReduce.

The $10^{7}$ sample set confirms that the network overhead is logarithmic.  
At that size, it is not effective to run the job concurrently and we start seeing overheard acting as the dominant factor in runtime.  
This matches the behavior predicted by our equation, $T_{n} = \frac{T_{1}}{n} + k \cdot \log_{2}(n)$. 
For a small $T_{1}$, $\frac{T_{1}}{n}$  approaches 0 as $n$ gets larger, while $k \cdot \log_{2}(n)$, our overhead, dominates the sample.  
The samples from our data set fit this behavior, establishing that our overhead increases logarithmically with the number of workers.


\begin{figure}
	\centering
	\includegraphics[width=0.5\linewidth]{figs/projTime}
	\caption{The projected runtime using ChordReduce for differently sized jobs.  Each curve projects the expected behavior for job that takes a single worker the specified amount of time.}
	\label{fig:projTime}
\end{figure}

\begin{figure}
	\centering
	\includegraphics[width=0.5\linewidth]{figs/projSpeed}
	\caption{The projected speedup for different sized jobs. }
	\label{fig:projSpeed}
\end{figure}

Since we were able to establish that $T_{n} = \frac{T_{1}}{n} + k \cdot \log_{2}(n)$, we created an estimate how long a job that takes an arbitrary amount of time to run on a single node would take using ChordReduce.  
Our data points indicated that the mean value of $k$ for this problem was 36.5.  
Figure \ref{fig:projTime} shows that any jobs that would take more than $10^{4}$ seconds for single worker, we can expect there would still be benefit to adding an additional worker, even when there are already 5000 workers already in the ring.  
Figure \ref{fig:projSpeed} further emphasizes this. Note that as the jobs become larger, the expected speedup from ChordReduce  approaches linear behavior.


Table \ref{tab:churnSpeed} shows the experimental results for different rates of churn. 
We discus these experimental results and their significance in Section \ref{sec:auto-load-bal}.
\begin{table}
	\centering
	\begin{tabular}{|r|r|r|}
		\hline
		Churn rate per second & Average runtime (s) & Speedup vs 0\% churn\\ \hline{}
		0.8\% & 191.25 & 2.15 \\ \hline
		0.4\% & 329.20 & 1.25 \\ \hline
		0.025\% & 431.86 & 0.95 \\ \hline
		0.00775\%  & 445.47 & 0.92 \\ \hline
		0.00250\% & 331.80  &  1.24 \\ \hline
		0\% & 441.57 & 1.00 \\ \hline
	\end{tabular}
	\caption{}
	\label{tab:churnSpeed}
\end{table}


%These results show the system  is relatively insensitive to churn.  
%We started with 40 nodes in the ring and generated $10^{8}$ samples while experiencing different rates of churn, as specified in Table \ref{churnSpeed}.  
%At the 0.8\% rate of churn, there is a 0.8\% chance each second that any given node will leave the network followed by another node joining the network at a different location. 
%The joining rate and leaving rate being identical is not an unusual assumption to make \cite{marozzo2012p2p} \cite{load}.

%Our testing rates for churn are an order of magnitude higher than the rates used in the P2P-MapReduce simulation  \cite{marozzo2012p2p}.  In their paper, the highest rate of churn was only 0.4\% per minute. Because we were dealing with fewer nodes, we chose larger rates to demonstrate that ChordReduce could effectively handle a high level of churn.


Our experiments show that for a given problem, ChordReduce can effectively distribute the problem, yielding a substantial speedup.  
Furthermore, our results showed that the larger the problem is, the more workers could be added before diminishing returns were incurred.  
During runtime, we experienced multiple instances where $plot$ would fail to run and the stager would report socket errors, indicating that it had lost connection with a node in the ring.  Despite this turbulence, every node managed to reestablish connection with each other and report back all the data.  
This further demonstrated that we were able to handle the churn in the network.


\subsection{Heterogeneity Calculation}

One of the advantages to using homogeneous hardware is that each machine, each core, each node is the same.
To evenly distribute the workload, you just have to give each machine the same amount of work.
While a

This is more difficult in a heterogeneous system, such as ChordReduce, as each machine can shoulder a different amount of work.
How do we distribute work evenly across a heterogeneous system?

We can solve this by adjusting the amount of nodes representing each machine in the network.
Machines that can handle a larger load create more nodes in the network.
Besides solving the heterogeneous load-balancing problem, increasing the number of nodes in the system increases the overall load-balancing of the system.

The question we must answer is ``how?''
We need to create some unit of measurement for a distributed computing system and research if any other researchers have asked this problem.
Furthermore, this measurement might need to be relative to other nodes in the network, since the only basis for comparison are the scores of the peers. 
Finally, this process needs to be handled autonomously by each node.
This is part of the proposed work discussed in Chapter \ref{chapter:experiments}.
\section{Autonomous Load Balancing}
\label{sec:auto-load-bal}


During our experiments testing the capabilities of ChordReduce, we experienced a significant and completely unexpected anomaly while testing churn.
One of the things previous research \cite{marozzo2012p2p}  \cite{leemap} in the same area we felt we needed to explore better was how a completely decentralized computation could handle churn.
Now, despite our initial prototype having numerous bugs and only able to handle small networks, we were fairly certain of it's ability to handle churn.

Marozzo et al.\ \cite{marozzo2012p2p} tested their network using churn rates of 0.025\%, 0.05\%, 0.1\%, 0.2\%, and 0.4\% per minute.
The churn rate of $cr << 1$ per minute means that each minute on average, $cr \cdot n$ nodes leave the network and $cr \cdot n$  new nodes join the network.\footnote{It is standard practice to assume the joining rate and leaving rate are equal.}
This could effectively be thought of as each node flipping a weighted coin every minute.
When the coin lands on tails, the node leaves.
A similar process happens for nodes wanting to join the network.

We wanted the robustness of our system to be beyond reproach, so we tested at rates from 0.0025\% to 0.8\% \textbf{\textit{per second}}, 120 times the fastest rate used to test P2P-MapReduce.
This is an absurdly fast and unrealistic speed, the only purpose of which was to cement the fault tolerance of the system.
Since we were testing ChordReduce on Amazon's EC2 and paying per instance per hour, we limited the number of nodes.
Rather than having a pool of nodes waiting to join the network, we conserved our funds by having leaving nodes immediately rejoin the network under a new IP/port combo.
The meant our churn operation was essentially a simultaneous leave and join.


What we found was that jobs on ChordReduce finished twice as fast under the unrealistic levels churn (0.8\% per second) than no churn (Table \ref{tab:churnSpeed}).
This completely mystified us.
Churn is a disruptive force; how can it be aiding the network?

\subsection*{Hypothesis}
We hypothesize this was due to the number of data pieces (larger) vs the number of workers (smaller).
There were more workers than there were pieces of data, so some workers ended up with more data than others in the initial distribution.
This means that there was some imbalance in the way data was distributed among nodes.
This was \textit{further} exacerbated by small number of workers distributed over a large hash space, leading some nodes to have larger swaths of responsibility than others.

Given this setup, without any churn, the operation would be:
Workers get triggered, they start working, and the ones with little work finish their work quickly, and the network waits for the node with higher loads of work.

Its important to note here that the work in ChordReduce was performed atomically, a piece at a time.
When a node was working on a piece, it informed it's successor, then informed them when it finished.
These pieces of work were also small, possibly too small.

As mentioned previously, under our induced experimental churn, we had the nodes randomly fail and immediately join under a new IP/port combination, which yields a new hash.
The failure rates were orders of magnitude higher than what would be expected in a ``real'' (nonexperimental) environment.
The following possibilities could occur:
\begin{itemize}
	\item A node without any active jobs leaves.
	It dies and and comes back with a new port chosen.
	This new ID has a higher chance of landing in a larger region of responsibility (since new joining nodes have a greater chance of hashing to a larger region than a smaller).
	In other words, it has a (relatively) higher chance of moving into an space where it becomes acquires responsibility for enqueued jobs.
	The outcomes of this are:
	\begin{itemize}
		\item The node rejoins in a region and does not acquire any new jobs.
		This has no impact on the network (Case I).
		\item The node rejoins in a region that has jobs waiting to be done.
		It acquires some of these jobs.
		This speeds up performance (Case II).
	\end{itemize}
	\item A node with active jobs dies.
	It rejoins in a new space.
	The jobs were small, so not too much time is lost on the active job, and the enqueued jobs are backed up and the successor knows to complete them.
	However, the node can rejoin in a more job-heavy region and acquire new jobs.
	The outcomes of this are:
	\begin{itemize}
		\item A minor negative impact on runtime and load balancing (since the successor has more jobs to handle) (Case III).
		\item A possible counterbalance in load balancing by acquiring new jobs off a busy node (Case IV).
	\end{itemize}
\end{itemize}

The longer the nodes work on the jobs, the more nodes finish and have no jobs.
This means as time increases, so do the occurrences of Case I and II.


This leads us to two hypotheses:
\begin{itemize}
	\item Deleting nodes motivates other nodes to work harder to avoid deletion (a ``beatings will continue until morale improves'' situation).
	\item Our high rate of churn was dynamically load-balancing the network.
	It appears even the smallest effort of trying to dynamically load balance, such as rebooting random nodes to new locations, has benefits for runtime.
	Our method is a poor approximation of dynamic load-balancing, and it still shows improvement.
\end{itemize}

The first hypothesis is mentally pleasing to anyone who has tried to create a distributed system, but lacks rigor.
We still have to verify the existence of this phenomena in an independent experiment, and establish that it does is part of the proposed work (Chapter \ref{chapter:experiments}).


Once we have established that it does exist, we need a better load-balancing strategy than randomly inducing.
We want nodes to have a precomputed list of locations in which they can insert nodes to perform load-balancing on an ad-hoc basis during runtime.
This precomputed list ties directly into the security research on DHTs we have done \cite{sybil-analysis}.


%The questions and goals here are straightforward:
%\begin{itemize}
%	\item Further establish the phenomena exists.
%	\item We stumbled across this phenomena with a brute force method and still got promising results.
%	Can we create a more accurate and mean
%	\item Can this phenomena be stochastically modeled or otherwise predicted via theoretical analysis?
%	\item In what contexts can this be used for DHTs?  Distributed computing?  Replication for file sharing?

%\end{itemize}






\section{Sybil Attacks and Injection}
One of the key properties of structured peer-to-peer (P2P) systems is the lack of a centralized coordinator or authority.
P2P systems remove the vulnerability of a single point of failure and the susceptibility to a denial of service attack \cite{sybil}, but in doing so, open themselves up to new attacks.

Completely decentralized P2P systems are vulnerable to \textit{Eclipse attacks}, whereby an attacker completely occludes healthy nodes from one another.
This prevents them from communicating without being intercepted by the adversary.
Once an Eclipse attack has taken place, the adversary can launch a variety of crippling attacks, such as incorrectly routing messages or returning malicious data \cite{srivatsa2004vulnerabilities}.

One way to accomplish this attack is to perform a \emph{Sybil attack} \cite{sybil}.
In a Sybil attack, the attacker masquerades as multiple nodes, effectively over-representing the attacker's presence in the network to maximize the number of links that can be established with healthy nodes.
If enough malicious nodes are injected into the system, the majority of the nodes will be occluded from one another, successfully performing an Eclipse attack.

%We discovered injecting replicas is easy and simple in P2P networks, we use a Sybil attack.
%This was the focus of my Data Security project
%I hypothesize we can Sybil attacks for improving load balancing on demand.

This vulnerability is well known \cite{dhtsec}.
Extensive research has been done assessing the damage an attacker can do after establishing themselves \cite{srivatsa2004vulnerabilities}.
%Especially when a hash value to assign neighbors
Little focus has been done on examining how the attacker can establish himself in the first place and precisely how easily the Sybil attack can be accomplished.

We did a project that focused on looking at the computational and memory costs of performing the Sybil attack.
The computation costs turn out to be fairly trivial and can be precomputed based on how IDs are assigned, a process we named \textit{mashing}.
If a node obtains their ID via an IP/Port combination, and we limit an attacker to using only ephemeral IP addresses (16383 total), the per node cost of mashing is quite low.
Per node, it takes 48 milliseconds to mash 16383 IP/Port combinations and only 352 kilobytes to store this information after precomputing it.

An attacker would do this for each of his nodes, then join the network and insert as many Sybils as possible.
We calculated that it would take only 1221 IP addresses to compromise 50\% of the links in a 20,000,000 node network \cite{sybil-analysis}.


\subsection{Experiments}
The primary experiment of our project was simulating the complete eclipse a network using a Sybil attack, starting with a single malicious node \cite{sybil-analysis}.
We simulated a network of $n$ nodes, each represented by an ID generated by SHA1 of a random IP/port combination.

The goal of the attacker was to mash as many pairs of adjacent nodes as possible.
We call this the \textit{Nearest Neighbor Eclipse} since the attacker seeks to become the nearest neighbors of each node.

The attacker was given $num\_ips$ randomly generated IP addresses, but could use any port between 49152 and 65535.
This means attacker had $ 16383 \cdot num\_ips $ Sybils at his disposal.
Each of these addresses could  be precomputed by the attacker and stored in a sorted list, requiring only 352 kilobytes per IP.

The adversary in this attack chooses any random hash key as a starting point to ``join'' the network.
This is their first Sybil and the join process provides information about a number of other nodes.
Most importantly, nodes provide information about other nodes that are close to it.
The adversary uses this information to inject Sybils in between successive healthy nodes.
For example, in Pastry, a joining node typically learns about the 16 nodes closest to it for fault tolerance, in addition to all the other nodes it learns about  \cite{pastry}.
In Chord, this number is a system configuration value $r$ \cite{chord}.

We simulated this attack on networks of up to 20 million nodes.
We chose 20 million since it falls neatly into the 15-27 million user range seen on Mainline DHT \cite{mainlineMeasure}.
We gave the attacker access to up to 19 IP addresses.
Our results are in Figures \ref{fig:exp2} and \ref{fig:size_prob_all}.

\begin{figure}
	\centering
	\includegraphics[width=0.5\linewidth]{figs/ip_prob_all}
	\caption[foo]{Our simulation results.  
		The $x$-axis corresponds to the number of IP addresses the adversary can bring to bear.
		The $y$-axis is the probability that any chosen region has been mashed.
		Each line maps to a different network size of $n$.
		The dashed line corresponds to values  $ P_{bad\_neighbor} =  \frac{num\_ips \cdot 16383}{num\_ips \cdot 16383 + n - 1}$, which is the probability a node has a malicious neighbor}
	\label{fig:exp2}
\end{figure}


\begin{figure}
	\centering
	\includegraphics[width=0.5\linewidth]{figs/size_prob_all}
	\caption[a]{These are the same as results shown in Figure \ref{fig:exp2}, but our $x$-axis is the network size $n$ in this case.  
		Here, each line corresponds to a different number of unique IP addresses the adversary has at their disposal.}
	\label{fig:size_prob_all}
\end{figure}


Our results show that an adversary, given only modest resources, can inject a Sybil in between the vast majority of successive nodes in moderately sized networks.
In a large network, modest resources still can be used to compromise more that a third of the network, an  important goal if the adversary  wishes to launch a Byzantine attack.

\subsection{Ramifications}
Our analysis and experiments show that an adversary with limited resources can easily compromise a P2P system and occlude the majority of the paths between nodes.
We can turn this attack around and use to benefit a DHT.
Some nodes will be responsible for larger regions than others and therefore will be responsible for a larger portion of the data.
If a node can detect when a peer is overloaded, the node can inject a virtual node into the region to shoulder some of the load.
The load could be defined by the size of the region or by the volume of traffic.

A network implementing this load-balancing strategy would be self-adaptive.
Nodes in this type of self-adaptive network would have a limited number of virtual nodes to mash.
This limit would protect nodes from becoming overloaded themselves and ensure network stability.
We discuss this further in Chapter \ref{chapter:experiments}.

\section{Summary}
Our previous work has focused primarily on various aspects of Distribute Hash Tables.
We can categorize our research into three distinct, but connected parts:
\begin{description}
	\item[Generalizing DHTs] We developed VHash, which uses the relation between Voronoi tessellation and DHTs to create a more abstract representation of how a DHT operates.
	Using this abstraction, we can embed certain properties into the DHT's topology and optimize these qualities.
	\item[Distributed Computing on a DHT] ChordReduce demonstrated how we can use a DHT to perform MapReduce in a completely decentralized and fault-tolerant environment.
	\item[Autonomous Load Balancing] We have shown that churn had a paradoxically beneficial effect on distributed computations.
	We postulated that we can use this effect to create a more intelligent mechanism for performing autonomous load-balancing, in part based off  the same techniques used to perform a Sybil attack.
	 
\end{description}
It is each of these three parts we wish to further research.
%TODO: godfrey2005heterogeneity
\chapter{Proposed Work}
\label{chapter:experiments}
In this chapter, we will present the three parts we propose for my research.
Each part can be completed independently of each other, but are related as to be much more beneficial to complete together.

\section{UrDHT: Our DHT Framework}
As mentioned in Chapter \ref{chapter:intro}, our plan is to create a highly configurable and easy to use DHT framework based off the DHT abstractions we have discovered.
Rather than making a fully-functional DHT application on our own, we will be making a minimally functional DHT framework that will be easy to fork for a variety of applications.

We call this proposed framework UrDHT, \textit{ur-} being the Germanic prefix denoting primal, or primitive, or original.
UrDHT is a project that presents a minimal and extensible implementation for all the essential components for a DHT: the different aspects for a protocol definition, the storage of values, and the networking components.
Every DHT has the same components, but there has yet to be an all-encompassing framework that clearly demonstrates this.

\begin{itemize}
	\item This will be done jointly with Brendan and anyone else who is interested in creating a completely open source framework for DHTs.
	\item In particular, my focus on the project will be implementing each of the DHTs I plan on using to test Distributed Computing on DHTs.
	This will require a formal description of each DHT and their components.
	\item The goal of this step is \textbf{not only} to create a DHT, but to create an easily extensible abstract framework for DHTs.
	\item The abstraction comes from implementing the relationship we found between DHT spheres of responsibility and Voronoi tessellations.  
	This is the key point of the project.
	Our previous research \cite{vhash} has led us to assert that there is a mathematical formulation for different aspects which every DHT shares in common, such as a distance metric and closeness definition.
	
\end{itemize}


There are two clear goals of this project: the mathematical definitions for distributed hash tables and the UrDHT application itself.
The UrDHT project is probably the stronger goal, since it is a fully fledged and novel framework.
Furthermore, as an open-source application, we it will be useful to many developers since it will provide an orderly way to create new DHTs and DHT-based applications.

The mathematical formulations, on the other hand, serve as novel formulations and definitions of DHTs.
They provide new insight, but do not serve as a new application or framework.
However, the formulations can and should be presented as an atomic unit of research.


\section{Distributed DHT Computing}

The next step is to use the UrDHT framework to re-implement ChordReduce.
Our goal is a DHT based platform for solving embarrassingly parallel problems using DHTs.
The steps involved in this are listed below.
\begin{itemize}
	\item We will use UrDHT to implement a few of the more popular DHTs.
	\begin{itemize}
		\item We want to compare each of the DHTs to see if there is a difference between using one or another for distributed computing.
		\item Using UrDHT for all the implementations will minimize the non-protocol differences between each DHT, which will allow for as fair a comparison as possible.
		\item Additionally this will serve as an example of how to implement our framework. 
	\end{itemize}
	\item Implement a distributed computing mechanism on each of the implemented DHTs for solving computing tasks.
	\begin{itemize}
		\item The emphasis of our distributed computing application is robustness and fault-tolerance.
	\end{itemize}
	\item Test each framework using a variety of embarrassingly parallel problems, such as:
	\begin{itemize}
		\item Brute-force cryptanalysis.
		\item MapReduce problems.
		\item Monte-Carlo computations.
	\end{itemize}
\end{itemize}

Here there is one very clear goal.
This would be a reimplementation of our original ChordReduce paper with completely new problems and new results.



\section{Autonomous Load Balancing}
As discussed earlier, it would be highly beneficial for each node to take advantage of their processing power.
The most straightforward way of doing this in a DHT context is to have more powerful nodes acquire a larger stretch of the network space.
The goals here are straightforward:

\begin{itemize}
	\item Further establish that the ``strategy'' of randomly inducing churn improves computational speeds.
	If we successfully establish it exists,  we must determine if the process can be modeled.%stochastically modeled or otherwise predicted via theoretical analysis.
	\item Create a processor scoring mechanism for creating virtual nodes.  
	Essentially, we need to create a mechanism to estimate how many virtual nodes a particular piece of hardware can handle.
	%\begin{itemize}
	%	\item This score is a relative score, since the amount of work a node can carry is determined by the amount other nodes can carry.
	%	\item Furthermore, the score should also be a function of the node's range, since the amount of work is dictated by the amount of range ``owned'' the node. 
	%	This means creating the first virtual node does not double the amount of work the node has.
	%	\item This is compared to the scores of neighboring nodes  of all the node's replicas and should give a gauge of the node's relative  power.
	%\end{itemize}
	\item Use this score in conjunction with various implemented strategies for autonomous load-balancing to find the most effective.
	A few of the strategies we will examine:
	\begin{itemize}
		\item Passive load balancing -  Here, each node looks at range of its Sybil locations.
		Based on its score, it choses the $k$ largest locations and injects Sybils.
		No actual communication with other nodes is used.
		\item IRM \cite{irm} based  strategy.
		\item Invitation based -  Here we flip the strategy around.
		If the node detects its region is too large or it has too much work; it invites other nodes to help.
		The nodes look at the range in question and offer to help if they have a Sybil that fits and are not overloaded themselves.
		The inviter looks at the offers of help and selects the best candidate.
		\item Another invitation based scheme is that each of the nodes submits itself as a potential filler to each of its Sybil locations.
		The nodes that would be affected by this node's entry add it to a list of fillers. 
		When the node decides it needs help, it selects one of the fillers to join.
		These invitation based schemes have the advantage of nodes having control of their range. 
		The question is it ``turtles all the way down'' here?
		Do we let the replicas also call for help?
		\item Using the force spring model used in VHash \cite{vhash}.
	\end{itemize}
	%\item This may necessitate new node types, such as tracker nodes which exist only to eavesdrop on traffic loads.
	%\item We also plan on modifying virtual nodes so they do not take on additional routing c.  
	%We want to see how this change will impact the system.
	\item We must determine under which contexts this kind of highly responsive load-balancing can be used. 
	Is it only useful during distributed computation, or is it useful in file-sharing as well?
	
	
\end{itemize}

Once we have created the various load balancing strategies, we would need to implement them and test them against each other.
Randomly induced churn would serve as a baseline strategy.
We can then present which strategy is most effective at evenly spreading the network's load.


\chapter{Conclusion}
\label{chapter:conclusion}
Distributed Hash Tables (DHTs) are extremely powerful frameworks for distributed applications that are based off the simple and powerful hash tables.
Because DHTs were designed with P2P applications in mind, DHTs are scalable, fault-tolerant, and load-balancing.
These are exactly the qualities needed in a distributed computing framework.

We created a new DHT called VHash, which was based off the relationship we discovered between the DHTs and Voronoi tessellation.
VHash is a DHT that operates over a multidimensional space, which allows the embedding of arbitrary metrics into this space.
We showed that we can optimize latency in VHash to obtain faster lookup speeds than a traditional DHT, such as Chord.
The key to VHash is our Distributed Greedy Voronoi Heuristic (DGVH).
DGVH is a sufficiently accurate and fast approximation of Delaunay triangulations.
Aside from its application in VHash, DGVH's applications extend to other areas, such as wireless sensor networks.

We have also shown in the previous chapters that we were able to create a prototype distributed computing application \cite{chordreduce} based on the Chord DHT \cite{chord}.
ChordReduce, as we named it, demonstrated how MapReduce could be performed on the Chord distributed hash table.
As a DHT, ChordReduce is completely decentralized and fault-tolerant, able to handle  nodes entering and leaving the network during churn.
We demonstrated that ChordReduce can efficiently distribute a problem whilst undergoing significant churn and achieve a significant speedup.

During our experiments with ChordReduce, we found an anomaly in which a computation undergoing a significantly high level of churn finished twice as fast than when no churn was involved.
This implied to us that there the churn was effectively shuffling around the nodes such that nodes with no work were taking work from nodes with large amounts of work.
We want to use this implication  develop a more intelligent autonomous load-balancing mechanism.
Such a mechanism would allow underworked nodes to steal work from overworked nodes in the network.

Part of autonomous load-balancing will involve exploiting heterogeneity in the network.
We can do this by having more powerful nodes take a proportionally higher amount of work.
This involves a process we dubbed \textit{mashing}, which we originally used to analyze the Sybil attack on DHTs \cite{sybil-analysis}.


Based on the work we have completed, we proposed creating a framework, called UrDHT, and use it to create a distributed computing. 
As the name implies, UrDHT is meant to be the prototypical DHT, from which we can derive all other DHTs.
This framework would make it easy for developers to create not only new DHTs, but new distributed and P2P applications.
The application that we plan on creating with UrDHT is a Distributed Computing framework based on ChordReduce.

Our new framework will be able to handle more than just MapReduce problems and incorporate an autonomous load-balancing mechanism,
Developers could use our framework to effortlessly organize a disparate set of nodes into a functional distributed computing system and run their own applications.
Our framework could be used in numerous contexts, be it P2P or a data center.
%
\chapter{Previous Writing to incorporate above}





\begin{itemize}
	\item Build off of pond, make a \textit{completely} decentralized means of network programing (chordReduce)
	\item Create a method of automatic load balancing via
	\begin{itemize}
		\item inducing churn (Me: ChordReduce)
		\item injecting nodes (Me: Sybil Analysis)
		\item Replica creation (not me: IRM)
	\end{itemize}
	\item Embedding metrics into the DHT (VHash)
    \begin{itemize}
        \item Geographic location (CAN)
        \item latency 
    \end{itemize}
\end{itemize}



\section{What I need to cover}

\begin{itemize}
	\item What is my problem
	\item Why is it interesting
\end{itemize}


\section{things I have discovered and want to further explore}


\subsection{ChordReduce}
I found that churn can help out a network in ChordReduce.
It works like this-
In ChordReduce, we hypothesize this was due to the number of data pieces (larger) vs the number of workers (smaller).
There were more workers than there were pieces of data, so some workers ended up with more data than others in the initial configuration.
This means that there was some imbalance in the way data was distributed among nodes.
This was \textit{further} exacerbated by small number of workers being assigned locations with a hash function.
This leads to some nodes having larger swaths of responsibility than others.

Given this setup, without any churn, the operation would be:
Workers get triggered, they start working, and the ones with little work finish their work quickly, and the network waits for the node with a bunch of work.

It's important to note here that the work in ChordReduce was performed atomically, a piece at a time.
When a node was working on a piece, it informed it's successor, then informed them when it finished.
These pieces of work were also small, possibly too small.

Under our induced churn, we had the nodes randomly fail and  immediately join under a new ip.
The failure rates were orders of magnitude higher than what would be expected in a ``real'' (nonexperimental) environment.
The following possibilities could occur:
\begin{itemize}
	\item An node without any active jobs leaves.
	It dies and and comes back with a new port chosen.
	This new ID has a higher chance of landing in a larger region of responsibility.
	In other words, it has a (relatively) higher chance of moving into an space where it becomes acquires responsibilty for enqueued jobs.
	The outcomes of this are:
	\begin{itemize}
		\item The node rejoins in a region are doesn't acquire any new jobs.
		This has no impact on the network (Case I).
		\item The node rejoins in a region that has a jobs waiting to be done.
		It acquires some of these jobs.
		This speeds up performance (Case II).
	\end{itemize}
	\item A node with active jobs dies.
	It rejoins in a new space.
	The jobs were small, so not too much time is lost on the active job, and the enqueued jobs are backed up and the successor knows to complete them.
	However, the node can rejoin in a more job-heavy region and acquire new jobs.
	The outcomes of this are:
	\begin{itemize}
		\item A minor negative impact on runtime and load balancing (since the successor has more jobs to deal with) (Case III).
		\item A possible counterbalance in load balancing by acquiring new jobs off a busy node (Case Doesn't matter).
	\end{itemize}
\end{itemize}

Now here's the trick.
The longer the nodes work on the jobs, the more nodes finish and have no jobs.
This means as time increases, so do the occurrences that Case I and II occur, with a bit more weight on Case II.


What have we learned:
\begin{itemize}
	\item Shooting nodes as a example to motivate other nodes works or
	\item Even the smallest effort of trying to dynamically load  balance, such as rebooting random nodes to new locations, has benefits for runtime.
	Our method is a horrible approximation of dynamic load-balancing, and it still shows improvement.
\end{itemize}



We still have to verify the existence of this phenomena in an independent experiment.
Besides that, we still have the following questions:
Can I stochastically model it?
Does it work for other problems?


This ties into the next bit.
\subsection{Sybil}
We discovered injection is easy and simple in P2P networks.
I hypothesize we can Sybil attacks for improving load balancing on demand.
Perhaps we need an entirely new node type.

\subsection{VHash}
Arguablely all DHT's are Voronoi tessilations or can be mapped to.


Hops is an estimate for time, not necessarily a good one!
We can embed latency in the DHT.
We can try embedding geographic location in as well using




\subsection{Pond}
P2P can be used for decentralized volenteer computing.





\section{In WSNs}
http://users.cis.fiu.edu/~carbunar/boundary.pdf


\section{DHTs as a volunteer Platform}
Rather than rely on a centralized administrative source,


Decentralized resource discovery.
The system is organized using a P2P system built on Brunet \cite{brunet}.





\subsection{PonD}

Middleware platforms like BOINC \cite{anderson2004boinc} HTCondor\footnote{Formerly Condor; the name was changed after a copyright dispute.} \cite{thain2005distributed} provide a way to push distributed computing tasks to idle CPUs.
These platforms effectively let users donate their unused processing power to some scientific collaberative effort.

While the work is distributed, the organizational backbone  is centralized.
Scalability is the primary issue here.


On-demand computing-as-a-resource is a thing and it needs to be scalable and fault tolerant.
P2P paradigms do that.

PonD \cite{lee2012pond} does decentralized resource discovery and centralized job management/centralized scheduler.
``Decouples resource discovery from job execution/monitoring.''
Scheduling time is $ O(\log N) $, for $N$ resouces in the pool.

\subsubsection{Improvements}
Why not decentralized or partially decentralized job management?


\section{Why DHTs for distributed computing?}

\cite{malkhi2001viceroy} -  Between congestion, cost of join/leaves, and lookup time there are tradeoffs.  
Optimizing for two can be done but has bad cost.
For example, a balanced binary tree has congestion at root.


\section{MapReduce}

%copied from CHRONUS
Google's MapReduce \cite{mapreduce} paradigm has rapidly become an integral part in the world of data processing and is capable of efficiently executing numerous Big Data programming and data-reduction tasks.  
By using MapReduce, a user can take a large problem, split it into small, equivalent tasks and send those tasks to other processors for computation.  
The results are sent back to the user and combined into one answer.  
MapReduce has proven to be an extremely powerful and versatile tool, providing the framework for using distributed computing to solve a wide variety of problems, such as distributed sorting and creating an inverted index \cite{mapreduce}. 

At it's core, MapReduce \cite{mapreduce} is a system for process key/value pairs, a that statement that equally describes DHTs.
However, MapReduce operates over a different set of assumptions \cite{hadoopAssumptions} than DHTs.
MapReduce platforms are highly centralized and tend to have single points of failure\cite{shvachko2010hadoop} as a result.   
A centralized design assumes that the network is relatively unchanging and does not usually have mechanisms to handle node failure during execution or, conversely, cannot speed up the execution of a job by adding additional workers on the fly.
Finally deploying these systems and developing programs for them has an extremely steep learning curve.

If we make MapReduce operate under the same assumptions as a DHT, we have effectively further abstracted the MapReduce paradigm and created a system that can operate both in a traditional large datacenter or as part of a P2P network.
The system would be highly resistant to failures at any point, scalable, and automatically load-balance. 
The administrator can add any number of heterogeneous nodes to the system to get it operate.

\subsection{Current MapReduce DHT/P2P combos}
There have been a few implementations combining MapReduce with a P2P framework, in varying capacities.  
I will present two here, as well as my own implementation, ChordReduce.

\subsubsection{P2P-MapReduce}
Marozzo et al. \cite{marozzo2012p2p} investigated the issue of fault tolerance in centralized MapReduce architectures such as Hadoop.  
They focused on creating a new P2P based MapReduce architecture built on JXTA called P2P-MapReduce.  
P2P-MapReduce is designed to be more robust at handling node and job failures during execution.

Rather than use a single master node, P2P-MapReduce employs multiple master nodes, each responsible for some job.  
If one of those master nodes fails, another will be ready as a backup to take its place and manage the slave nodes assigned to that job.  
This avoids the single point of failure that Hadoop is vulnerable to. Failures of the slave nodes are handled by the master node responsible for it.

Experimental results were gathered via simulation and compared P2P-MapReduce to a centralized framework. 
Their results showed that while P2P-MapReduce generated an order of magnitude more messages than a centralized approach, the difference rapidly began to shrink at higher rates of churn.  
When looking at actual amounts of data being passed around the network, the bandwidth required by the centralized approach greatly increased as a function of churn, while the distributed approach again remained relatively static in terms of increased bandwidth usage. 
They concluded that P2P-MapReduce would, in general, use more network resources than a centralized approach. 
However, this was an acceptable cost as the P2P-MapReduce would lose less time from node and job failures \cite{marozzo2012p2p}.

\subsubsection{Parralel Processing Framework on a P2P System}
Lee et al.'s work \cite{leemap} draws attention to the fact that a P2P network can be much more than a way to distribute files and demonstrates how to accomplish different tasks using Map and Reduce functions over a P2P network.  
Rather than using Chord, Lee et al. used Symphony \cite{symphony}, another DHT protocol with a ring topology.  
To run a MapReduce job over the Symphony ring, a node is selected by the user to effectively act as the master.  
This ad-hoc master then performs a bounded broadcast over a subsection the ring.  
Each node repeats this broadcast over a subsection of that subsection, resulting in a tree with the first node at the top.  

Map tasks are disseminated evenly throughout the tree and their results are reduced on the way back up to the ad-hoc master node.  
This allows the ring to disseminate Map and Reduce tasks without the need for a coordinator responsible for distributing these tasks and keeping track of them, unlike Hadoop.  
Their experimental results showed that the latency experienced by a centralized configuration is similar to the latency experienced in a completely distributed framework.





\subsubsection{ChordReduce}
ChordReduce is designed as a more abstract framework for MapReduce, able to run on any arbitrary distributed configuration.
ChordReduce leverages the features of distributed hash tables to handle distributed file storage, fault tolerance, and lookup.  
ChordReduce  was designed to ensure that no single node is a point of failure and that there is no need for any node to coordinate the efforts of other nodes during processing.  



\subsection{Experiment Description: Comparison of MapReduce paradigm on different DHTs}
In order to test MapReduce over a DHT, I will do the following:
\begin{itemize}
	\item Implement CAN \cite{can}, Pastry \cite{pastry}, Chord \cite{chord}, Kademlia \cite{kademlia}, VHash, and ZHT \cite{li2013zht} /similar
	\begin{itemize}	
		\item This covers different geometries with different base parameters.
		\item This also necessitates the creation of an extensible DHT framework.
		\item  The DHT should be extended with more powerful search functionality (see distributed database below), and built-in policies for virtual nodes.
		
	\end{itemize}
	\item Compare results with each other and a traditional MapReduce platform, such as Hadoop.
	\item Certain DHTs may be better suited to different problem formulation
	
\end{itemize}



%\section{High End Computing}
%PonD?
%\subsection{Metadata Management}
%\subsection{Robustness}

%\subsection{Experiment Description:}

%\section{Graph Processing on a DHT} 
%Lookup Graphlab
%\subsection{Embedding}

%\subsection{Experiment Description:}
%\subsection{Distribute the work for solving a graph on a DHT}
%\subsection{Comparison to well established or state of the art methods}



%\section{Machine Learning Problems on A DHT}


%\subsubsection{Bayesian Learning}
%%\subsection{Experiment Description:}
%Take MapReduce machine learning algorithm

\section{Distributed ``Databases'' and Complex Queries}


DHTs run off the assumption that you know what you want to find.
I'm looking for node responsible for this key, becuase I need the associated value.

Complex queries and databases are used to find stuff based on attributes, and since you're asking, you don't know what (primary) keys you need.


Want to find all files that match the criteria?





Simple: Find all files with ``author = John Smith''.  Idiot solution, assign ``author = John Smith'' a hash key,  it's value is a file with all the files with the (that doesn't scale) 


Complex: Processing database queries.   Find all files with age < 20 and niceness >12

One proposed soultion is to use montonic (order-preserving) hash functions \cite{triantafillou2004towards}, but monotonic functions don't have randomness and uniformity.


Could embed as dimension with VHash!


See lee pond for their multi-tree broadcast which they got from other papers \cite{lee2012pond}



Koorde?HpyerChord
Each node has it's own subset chord ring

Query to a space, then within the space, deluge and spam the query which gives a reduce path
\subsection{Paper summaries}

P2P systems have two limitations: scaling and complex queries \cite{harren2002complex}.  
I would also say robustness.
DHTs have solved the issue of sclaing for the most part but have a significantly limitted querying capability:  only native-exact matches are supported.  

A database is to be avoided since most users will not want to manage a database.  
The authors wish to focus purely on the abstract relational algebra operations, since the objective is only query processing.

The authors proposed extending the DHT API.
The data storage layer needs a way to iterate thru the data objects it has stored.
Each object needs a unique local identifier and the data itself, as well as a way of accessing stored metadata for each object. 


A big issue is supporting the range queries, since exact matches can be solved in many different ways.
One proposed soultion is to use montonic (order-preserving) hash functions \cite{triantafillou2004towards}, but monotonic functions don't have randomness and uniformity.

Triantafillou et al.\ propose keying the ID of file off the primary key of a relation and using an order-preserving hash function for each of the other attributes.\footnote{This technique was designed only for \texttt{ints}}
For each attribute $DA_i$, the entire keyspace is partitioned into $s_i = \frac{2^{m}}{|DA_{i}|}$ and the monotonic hash function $h_i$ for $DA_i$ is defined as  $h_{i}(a_{i})=\lceil (a_{i} - \min(DA_i)) \times s_i \rceil$ for each individual attribute $a_i$  (may have botched terminology).


When a tuple with key $t$ is \texttt{put} on the network, it's stored at \texttt{successor(t)}.  It is then replicated at each peer responsible for each attribute, which is found using the order-preserving hash functions.


\subsubsection*{PHT}
%Abstract level summary.  What can it do, what can't it do.


\section{Semiautomagic Load Balancing}
What: Automagic load balancing.  One of two possiblilities:  inject new node into region or create new virtual node in region. 
Requires Where, when, and how/which

\begin{itemize}
    \item Where: Can be answered with Sybil selection.
    \item When:  Can be answered with IRM for hot spots.  Can be answered with neighbor monitoring
    \item How:  The remaining peace
    \item Symphony demonstrates how to estimate $n$
\end{itemize}

2/28: joiner asks parent holds the most, who asks his log n nodes who holds the most


Need equation for optimal number of replicas in the network.

\section{Resources}
\subsection{Planetlab}
\subsection{Local Cluster}


\bibliography{notes,dht,mapreduce,voronoi,dns,botnets,mine}
\bibliographystyle{plain}
\end{document}





