\documentclass[]{article}

%opening
\title{Summary of Accomplishments}
\author{Andrew Rosen\\Assistant Professor of Instruction}

\begin{document}

\maketitle


\section{Preface}
This document serves as a overview of my accomplishments thus far at Temple University.  
I came to Temple in the Fall of 2016.
I was hired as an Assistant Professor of Instruction because of my record of teaching excellence and remarkable amount of experience, having taught numerous courses solo as a PhD student.
Despite the relatively short time-frame between my hiring and now, I believe I have proven myself to be an exceptional teacher.

I teach the CIS 1051 and CIS 1068, which serve as introductions to computer science.
I also teach CIS 2168,  which serves as prerequisite to almost all upper level classes in computer science and is one of the courses large tech companies look at when evaluating and interviewing students (interview problems often draw from this class).  
I also have served as the course coordinator for CIS 2168 since the fall of 2018, writing the common final and coordinating the material to be taught between sections.
These classes expose me to a great variety of students and can be quite challenging to accommodate the skill disparity often seen in Computer Science courses.


What distinguishes me as a Professor is dedication constant improvement of my courses and making as great of an experience for my students as possible.
I am always willing to go that extra mile for my students.





I meet with my students whenever they need help; I've met with numerous students late at night via video meetings


I am awesome because
\begin{itemize}
	\item I came in with a relatively huge amount of experience.  Fully focused on teaching
	\item Student frickin love me
	\item My SFFs say so
	\item JAPAN
	\item ONline course
	\item Online mentoring
	\item 
\end{itemize}

Therefore gimme promotion

\section{Teaching Portfolio}


\subsection{Student SFF Analysis}


4.9 and 5.0.  
Try finding someone better in the college.

Obviously the SFFs were good enough to get an award.

\subsection{Student SFF Excerpts}

My comments left by students have been overwhelmingly positive and can be found in the student feedback forms.
I've included some choice examples (misspellings and grammar are unaltered):

\subsubsection{What aspects of the course or the instructor’s approach contributed most to your learning?}

\begin{small}
	
	\begin{quotation}
		Flipped lectures are a fabulous idea for a programming class. A student can spend their own time out of class reviewing the material necessary for the upcoming assignment and then ask any and all questions to the professor in class. This avoids the constant struggle in a programming class of not being able to find the error in your code the night before the lab is due.
	\end{quotation}

\begin{quotation}
	Good sense of humor, more enthusiasm than youd expect from a married guy with a baby in the morning, overall energetic personality that gave life to the class. Also straightforward lectures that were recorded to make it easy for anyone who missed class or needed a reminder on certain things.
\end{quotation}

\begin{quote}
	 I loved the flipped classroom style he used. Quizzes were fair, if on the easy side. Overall I felt like material covered was pretty simple, and could have been more aggressive in difficulty, but Rosen particularly did a good job of clarifying common tripping points, and made effort to individualize his approach per student. Clearly he has a desire to teach, and isnt just a professor by mandate. Incredible man
	
\end{quote}

\begin{quotation}
	
	Rosen was an awesome professor and is clearly very passionate about data structures. Very helpful when questions were asked. Also speaks volumes about him that he had a newborn kid during the earlier weeks of class, and was still very present!
	
\end{quotation}
	
	
\begin{quotation}
	Professor Rosen has a unique ability to make a two and a half hour class not drag by. His insights into the way data structures fundamentally work is what really makes him an excellent professor.
	
\end{quotation}


\begin{quotation}
	Rosen is the man. Explains clearly and is PATIENT with students as they learn. A lot of professors struggle with this. Encouraging and nice guy.
\end{quotation}

\begin{quotation}
	Dear CIS department, GET MORE PROFESSORS LIKE ANDREW ROSEN. I actually felt encouraged in the class to learn because he laid the material out as simply as he could. [\textellipsis]\footnote{I removed a remark about another Professor.} I did not feel like I had to come into class with a whole lot of knowledge in order to succeed. He gave examples, hints and spent time explaining each topic to us.  Also his pratice tests, reviews of the pratice test, use of slides and recordings of the class were extremely helpful. I honestly appreciate this teacher.
	
\end{quotation}

\begin{quotation}
	It was very clear hes thought a lot about how to teach this course. This was clear in the ways he sometimes defied teaching convention in thoughtful ways. For example, the requirement for students to demonstrate their code to him or the TA, his insistence that we learn how to use libraries on our own, and any other numerous example (some of which were very subtle.) Thats not to mention the fact that his wife had a baby the second day of the course, and yet he was always there for lab hours. Dr. Andrew Rosen deeply impressed me. He may very well be the most skilled teacher Ive ever had.
	
\end{quotation}

\begin{quotation}
	Professor Rosen is the best teacher Ive had at Temple, and its not even close. He really enjoyed the teaching and I think the enthusiasm showed, without being over the top and annoying. He is the most organized teacher. We knew exactly where we were on relation to the totality of the class. He answered every question, and I mean every question, without ever getting annoyed.he recorded the class, which I still why every doesnt do it. Even the little things, he always made sure he was in view of the camera when he wrote on the board. He writes his own test questions, or at the very least selected questions that sounded like he wouldve wrote them, which during the stress of a test im already used to the meaning which rereading 50 times. He was extremely fair but not a pushover. His tests werent easy but I felt very prepared. He is an amazing and I hope Im able taught by him again.
	
\end{quotation}

\begin{quotation}
	The homeworks were often difficult in a good way. On an unrelated side note, Dr. Rosen is hilarious and I always enjoyed going to Class.
\end{quotation}

\end{small}

Flipped lectures are a fabulous idea for a programming class. A student can spend their own time out of class reviewing the material necessary for the upcoming assignment and then ask any and all questions to the professor in class. This avoids the constant struggle in a programming class of not being able to find the error in your code the night before the lab is due.


\section{CAT and Continuing Education}


I attended the
I attended the Flipped Learning workshop and received guidance from cat on my coursre
\section{Curriculum Development}
Course Coordinator for 2168


\subsection{Recordings}

\subsection{Flipped Course}

\subsection{Japan}

\subsection{Online Graduate Course}

\subsection{Online Course Mentoring}

\section{Devotion to Students}


\subsection{Accommodations}


\section{Awards?}
\end{document}
