\documentclass[]{article}
\usepackage{hyperref}
%opening
\title{Summary of Accomplishments}
\author{Andrew Rosen\\Assistant Professor of Instruction}

\begin{document}

\maketitle


\section{Preface}
This document serves as a overview of my accomplishments thus far at Temple University.  
I came to Temple in the Fall of 2016.
I was hired as an Assistant Professor of Instruction because of my record of teaching excellence and remarkable amount of experience, having taught numerous courses solo as a PhD student.
Despite the relatively short time-frame between my hiring and now, I believe I have proven myself to be an exceptional teacher.

I teach CIS 1051 and  have taught CIS 1068, both of which serve as introductory courses to computer science.
I also teach CIS 2168,  which serves as prerequisite to almost all upper level classes in computer science and is one of the courses large tech companies look at when evaluating and interviewing students (interview problems often draw from this class).  
I also have served as the course coordinator for CIS 2168 since the fall of 2018, writing the common final and coordinating the material to be taught between sections.
%These classes expose me to a great variety of students and can be quite challenging to accommodate the skill disparity often seen in Computer Science courses.

What distinguishes me as a Professor is dedication constant improvement of my courses and making as great of an experience for my students as possible.
I am always willing to go that extra mile for my students and to improve our department's curriculum.



My accomplishments and reasons for promotion are detailed  in this document, but briefly, they are:

\begin{itemize}
	\item I won the 2018 Dean’s Distinguished Teaching award for my excellence in teaching.
	\item I became the first faculty member to teach Computer Science courses at Temple University Japan and setup the courses to be run on campus locally for the first  time.  I gave the faculty there the tools to continue to provide Computer Science courses locally at TUJ, with our eventual goal is to offer a Computer Science minor that can be completed solely at the Japan campus.
	
	\item I find and develop assignments that are engaging to our students, build a portfolio for students to show prospective employers, and develop the skills they need to succeed in the industry.
	
	\item I have become the first faculty member in my department to ``flip'' a classroom, which has students watch lectures before coming to class, where they work on assignments and other active learning exercises.
	\item My student feedback forms show repeatedly high evaluations from students, scoring a 4.9 and 5.0 in my most recent semesters in ``I learned a great in this course'' and ``The instructor taught this course well.''  My students continually report that the assignments I develop, the techniques I use, the novel approaches I use to teach my course, the care I show and  accommodations make for them overwhelmingly to better outcomes. 

	 
	\item I have  developed a new course for graduate students in our Masters in IST program, which is designed to transition a student with very little programming experience to a student who is ready for graduate level coursework.
	\item I am one of the first two faculty members in our department to develop and teach an online course.
	\item I have been working with the Office of Digital Education to create high quality videos for our online courses that can also be fed back into our undergraduate courses.  An example video can be found here: \url{https://player.vimeo.com/video/313364524}
	\item I have attended a number of workshops to further my teaching skills and to learn how to improve my course.
	\item I am mentoring other faculty members on how to develop online courses and create effective recorded lectures for programming courses.
\end{itemize}

%Computing is a field that is in high demand right now, and students who successfully complete their degree are able to command extremely high salaries

These all demonstrate that I am a highly skilled and effective teacher who is always looking for a way to improve upon what we can offer our students. I am always willing to step forward and be the first to try something new or to build something we are lacking.

However, what I think makes me a great teacher more than any of the above is that I find it \textit{fun}.  
Teaching people how to think critically and solve puzzles is \textit{fun} and exciting and feels great.
It's hard for me to contain my excitement when I go  over some problems and that enthusiasm infects the students.
It's thrilling to watch them see that the problem that looks insurmountable is just a bunch of small, manageable problems that they've just been given the tools to solve,  and it suddenly becomes, for want of a better word, surmountable.


The students see this and respond to it.
It creates a relationship where we're more than just cogs on opposite ends of a giant education machine.
This creates a positive and exciting atmosphere for the classroom, where students feel comfortable collaborating and asking for help, either from their peers or from me.


Yes, computing is hard and programming is difficult to learn.
It involves rigorous mathematics, brain-bending logic, and many late nights debugging that one line that just won't work.
By their sophomore year, most programmers develop some level of impostor syndrome that will probably  stick around for the better part of a decade. 
I see no reason why it shouldn't be fun too.


\section{Teaching Portfolio}
Included in my documents are my class contents for all the classes I teach and my teaching philosophy.




\subsection{Student SFF Analysis}
My student feedback forms show a consistently high level of student satisfaction in my courses.
I am not only continually rated ``U'' in student assesments, but I repeatedly score 4.8 or above in what I consider to be the most  

My most recent semester had a perfect 5.0


4.9 and 5.0.  
Try finding someone better in the college.

Obviously the SFFs were good enough to get an award.

\subsection{Student SFF Excerpts}

My comments left by students have been overwhelmingly positive and can be found in the student feedback forms.
I've included some choice examples (misspellings and grammar are unaltered):

\subsubsection{What aspects of the course or the instructor’s approach contributed most to your learning?}

\begin{small}
	
	\begin{quotation}
		Flipped lectures are a fabulous idea for a programming class. A student can spend their own time out of class reviewing the material necessary for the upcoming assignment and then ask any and all questions to the professor in class. This avoids the constant struggle in a programming class of not being able to find the error in your code the night before the lab is due.
	\end{quotation}

\begin{quotation}
	Good sense of humor, more enthusiasm than youd expect from a married guy with a baby in the morning, overall energetic personality that gave life to the class. Also straightforward lectures that were recorded to make it easy for anyone who missed class or needed a reminder on certain things.
\end{quotation}

\begin{quote}
	 I loved the flipped classroom style he used. Quizzes were fair, if on the easy side. Overall I felt like material covered was pretty simple, and could have been more aggressive in difficulty, but Rosen particularly did a good job of clarifying common tripping points, and made effort to individualize his approach per student. Clearly he has a desire to teach, and isnt just a professor by mandate. Incredible man
	
\end{quote}

\begin{quotation}
	
	Rosen was an awesome professor and is clearly very passionate about data structures. Very helpful when questions were asked. Also speaks volumes about him that he had a newborn kid during the earlier weeks of class, and was still very present!
	
\end{quotation}
	
	
\begin{quotation}
	Professor Rosen has a unique ability to make a two and a half hour class not drag by. His insights into the way data structures fundamentally work is what really makes him an excellent professor.
	
\end{quotation}


\begin{quotation}
	Rosen is the man. Explains clearly and is PATIENT with students as they learn. A lot of professors struggle with this. Encouraging and nice guy.
\end{quotation}

\begin{quotation}
	Dear CIS department, GET MORE PROFESSORS LIKE ANDREW ROSEN. I actually felt encouraged in the class to learn because he laid the material out as simply as he could. [\textellipsis]\footnote{I removed a remark about another Professor.} I did not feel like I had to come into class with a whole lot of knowledge in order to succeed. He gave examples, hints and spent time explaining each topic to us.  Also his pratice tests, reviews of the pratice test, use of slides and recordings of the class were extremely helpful. I honestly appreciate this teacher.
	
\end{quotation}

\begin{quotation}
	It was very clear hes thought a lot about how to teach this course. This was clear in the ways he sometimes defied teaching convention in thoughtful ways. For example, the requirement for students to demonstrate their code to him or the TA, his insistence that we learn how to use libraries on our own, and any other numerous example (some of which were very subtle.) Thats not to mention the fact that his wife had a baby the second day of the course, and yet he was always there for lab hours. Dr. Andrew Rosen deeply impressed me. He may very well be the most skilled teacher Ive ever had.
	
\end{quotation}

\begin{quotation}
	Professor Rosen is the best teacher Ive had at Temple, and its not even close. He really enjoyed the teaching and I think the enthusiasm showed, without being over the top and annoying. He is the most organized teacher. We knew exactly where we were on relation to the totality of the class. He answered every question, and I mean every question, without ever getting annoyed.he recorded the class, which I still why every doesnt do it. Even the little things, he always made sure he was in view of the camera when he wrote on the board. He writes his own test questions, or at the very least selected questions that sounded like he wouldve wrote them, which during the stress of a test im already used to the meaning which rereading 50 times. He was extremely fair but not a pushover. His tests werent easy but I felt very prepared. He is an amazing and I hope Im able taught by him again.
	
\end{quotation}

\begin{quotation}
	The homeworks were often difficult in a good way. On an unrelated side note, Dr. Rosen is hilarious and I always enjoyed going to Class.
\end{quotation}

\end{small}


Whilst I am skeptical of student's attribution of being super organized (a quality I think many of those reading this document would be horrified to have applied to them as well), these quotations highlight many of the aspects that I try can cultivate in my classroom.  


I record my lectures so my students can peruse them at their convenience and review the content whenever they need it.  
\section{CAT and Continuing Education}


I attended the
I attended the Flipped Learning workshop and received guidance from cat on my coursre
I have attended SIGCSE
\section{Curriculum Development}
Course Coordinator for 2168



\subsection{Recordings}

I record becuase


\subsection{Code}

\subsubsection{Demoing}
I do know any professors who look forward to grading a giant pile of student work.

\subsection{Flipped Course}
The idea of a flipped classroom for programming classes seemed like a natural fit and  a clear evolution from recording my lectures.  
The \textit{only} way to learn programming is by actually getting your hands dirty and trying to solve a problem, making and correcting innumerable mistakes along to way.  
I often tell my class the only reason I can debug a problem  they have  in just a few sections is that I've made that same mistake about 50 times and they just made that mistake for the first time.

But that first time, as they say, is a doozy.  
I know from both personal experience  and student feedback that the first time you encounter some errors they can take hours to solve, if not longer.  
The mistake can  something utterly trivial, such as using an ``l'' where an ``I'' should be, or just forgetting to call the piece of code you just wrote.
These mistakes are human and you feel dumb and inadequate when you figure out the simple think that just ate up three hours.
It's not an enjoyable experience for programmers and its a common one.  It shouldn't have to be.

Flipping a course turns it on its head.  
Students watch prerecorded content before coming to class for the week, then they do active learning exercises in class and take quizzes.  The rest of the time in the lecture is made available for the students to do their homework assignments.  Here, they make same mistakes any beginner makes, but the difference is I am here to immediately fix it or guide them in the right direction before they get too off track.


But more importantly, it gives me the one on one time with students that is just so lacking in a traditional lecture.  
I get to meet with them and addres their individual needs.
Are they having trouble with a particular topic? I can spend a few minutes sitting down with the student and personally explainign it to them.  Are students nearby listening in?  I can address the whole class for a bt and give some good examples and pointers.  Does the student need a bigger challenge? I know ways to make most progjects more exciting for many students




\subsection{Japan}

\subsection{Online Graduate Course}

\subsection{Online Course Mentoring}

\section{Devotion to Students}


\subsection{Accommodations}


\section{Awards?}
\end{document}
