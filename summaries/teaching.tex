%\documentclass[11pt,aspectratio=169]{beamer}
\documentclass[11pt]{beamer}
\usetheme{Darmstadt}
%\usecolortheme{dolphin}

%\setbeameroption{show notes}
%\setbeameroption{show only notes}
%\usepackage{pgfpages}
%\pgfpagesuselayout{4 on 1}[a4paper]

\usepackage[utf8]{inputenc}
\usepackage[english]{babel}
\usepackage{amsmath}
\usepackage{amsfonts}
\usepackage{amssymb}
\usepackage{graphicx}
\usepackage{subcaption}
\usepackage{algorithmic}
\captionsetup{compatibility=false}

\author{Andrew Rosen}
\title[Teaching Experiences]{Computer Science Curriculum and Teaching}

%\logo{\includegraphics[height=1cm]{figs/logo}}
%\institute{Georgia State University}
\date{May 18th, 2016}
%\subject{}
%\setbeamercovered{transparent}
%\setbeamertemplate{navigation symbols}{}



\AtBeginSection[]
{
  \begin{frame}
    \frametitle{Table of Contents}
    \tiny{\tableofcontents[currentsection]}
  \end{frame}
}

\begin{document}
	
	
	
\maketitle


\section{Introduction}

\subsection{About Me}

\begin{frame}{Research}
	\begin{itemize}
		\item Fault Tolerant Systems
		\item Distributed Hash Tables
		\item  Non-traditional systems and problems 
		\begin{itemize}
			\item DHT Computing, 
			\item VANETS
			\item Delay Tolerant Networks
		\end{itemize}
	\end{itemize}
\end{frame}


\begin{frame}{Teaching}
	Taught:
	\begin{itemize}
		\item Principles of Computer Science
		\item System Level and C Programming (eg Unix and C)
		\item Data Structures
	\end{itemize}
	
	TA'd the above and:
	\begin{itemize}
		\item Networks
		\item Security
		\item Principles of Computer Science (with robots)
	\end{itemize}
\end{frame}


\begin{frame}{Personal Bias}
	\begin{itemize}
		\item I love Python
		\pause
		\item Draw on history and other disciplines whenever possible
		\pause 
		\begin{itemize}
			\item Creates real world examples
			\item Creates better stories
		\end{itemize}
		\pause
		\item I limit the use of slides
		\begin{itemize}
			\item I often create my own materials and lecture notes for students
			\item I live code when I can
		\end{itemize}
		\pause
		\item \LaTeX
	\end{itemize}
\end{frame}

\section{The Courses I Taught}

\subsection{Intro to CS}
\begin{frame}{Principles of Computer Science}
	\begin{itemize}
		\item First real programming class
		\item Arguably the hardest to teach
		\item Taught 100 class, twice 
		\item Four lab sections
		\item Uses Java
	\end{itemize}
\end{frame}



\begin{frame}{Curriculum}
	\begin{itemize}
		\item History and Definitions
		\item Syntax
		\item Methods
		\item Loops and Conditions
		\item OO Basics and Strings
		\item Binary
		\item CS Breadth
	\end{itemize}
\end{frame}


\subsection{Unix and C}

\begin{frame}{System Level Programming}
	\begin{itemize}
		\item Taught once, TA'd once
		\item 51 students
		\item Covers proficiency in Unix
		\item Intermediate C
		\item Huge amount of subject matter
	\end{itemize}
\end{frame}

\begin{frame}{Unix Content}
	\begin{itemize}
		\item Bash basics and commands
		\item Permissions
		\item Regex
		\item awk/sed/grep and other scripting tools
		\item Malfeasance
		\item A tad bit of Python
		
	\end{itemize}
	
\end{frame}

\begin{frame}{C Content}
	\begin{itemize}
		\item Differences between C and Java
		\item Pointers and pointer arithmetic
		\item Memory management
		\item Some brief exposure to compilers and interpreters
	\end{itemize}
	
\end{frame}

\subsection{Data Structures}

\begin{frame}{My Favorite Course: Data Structures}
	\begin{itemize}
		\item Taught Twice
		\item 25 students
		\item Best class for first time instructors due to :
			\begin{itemize}
				\item Class makeup and experience
				\item Modular content
			\end{itemize}
		\item Our course had an emphasis on documentation. 
	\end{itemize}
	
\end{frame}


\begin{frame}{My Favorite Course: Data Structures}
	\begin{itemize}
		\item Taught Twice
		\item 25 students
		\item Best class for first time instructors due to:
		\begin{itemize}
			\item Class makeup and experience
			\item Modular content
		\end{itemize}
		\item Our course had an emphasis on documentation. 
	\end{itemize}
	
\end{frame}


\begin{frame}{Content}
	\begin{itemize}
		\item Review and last bits of Java
		
		\begin{itemize}
			\item Sometimes grad students w/ different background
			\item Try/Catch
			\item File I/O
		\end{itemize}
		\pause
		\item Lists
		\begin{itemize}
			\item ArrayList
			\item Linked Lists
		\end{itemize}
		\pause
		\item Stacks and Queues
		\begin{itemize}
			\item Good time to introduce Java's Generics
		\end{itemize}
		\pause
		\item Trees
		\pause
		\item Graphs (if time remains)
	\end{itemize}
\end{frame}

\section{Challenges and Solutions}

\subsection{General}

\begin{frame}{The Skill Gap}
	\begin{itemize}
		\item Skill gap has many sources
		\item Inform experienced students
		\item Address skill gap; students feel less intimidated
		\item Target and gauge the middle row
	\end{itemize}
\end{frame}



\begin{frame}{Fear and Shyness}
	
	Students hate being wrong and are shy\footnote{Sweeping Generalization}.  How do we encourage interaction?
	\pause
	\begin{itemize}
		\item Focus on the middle row
		\pause
		
		\item I use a story about cold reading
		\begin{itemize}
			\item People remember hits more than misses.
		\end{itemize}
		\pause
		
		\item Make yourself available 
		\pause
		
		\item Foster relationships
		\pause
		
		\item Go out of your way to help
	\end{itemize}
	\pause
	These seem obvious, but they require repetition in class.
\end{frame}


\begin{frame}{Unfamiliarity}
	\begin{itemize}
		\item CS courses are different.
		\begin{itemize}
			\item Most similar to Math
			\item Expectations most similar to Music
		\end{itemize}
		
		\item Relate unfamiliar concepts to:
		\begin{itemize}
			\item The real world
			\item Other domains
			\item Previous knowledge
		\end{itemize}
		\item Build on what students know
		\item Revisit old material
		\item Make it cool
		
	\end{itemize}
\end{frame}

\begin{frame}{Example Problem}
	Recently, NASA demonstrated a laser communication system which was able to transmit data from the Moon to Earth over a link with a bandwidth of 622 megabits/second.  How long would it take an astronaut to send a 500 megabyte video from the Moon to Earth?
	
	\begin{itemize}
		\item The average distance from Earth to its moon is 384,400 kilometers
		
		\item Speed of light ~ 300,000,000 $ \frac{m}{s} $
		
	\end{itemize}
\end{frame}

\subsection{Course Specific}


\begin{frame}{Beware of the Pitfalls}
	Know where students begin to fall behind.
	\begin{itemize}
		\item Consistent model of assignment in Intro
		\item Pointers and arrays in C
		\item Everything is a file in Unix
		\item Linked Lists and pointers in Data Structures
	\end{itemize}
\end{frame}
\section{Changes}

\begin{frame}
\begin{quotation}
	``There are only two kinds of languages: the ones people complain about and the ones nobody uses.''
\end{quotation}
-- Bjarne Stroustrup, creator of C++
\end{frame}


\subsection{What I Would Change in Intro}
\begin{frame}{Why Do We Use Java}
	\begin{itemize}
		\item Universal
		\item A whole lot like C and C++ (and anything based off them)
		\item Good teaching resources
		\item References a good enough starting point for pointers
	\end{itemize}
\end{frame}

\begin{frame}{I Prefer Python}
	\begin{itemize}
		\item No ``black magic'' 
		\begin{itemize}
			\item (well less of it is immediately apparent)
		\end{itemize}
		\item Easy to teach concepts and pseudocode 
		\item Syntax is easy and forgiving
		\item Dictionaries
	\end{itemize}
\end{frame}



\subsection{General Changes}

\begin{frame}{Other Suggestions}
	\begin{itemize}
	\item Hard concepts earlier
	\item More complete examples of code before we make students code.
	\item Experiment
		\begin{itemize}
			\item Active Learning
			\item Flipped Classroom
		\end{itemize}
	\end{itemize}
		
\end{frame}

\begin{frame}{Live Coding}
	\begin{itemize}
		\item Learn from mistakes
		\item Demonstrate thought process
		\item Experimentation
		\item Provides worked examples 
		
	\end{itemize}
\end{frame}


\section{Research Background}

\subsection{Distributed Computing and Challenges}
\begin{frame}{Challenges of Distributed Computing}
	Distributed Computing platforms experience these challenges:
	\begin{description}
		\item<1->[Scalability] As the network grows, more resources are spent on maintaining and organizing the network. 
		\note<1>{Remember, computers aren't telepathic. There's always an overhead cost.  It will grow.   The challenge of scalability is designing a protocol  in which this cost grows at an extremely slow rate. 
			For example, a single node keeping track of all members of the system might be a tenable situation up to a certain point, but eventually, the cost becomes too high for a single node.}
		\item<2->[Fault-Tolerance] As more machines join the network, there is an increased risk of failure. \note<2>{Failure Hardware failure is a thing that can happen. Individually the chances are low, but this becomes high when we're talking about millions of machines.  Also, what happens in a P2P environment.  Nodes leaving is treated as a failure.}  
		\item<3->[Load-Balancing] Tasks need to be evenly distributed among all the workers. \note<3>{If we are splitting the task into multiple parts, we need some mechanism to ensure that each worker gets an even (or close enough) amount of work.}
	\end{description}
	
\end{frame}


\subsection{What Are Distributed Hash Tables?}

\begin{frame}{Distributed Key/Value Stores}
	\textbf{Distributed Hash Tables }are mechanisms for storing values associated with certain keys.
	\begin{itemize}
		\item Values, such as filenames, data, or IP/port combinations are associated with keys.
		\item These keys are generated by taking the hash of the value.
		\item We can get the value for a certain key by asking any node in the network.
	\end{itemize}
\end{frame}


\note{At their core, Distributed Hash Tables are giant lookup  tables.  Given a key, it will return the value associated with that key, if it exists.
	These keys, or hash keys, are generated by a hash function, such as SHA1 or MD5.
	These hash functions use black magic using prime numbers and modular arithmetic to return a close to unique identifier associated with a given input.
	The key about the keys is the same input will always produce the same output.
	From a probability standpoint, they are distributed uniformly at random.
}



\begin{frame}{How Does It Work?}
	
	\begin{itemize}
		\item DHTs organize a set of nodes, each identified by an \textbf{ID}. 
		\item Nodes are responsible for the keys that are closest to their IDs.
		\item Nodes maintain a small list of other peers in the network.
		\begin{itemize}
			\item Typically a size $ \log(n)$ subset of all nodes in the network.
		\end{itemize}
		\item Each node uses a very simple routing algorithm to find a node responsible for any given key.  %Explain what that is
	\end{itemize}
\end{frame}

\note[itemize]{
	
	\item We'll explain in greater detail later, but briefly:
	\item DHTs are composed of a set of nodes, each identified by a hashed ID
	\item Each node is responsible for the key/value pairs that fall within its zone of responsibility, which can be thought of as the nodes closest to it/.
	\item Nodes keep a list of other nodes in the network, composed of peers that are close to it in terms of ID, and shortcuts to achieve sublinear lookup time.
	\item To lookup a particular key, the node asks ``Am I responsible for this key?'' If yes, yay! If no, I forward this message to the peer I know who I think is best able answer this question.
}

\begin{frame}{Current Applications}
	Applications that use or incorporate DHTs:
	\begin{itemize}
		\item P2P File Sharing applications, such as BitTorrent.
		\item Distributed File Storage.
		\item Distributed Machine Learning.
		\item Name resolution in a large distributed database.
	\end{itemize}
\end{frame}

\note[itemize]{
	\item DHTs weren't necessarily designed with large-scale P2P applications in mind, but that use case was never ignored.
	\item BitTorrent uses a DHT, called MainlineDHT, and has about 20 million nodes active at any given time, and has a churn of about 50 \% per day.
	
}

%\begin{frame}{Strengths of DHTs }
%	DHTs are designed for large P2P applications, which means they need to be (and are):
%	\begin{itemize}
%		\item[Scalable] 
%		\begin{itemize}
%			\item Each node knows a \emph{small} subset of the entire network.
%			\item Join/leave operations impact very few nodes.
%		\end{itemize}
%		\item[Fault-Tolerant] 
%		\begin{itemize}
%			\item The network is decentralized.
%			\item DHTs are designed to handle \alert{churn}.
%
%		\end{itemize}
%		\item[Load-Balancing]
%		\begin{itemize}
%			\item Consistent hashing ensures that nodes and data are close to evenly distributed.
%			\item Nodes are responsible for the data closest to it.
%		\end{itemize}
%	\end{itemize}
%	
%\end{frame}
\subsection{Why DHTs and Distributed Computing}

\begin{frame}{Strengths of DHTs }
	DHTs are designed for large P2P applications, which means they need to be (and are):
	\begin{itemize}
		\item Scalable
		\item Fault-Tolerant
		\item Load-Balancing
	\end{itemize}
	
\end{frame}

\note[itemize]{
	\item Scalability 
	\begin{itemize}
		\item Each node knows a \emph{small} subset of the entire network.
		\item Join/leave operations impact very few nodes.
		\item The subset each node knows is such that we have expected $ \lg(n) $ lookup
	\end{itemize}
	
}

\note[itemize]{
	\item Fault-Tolerance 
	\begin{itemize}
		\item The network is decentralized.
		\item DHTs are designed to handle churn.
		\item Because Joins and node failures affect only nodes in the immediate vicinity, very few nodes are impacted by an individual operation.
	\end{itemize}
	\item Load Balancing
	\begin{itemize}
		\item Consistent hashing ensures that nodes and data are close to evenly distributed.
		\item This allows a large-scale failure, like California being hit by a massive earthquake, to be absorbed throughout the network, rather than a contiguous portion being knocked out.
		\item Nodes are responsible for the data closest to it.
		\item The space is large enough to avoid Hash collisions.
	\end{itemize}	 
}


\subsection{Results}

\begin{frame}{Research Projects}
	\begin{itemize}
		\item ChordReduce
		\item UrDHT
	\end{itemize}
\end{frame}



\begin{frame}{Other Research}
	\begin{itemize}
		\item Mapped DHTs to Voronoi/Delaunay
		\item Created a greedy heuristic for calculating Voronoi regions
		\item Sybil Attack Analysis
	\end{itemize}
\end{frame}

\section{Incorporating Research in Class}

\subsection{Networking}

\begin{frame}{P2P}
	\begin{itemize}
		\item Networking typically glazes over P2P applications
		\item Presents another way of discussing fault tolerance
		\item DHTs are the backbone for most P2P networks
		\item Generalized model for distributed computing
		\item Could be used as part of a Special Topics for distributed and decentralized systems.
	\end{itemize}
\end{frame}




\begin{frame}{MANETs}
\begin{itemize}
	\item Voronoi Regions are useful for sleep scheduling
\end{itemize}
\end{frame}

\subsection{Security}


\begin{frame}{Cryptographic Hashes}
	\begin{itemize}
		
	\item Essential part of DHTs
	\item Ties into:
	\begin{itemize}
		\item Message Authentication
		\item Signatures
	\end{itemize}
	\end{itemize}
	
\end{frame}


\begin{frame}{Security}
	\begin{itemize}
		\item Nothing is secure
		\item Human elements are the weakest
		\item CIA triad
		\begin{itemize}
			\item Little focus paid to "A"
			
		\end{itemize}
		\item Security of distributed systems is an aside.
		\item Systems are not more secure just because they are decentralized
		
	\end{itemize}
	
\end{frame}


\begin{frame}{Attacks on Distributed Systems}
	\begin{itemize}
		\item Sybil Attack
		\begin{itemize}
			\item Attacker aggressively creates and maintains multiple identities
			\item Goal is to become at least 33\% of network
			\item Prefer majority to have complete control.
		\end{itemize}
		\item Eclipse Attack is like Sybil but:
		\begin{itemize}
			\item Goal is to prevent ``good'' nodes from communicating directly
			\item Occlude or eclipse all connections
			\item Attack can poison routing tables and force bad updates
		\end{itemize}
		\item How do you prevent these attacks?
		
	\end{itemize}
	
\end{frame}

\end{document}
