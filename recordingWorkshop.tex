\documentclass[10pt,letterpaper]{article}
\usepackage[utf8]{inputenc}

\usepackage{amsmath}
\usepackage{amsfonts}
\usepackage{amssymb}
\usepackage{graphicx}
\usepackage{titlesec}


\usepackage[letterpaper, margin=1in]{geometry}



\titleformat*{\section}{\normalsize\bfseries}
\titleformat*{\subsection}{\normalsize\bfseries}
\titleformat*{\subsubsection}{\normalsize\bfseries}
\titleformat*{\paragraph}{\normalsize\bfseries}
\titleformat*{\subparagraph}{\normalsize\bfseries}
\begin{document}
	
\section*{Workshop Title: How to Record Your Lectures for Fun and Possible Profit}

\section*{Presenters:}
\textbf{Andrew Rosen} (Contact Person)\\
Department of Computer and Information Sciences\\
Temple University\\
Information Science 1925 N. 12th St, Rm. 349 \\
Philadelphia, PA 19122\\
Phone: 678-665-1415\\
andrew.rosen@temple.edu
\newpage
\section*{Abstract}
This workshop teaches participants how to utilize their laptops and open source software to create high quality lecture recordings (although we will mention proprietary alternatives).
We demonstrate how to use Open Broadcasting Software to record lectures and do screen captures, as well as use Kdenlive for basic video editing.
We show how to use more advanced features, such as chroma key with a green screen, which can make your lectures look more professional.
We discuss the various benefits and issues that surround various hosting options.

Our workshop can be roughly split into four parts: presenting and using recording software, exploring hosting options, basic editing, 
The presenter will record the the workshop and make the recording available to the participants.
This workshop caters to all levels of recording experience and all goals, be it the instructor who simply wants to capture their live-coding, or teachers looking to create professional looking lectures at home for flipped classrooms.

Finally, we will discuss various ways to monetize your content.

\section*{Advertisement}
Want to record your classroom lectures or create recordings for a flipped classroom?
Overwhelmed by the options you have?
Don't know where to start?
Does the software your school contracted with not work with your operating system?
Just want to learn more in depth about how to make high quality lectures in general?
Then this workshop is the right place for you!

This workshop will show you how to use your laptop to record lectures for any purpose and how to build up to making high quality, professional looking recordings. 
Participants should bring a laptop.
Linux is supported.

\section*{Significance and Relevance of the Topic:}

Motivation to record lectures have become more pronounced in  recent years, with many schools pushing to create online programs and courses to remain competitive.
Furthermore, declining vaccination rates in the United States have led to greater fears of outbreaks among student populations might become more common, such the measles scare at UCLA  and the mumps outbreak at Temple University in 2019.
Providing recorded lectures would help mitigate the effects of a quarantine situation on student outcomes.


At the same time, laptops have become powerful enough to work as stand alone recording devices, obviating the hard requirement of external microphones and cameras for recording.
Modern laptops are equipped with a processor powerful enough to run recording software and a resource-hungry IDE, have at least a 1080p screen, and come standard with at least a 720p webcam.

\section*{Expected audience:}  The expected audience is CS educators who are interested in recording their lectures or who already record their lectures, but want to learn more about option that are available and how to make the recordings more professional.

\section*{Space and Enrollment restrictions:} 
This workshop is extremely flexible. 
Enrollment can be limited to 30 participants, but this is not a hard limit depending on interest.
Ideally there should be enough room to move among the participants and aid them with any technical issues.
The only hard requirement is a projector or display of some kind, so the presenter can show how to use the software.

\newpage

\section*{Expertise of Presenter:}
Andrew Rosen has been utilizing lecture recordings since 2016.  
He began recording his lectures for students taking a data structures class remotely at Temple University's Japan campus.
Since the groundwork for recording one class had already been done, he began to record all his classes and make all the recording available to students.
He initially used Panopto to record, but eventually switched to recording using OBS and uploading the lectures to YouTube after some reported student issues and a desire to have more control over the recording process.
This eventually led to making lectures for a flipped data structures class and taking the lead in creating one of the first online course's for Temple's online Masters in IST program. 
\section*{Rough Agenda: }
\begin{description}
	\item[30 minutes] Introduction to OBS and its features.
	\item[20 minutes] Participant experimentation with video recording.
	\item[15 minutes] Overview of hosting on Youtube with special focus on accessibility.
	\item[15 minutes] Overview of hosting on Vimeo with special focus on collaboration.
	\item[15 minutes] Break.
	\item[15 minutes] Introduction to video editing with Kdenlive.
	\item[30 minutes] Cutting mistakes and using wipes.
	\item[15 minutes] Using a greenscreen.
	\item[25 minutes] Discussion and technical support.
\end{description}
\section*{Audio/Visual and Computer requirements:  }

The workshop requires the following:
\begin{itemize}
	\item Internet: wireless access for participants to download software
	\item Power Outlets: would be ideal
	\item Projector: HDMI compatible
	\item Computers: Laptop required. 
	\item Software: provided by presenter
\end{itemize}

\end{document}