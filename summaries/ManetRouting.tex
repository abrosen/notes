\documentclass[a4paper]{article}
\usepackage[utf8]{inputenc}
\usepackage{amsmath}
\title{Manet Routing Algorithms}
\author{Andrew Rosen}
\date{\today}
\begin{document}

\maketitle

\section{Terminology}

Routing protocols for MANETs can be classified as the following types \cite{singh2012ant} \cite{correia2010simple}  [Deepalakshmi and Radgkrishnan]:

\begin{itemize}
	\item Proactive or table-driven routing algorithms.  Routing tables are calculated in advanced and stored in tables.  The mobile nature of MANETs make it difficult to maintain the network
	\item Reactive and on demand routing.   No route data is maintained
	\item Hybrid
	\item Opportunistic/Epidemic
\end{itemize}


\section{Geographic Routing }
Began with \cite{ratnasamy2002ght}




\section{LORD}
How do we search for data in dense and mobile networks?  Flooding techniques are, as a rule, not energy efficient.  Neither are local broadcast methods with the additional drawback of being unreliable for data discovery.  Geographic routing, whihc maps files to locations, is interesting but with mobile nodes, maintanence costs become an issue.


Shen et al.\ present LOcality-based distRibuted Data search system.\cite{shen2013scalable}.  LORD maps metadata of files to a geographic region and stores the metadata in a subset of the nodes\footnote{All nodes in a region is a subset of all nodes in a region} in the region.  Nodes only update their data mapping when they move between regions.  Nodes only need to use the located region (not their exact location) and their angle information.

Of note is the dijointness between the storage of the metadata and the file itself.  When a node stores it's metadata for a file, it maps the metadata to a key, which is $(x,y)$ geographic coordinates in this case, and that metadata is stored in the region covering those coordinates.


\begin{description}
	\item[Cases] LORD assumes operation in a dense and mobile environment with landmarks.  Nodes in the network are ``idependently and identically distributed.''  Nodes are not equipped with GPS but do have Angle of Arrival information in some form.
	\item[Patitioning] Regions are partitioned by landmarks.  Not sure how, but assume Voronoi.  Nodes are members of the region if the region's landmark is the closest landmark
	\item[Mapping] The metadata of files is hashed using LSH \cite{datar2004locality} to produce some location $(x,y)$  and stored at the region responsible for that location. 
	
	Nodes
\end{description}


\section{Swarm Intelligence}

\section{Key Concepts}

\section{Ara}
Ara \cite{ara-ants} is short for Ant-Colony-Based Routing Algorithm; this paper is an application of ant communication applied on-demand ad-hoc routing in Mobile Ad-Hoc Networks (MANETs).
Ant colony algorithms are a type of swarm intelligence where simple workers following simple rules produce complex behaviors.  


A key feature is that ants don't directly communicate with each other; ants rely on stigmergy, whereby communcation is established by modifying the environment.
When an individual ant is seeking food for the nest, they deposit pheromones along the path.  An important aspect of pheromones is that they dissipate over time.  In addition, ants carrying food back to the nest deposit greater amounts of pheromone, avoiding dead end behaviors.  This allows for the ant colony to find the shortest path by choosing the route with the greater concentrations of pheromones. 

The behavior is modelled by a graph $G = (V, E)$ with ants trying to get from from a source to a destination node. Edges have a pheromone strength $\varphi_{i,j}$, the stronger the pheromone, the more likely the ant is to traverse the edge from $i$ to $j$.  When an ant travels along an edge, that edge's pheromone is incremented.  In addition, when an ant travels alopng the edge, all edges in the routing table lose some of the pheromone strength (This is a simplified version of the process below).
\[
\varphi_{i,j} =
\begin{cases}
    (1-q) \cdot \varphi_{i,j} + \Delta\varphi & \text{if j is the next node}  \\
    (1-q) \cdot \varphi_{i,j}       & \text{else}
\end{cases}
\]

Ants (or messages) will positively reinforce the correct path, even in a changing topology.  There's no reason why other attributes, such as signal strength can't be used to help impact a local node's table.

The actual algorithm:
\begin{itemize}
    \item \textit{Route Discovery}, in which the routing table tuples \texttt{destination, next hop, pheromone} are established via initial broadcast messages.
    \item \textit{Route Maintenence} multiplicatively decreases the pheromone strength of entries in the routing table each second, but entries increase whenever a message is forwarded along that entry (the decrease and increase are seperated in ARA).
    \item The algorithm handles \textit{Loops} by having nodes keep track of message identifiers and sending a special message to any node fowarding a duplicate. 
    \item  The \textit{fault tolerance} aspect still needs some more description, but failure to communicate sets the entry's  $\varphi$ to 0.  When a node renenters the network, it reinitiates route discovery.
\end{itemize}


In simulations the messages took the shortest path more often than in DSR or AODV.  However, the author's conclusions were that the algorithm still needed more work.


\section{SARA - Simple Ant Routing Algorithm}


\subsection{Controlled Neighbor Broadcast}
Normally route discovery works by the mechanism of ``Forward Ant'' or FANT being broadcast by the source node, which is then rebroadcast by all nodes in the network.  Controlled Neighbor Broadcast (CNB) is used to help with this process, by limiting the rebroadcasting of the FANT to a single node.

When a node with a route to the destination receives a FANT, it broadcasts a BANT back.

\section{RatNest}

Of note is that RatNest is for a specific military survailence application.  Nodes are assumed to have essential unlimited power.
RatNest is not "bio-inspired" in the same way the other algorithms are, the link to naked mole-rats (eusocial mammals) is for aiding the metaphor

\section{Distributed Hash Tables on MANETs}
One of the key features that make a DHT ideally suited for MANETs is the WYZYG nature of the network.  When you create an overlay on computers over the Internet, each hop on the overlay is actually multiple hops on the ``underlay''.  This is never the case on MANETs.





\bibliography{notes}
\bibliographystyle{plain}
\end{document}


























